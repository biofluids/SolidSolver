%
\section{Finite Element Formulation}
\label{formulation}
In this section, we introduce the finite element formulation to solve static problems for hyperelastic materials. Starting from the basic principles, we will construct the equations to be solved using the stress tensors we derived in Section \ref{general}. Furthermore, we will show how to linearize the nonlinear equations in order to solve them. The final linear equation system is given in matrix form which is ready to be solved with proper matrix solver. In particular, the mixed formulation will be discussed in detail.

%
\subsection{Principle of Virtual Work} \label{PVW}
The variational approach is the cornerstone for finite element methods. The most fundamental variational principle that origins from Newton's law is the principle of virtual work. Neglecting the kinematic term, it can be written as:
\begin{subequations} \label{virtualwork}
\begin{align}
\delta{W_\mathrm{int}} &= \delta{W_\mathrm{ext}} \label{basic} \\
\delta W_\mathrm{int}(\bold{u}, \delta\bold{u}) &= \int_\Omega \boldsymbol\sigma : \delta\bold{e}dV = \int_{\Omega_{0}}\bold{S} : \delta\bold{E}dV \label{Wint} \\
\delta W_\mathrm{ext}(\bold{u}, \delta\bold{u}) &= \int_\Omega\bold{b}\cdot\delta\bold{u}dV +  \int_{\partial\Omega}\overline{\bold{t}}\cdot\delta{\bold{u}}dS = \int_{\Omega_0}\bold{B}\cdot\delta\bold{u}dV +  \int_{\partial{\Omega_0}}\overline{\bold{T}}\cdot\delta{\bold{u}}dS \label{Wext1}
\end{align}
The internal work is done by Cauchy stress $\boldsymbol\sigma$ along the virtual Euler-Almansi strain $\delta\bold{e}$ about region $\Omega$ and the external work is done by the body force $\bold{b}$ and the surface traction $\overline{\bold{t}}$, along the virtual displacement $\delta{\bold{u}}$ about region $\Omega$ and its boundary surface $\partial\Omega$, respectively. They can be mapped to the reference configuration $\Omega_0$ using corresponding stress and strain tensors, and external body forces and tractions. 

If the external loading is given as fixed traction in the reference configuration, the virtual external work is constant. An example for this type of loading is a tension test where the external force with respect to reference configuration is given as a constant. Another type of loading, which is more frequently encountered in biomechanics is the fixed pressure loading.
In this case, the pressure on the current boundary surface $\partial\Omega$ is constant, i.e. $\overline{\bold{t}} = \boldsymbol{\sigma} \cdot \bold{n} = p_0\bold{n}$ where $p_0$ is a given constant, $\bold{n}$ is the unit normal vector of the current surface. The external virtual work done by the constant pressure $p_0$ along the virtual displacement $\delta{\bold{u}}$ is defined as:
\begin{align} 
\delta{W_\mathrm{ext}}(\bold{u}, \delta{\bold{u}}) &= p_0\int_{\partial\Omega} \bold{n}\cdot\delta{\bold{u}}dS\label{Wext2}
\end{align}
\end{subequations}

For the second type of loading as in Equation \ref{Wext2}, the external work is changing with the current configuration due to the change of normal $\bold{n}$ and integration boundary $\partial\Omega$. To evaluate it, let $\bold{x}$ be an arbitrary point on the surface of a 3D element, $\xi$ and $\eta$ be the parent coordinates on that surface. $\Omega_{\xi}$ represents the parameter plane characterized by $\xi$ and $\eta$. Then the unit normal vector of that surface $\bold{n}$ as well as the infinite small area $dS$ can be expressed as:

\begin{equation}
\bold{n} = \frac{  \frac{\partial{\bold{x}}}{\partial{\xi}} \times  \frac{\partial{\bold{x}}}{\partial{\eta}} }{\left| \frac{\partial{\bold{x}}}{\partial{\xi}} \times  \frac{\partial{\bold{x}}}{\partial{\eta}} \right|}
\end{equation}
\begin{equation}
dS = \left|\frac{\partial{\bold{x}}}{\partial{\xi}} \times  \frac{\partial{\bold{x}}}{\partial{\eta}}\right| d\xi{d\eta}
\end{equation}
With these two equations, $\delta W_\mathrm{ext}$ in Equation \ref{Wext2} for a constant pressure applied on the current boundary can be expressed as:

\begin{equation}
\delta{W_\mathrm{ext}}(\bold{u}, \delta{\bold{u}}) = p_0\int_{\Omega_{\xi}}  \left(\frac{\partial{\bold{x}}}{\partial{\xi}} \times  \frac{\partial{\bold{x}}}{\partial{\eta}}\right) \cdot\delta{\bold{u}}d\xi{d\eta}
\end{equation}

%
\subsection{Principle of Stationary Potential Energy}
In the principle of virtual work, the stresses are considered independently from the strains. It is general as it does not involve any assumption on the material behavior. The principle of stationary potential energy, on the other hand, is built upon the assumptions that the mechanical system is conservative, where there exists an energy functional $\Pi$ for both the stresses and the loadings whose sum is a constant. The formulation based on energy functionals is very useful in many areas such as the mathematical optimization problems and the Hamiltonian systems in physics. In the displacement-based formulation, $\Pi$ is a function of displacement $\bold{u}$ only, and it leads to the same equations as the virtual work balance. While in the displacement/pressure mixed formulation, $\Pi$ is a function of both displacement $\bold{u}$ and pressure $p$. It treats pressure as an independent variable and has advantages of avoiding locking and ameliorate the ill-condition of the stiffness matrix in certain situation.

In displacement-based formulation, the potential energy is expressed as:
\begin{subequations}
\begin{align}
\Pi(\bold{u}) &= \Pi_\mathrm{int}(\bold{u}) + \Pi_\mathrm{ext}(\bold{u}) \\
\Pi_\mathrm{int}(\bold{u}) &= \int_{\Omega_{0}}\Psi(\bold{F}(\bold{u}))dV  \label{pint} \\
\Pi_\mathrm{ext}(\bold{u}) &=  - \int_{\Omega_0}\bold{B}\cdot\bold{u}dV -  \int_{\partial{\Omega_0}}\overline{\bold{T}}\cdot{\bold{u}}dS
\label{pext} 
\end{align}
\end{subequations}

Our objective is to find the state of equilibrium for which the system is stationary. In the displacement-based formulation, this means the directional derivative with respect to the displacement $\bold{u}$ to vanish in all directions of $\delta{\bold{u}}$.
\begin{equation} \label{equilibrium}
D_{\delta\bold{u}}\Pi(\bold{u}) = D_{\delta\bold{u}}\Pi_\mathrm{int}(\bold{u}) + D_{\delta\bold{u}}\Pi_\mathrm{ext}(\bold{u}) = 0
\end{equation} 
In other words, we require the first variation of the total energy potential $\delta\Pi$ to vanish:
\begin{equation} \label{potential}
\delta\Pi(\bold{u}, \delta\bold{u}) =\delta\Pi_\mathrm{int}(\bold{u}) + \delta\Pi_\mathrm{ext}(\bold{u}) = 0
\end{equation}
Here we adopt the convention in \cite{Holzapfel} using $D_{\Delta{x}}f(x, y, \mathrm{etc.})$ to represent the directional derivative of $f$ which is a function of $x$, $y$, etc. along the direction of $\Delta{x}$, where $x$, $y$, etc. can be a scalar or a tensor of any other order. The directional derivative $D_{\Delta{x}}f(x, y, \mathrm{etc.})$ is equivalent to the first variation of functional $f$ with respect to $x$.
Equation \ref{equilibrium} or \ref{potential} is the equation to be solved in the displacement-based formulation. It is easy to prove the equivalence between the principle of directional derivative of the potential energy and the virtual work:

\begin{equation}
D_{\delta\bold{u}}\Pi_\mathrm{int}(\bold{u}) = \delta W_\mathrm{int}(\bold{u}, \delta\bold{u}) \quad
D_{\delta\bold{u}}\Pi_\mathrm{ext}(\bold{u}) = - \delta W_\mathrm{ext}(\bold{u}, \delta\bold{u})
\end{equation}

In the mixed formulation, $\Pi_\mathrm{int}(\bold{u}, p) = \int_{\Omega_{0}}\Psi(\bold{F}(\bold{u}), p)dV$ involves two independent variables, $\bold{u}$ and $p$. The state of equilibrium is with respect to both of them. Our objective equation becomes:
\begin{equation} \label{target}
D_{\delta\bold{u}}\Pi(\bold{u}, p) = 0 \quad D_{\delta{p}}\Pi(\bold{u}, p) = 0
\end{equation}
To summarize, in displacement-based formulation, our goal is to find an equilibrium state for total potential energy with respect to $\bold{u}$; in the mixed formulation, we need to find an equilibrium state with respect to both $\bold{u}$ and $p$. To solve these equations, we have to linearize them first. That is, to find the second variation of the potential energy.


%
\subsection{Linearization of the Principle of Stationary Potential Energy}
In the displacement-based formulation, Equation \ref{equilibrium} is solved using Newton's method. At each iteration, a residual $\bold{r}$ of Equation \ref{equilibrium} is evaluated. Then the increment of the displacement $\Delta\bold{u}$ is obtained from solving linear system $\bold{A} \cdot \Delta\bold{u} = -\bold{r}$ where $\bold{A}$ is the tangent stiffness matrix. It is based on Equation \ref{Wint}, by means of differentiation:

\begin{equation} \label{Kint}
D^2_{\delta\bold{u}, \Delta\bold{u}}\Pi_\mathrm{int}(\bold{u}) = D_{\Delta{\bold{u}}}\delta{W_\mathrm{int}}(\bold{u}, \delta{\bold{u}}) = \int_{\Omega_0}(\nabla_{\bold{X}}\delta\bold{u} : \nabla_{\bold{X}}\Delta\bold{u}\bold{S} + \bold{F}^T\nabla_{\bold{X}}\delta{\bold{u}} : \mathbb{C} : \bold{F}^T \nabla_{\bold{X}}\Delta\bold{u})dV
\end{equation}
The first term in the integration is called the initial stress contribution since $\bold{S}$ characterizes the initial stress at every increment. The second term is called the material contribution as $\mathbb{C}$ characterizes the material response to the displacement.

As discussed in Section \ref{PVW}, for the first type of external loading as in Equation \ref{Wext1}, the external work is independent of the displacement. So the variation vanishes:

\begin{equation} \label{Kext}
D^2_{\delta\bold{u}, \Delta\bold{u}}\Pi_\mathrm{ext}(\bold{u}) = - D_{\Delta{\bold{u}}}\delta{W_\mathrm{ext}}(\bold{u}, \delta{\bold{u}}) = 0
\end{equation}
For the second type of external loading the linearization is obtained by a straight-forward derivation with respect to $\Delta\bold{u}$, the increment of displacement. 
\begin{equation}  \label{Kext2}
D^2_{\delta\bold{u}, \Delta\bold{u}}\Pi_\mathrm{ext}(\bold{u}) = - D_{\Delta{\bold{u}}}\delta{W_\mathrm{ext}}(\bold{u}, \delta{\bold{u}}) = - p_0\int_{\Omega_{\xi}}  \left(\frac{\partial \Delta\bold{u}}{\partial{\xi}} \times
\frac{\partial\bold{x}}{\partial\eta} - \frac{\partial \Delta\bold{u}}{\partial{\eta}} \times
\frac{\partial\bold{x}}{\partial\xi}    \right) \cdot \delta{\bold{u}} d\xi d\eta
\end{equation}
Therefore with Equations \ref{Kint} and \ref{Kext} or \ref{Kext2} we are able to obtain the matrix form for displacement-based formulation:
\begin{equation} \label{matrix}
\left( D^2_{\delta\bold{u}, \Delta\bold{u}}\Pi_\mathrm{int}(\bold{u}) + D^2_{\delta\bold{u}, \Delta\bold{u}}\Pi_\mathrm{ext}(\bold{u})  \right) \Delta\bold{u} = - (D_{\delta\bold{u}}\Pi_\mathrm{int}(\bold{u}) + D_{\delta\bold{u}}\Pi_\mathrm{ext}(\bold{u}) )
\end{equation}
The terms in the parenthesis on the lefthand side is the tangent stiffness matrix, the righthand side is the opposite of the residual of the system potential energy.

The matrix form for the mixed formulation in Equation \ref{target} is more complicated as it requires the linearization with respect to both $\bold{u}$ and $p$. The incremental matrix form becomes:

\begin{equation} \label{linear}
\begin{bmatrix}
D_{\delta\bold{u}, \Delta\bold{u}}^2 \Pi(\bold{u}, p)  && D_{\delta\bold{u}, \Delta{p}}^2 \Pi(\bold{u}, p)  \\ D_{\delta{p}, \Delta\bold{u}}^2 \Pi(\bold{u}, p)  && D_{\delta{p}, \Delta{p}}^2 \Pi(\bold{u}, p) 
\end{bmatrix}
\begin{bmatrix}
\Delta\bold{u} \\ \Delta{p}
\end{bmatrix}
= -
\begin{bmatrix}
D_{\delta\bold{u}}\Pi(\bold{u}, p) \\ D_{\delta{p}}\Pi(\bold{u}, p) 
\end{bmatrix}
\end{equation}
We first examine the righthand side. For incompressible materials, recall Equations \ref{energy_split} and \ref{Lagrange}, the potential energy is expressed as the sum of the internal and external energies:
\begin{equation} \label{split3}
\Pi(\bold{u}, p) = \int_{\Omega_0} \left[ p(J(\bold{u}) - 1) + \Psi_\mathrm{iso}(\overline{\bold{C}}(\bold{u}))\right] dV + \Pi_\mathrm{ext}(\bold{u})
\end{equation}
Using the chain rule, the righthand side of Equation \ref{linear} can be expressed as:

\begin{equation}\label{rhs1}
D_{\delta\bold{u}}\Pi(\bold{u}, p) = \int_{\Omega_0}\left( J(\bold{u})p\bold{C}^{-1}(\bold{u}) + 
2\frac{\partial{\Psi_\mathrm{iso}(\overline{\bold{C}}(\bold{u}))}}{\partial{\bold{C}}}  \right) : \delta\bold{E}(\bold{u}) dV + D_{\delta\bold{u}}\Pi_\mathrm{ext}(\bold{u})
\end{equation}
\begin{equation}\label{rhs2}
D_{\delta{p}}\Pi(\bold{u}, p) = \int_{\Omega_0} (J(\bold{u}) - 1 )\delta{p} dV
\end{equation}
For the lefthand side, $D_{\delta\bold{u}, \Delta\bold{u}}^2 \Pi(\bold{u}, p)$ has already been derived for the displacement-based formulation as the sum of Equation \ref{Kint} and Equation \ref{Kext} (the first type of external loading where $\Pi_\mathrm{ext} = 0$) or Equation \ref{Kext2} (the second type of external loading where a constant pressure is applied on the current boundary), i.e.
\begin{equation} \label{lhs0}
D_{\delta\bold{u}, \Delta\bold{u}}^2 \Pi(\bold{u}, p) =  \int_{\Omega_0}(\nabla_{\bold{X}}\delta\bold{u} : \nabla_{\bold{X}}\Delta\bold{u}\bold{S} + \bold{F}^T\nabla_{\bold{X}}\delta{\bold{u}} : \mathbb{C} : \bold{F}^T \nabla_{\bold{X}}\Delta\bold{u})dV  \end{equation}
When pressure boundary condition is applied, it becomes:
\begin{equation} \label{lhs1}
\begin{split}
D_{\delta\bold{u}, \Delta\bold{u}}^2 \Pi(\bold{u}, p) 
&= \int_{\Omega_0}(\nabla_{\bold{X}}\delta\bold{u} : \nabla_{\bold{X}}\Delta\bold{u}\bold{S} + \bold{F}^T\nabla_{\bold{X}}\delta{\bold{u}} : \mathbb{C} : \bold{F}^T \nabla_{\bold{X}}\Delta\bold{u})dV  \\
&-  p_0\int_{\Omega_{\xi}}  \left(\frac{\partial \Delta\bold{u}}{\partial{\xi}} \times
\frac{\partial\bold{x}}{\partial\eta} - \frac{\partial \Delta\bold{u}}{\partial{\eta}} \times
\frac{\partial\bold{x}}{\partial\xi}    \right) \cdot \delta{\bold{u}} d\xi d\eta
\end{split}
\end{equation}
It is a little tricky but not complicated to obtain that:
\begin{equation} \label{lhs2}
D_{\delta\bold{u}, \Delta{p}}^2 \Pi(\bold{u}, p) = \int_{\Omega_0} J(\bold{u})\Delta{p}(\mathrm{div}\delta{\bold{u}})dV
\end{equation}
\begin{equation} \label{lhs3}
D_{\delta{p}, \Delta\bold{u}}^2 \Pi(\bold{u}, p) = \int_{\Omega_0} J(\bold{u})(\mathrm{div}\Delta{\bold{u}})\delta{p}dV
\end{equation}
Since $\Pi$ is a first order function with respect to $p$ in the mixed formulation, we have:
\begin{equation} \label{lhs4}
D_{\delta{p}, \Delta{p}}^2 \Pi(\bold{u}, p) = 0
\end{equation}
Inserting Equations \ref{rhs1} and \ref{rhs2} to the righthand side, and Equations \ref{lhs1} to \ref{lhs4} to the lefthand side of Equation \ref{linear} we can obtain the tangent stiffness matrix. Since $p$ is arbitrary for incompressible material, we need additional boundary conditions for it to solve Equation \ref{equilibrium}.

While for nearly incompressible materials, the incompressibility condition is no longer strictly enforced. Recall Equations \ref{penalty} and \ref{pressure}, Equation \ref{rhs2} is relaxed to:

\begin{equation} \label{relation}
D_{\delta{p}}\Pi(\bold{u}, p) = \int_{\Omega_0}\left(  \frac{dG(J(\bold{u}))}{dJ} - \frac{p}{\kappa} \right)\delta{p}dV = 0
\end{equation}
which reflects the fact that $p$ is not arbitrary but related to $\bold{u}$.  
Accordingly, the second variation on $p$ is no longer $0$. Equation \ref{lhs4} is modified as:
\begin{equation} \label{lhs42}
D_{\delta{p}, \Delta{p}}^2 \Pi(\bold{u}, p) = - \int_{\Omega_0}\frac{1}{\kappa}\Delta{p}\delta{p}dV
\end{equation}
As a result, for nearly incompressible materials, the righthand side of Equation \ref{linear} uses Equations \ref{rhs1} and \ref{relation}, the lefthand side uses Equations \ref{lhs1}, \ref{lhs2}, \ref{lhs3} and \ref{lhs42}. The linear equation is complete, we do not need additional information on $p$.





