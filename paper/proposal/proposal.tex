\documentclass[12pt]{article}
\usepackage{fullpage}
\usepackage{graphicx}
\usepackage{subfigure}
\usepackage{amsmath}
\usepackage{amsfonts}
\usepackage{amssymb}
\usepackage{amsbsy}
\usepackage{setspace}
\usepackage{fullpage}
\usepackage{wrapfig}
\usepackage{cite}

\usepackage[skip=2pt,font=footnotesize]{caption}

\begin{document}
\begin{center}
 \LARGE\textbf{Coupled Growth and Fluid Medium for Time-Lapse Modeling}
\end{center}

\tableofcontents
%\newpage
\clearpage

\section{BACKGROUND AND MOTIVATION - 2 pages}

\subsection{Growth of Cells and Tissues - 0.5 page}
In the mechanical view, the most distinguishable feature of living matters is their ability to respond to environmental changes while common engineering materials cannot. Growth happens within days, weeks, months and years. Hard tissues increase their density and grow strong. One common example is that tennis players have a significantly dominant arm. While training increases the strength of bones, it is also the cause of chronic pain that many athletes suffer. Soft tissues increase their volume and grow large. For instance, myocardial infarction could cause the infarcted ventricle to dilate in response to volume overload, attempting to maintain normal cardiac output. Meanwhile the pumping efficiency may be reduced, resulting in the medical condition called cardiac dilation. Growth plays a key role in many biological and medical processes including the cardiovascular disease, tumor growth, restenosis, trauma healing, twisted bone growth and so on. Microscopically, growth is the result of the reproduction and rearrangement of microstructures. Macroscopically, growth is the adaption of materials to the mechanical loading governed by certain laws. Therefore, understanding the growth of cells and tissues requires inputs from biology, mechanics, and mathematics communities.

\subsection{Review of Mathematical and Mechanical Growth Models - 1 page}
The contribution from the mechanics community to the understanding of growth of living biological materials dates back to a hundred years ago. Thompson \cite{Thompson} and Huxley \cite{Huxley} brought up the idea of organism-level morphological view where growth is characterized as a change in form. In the recent half century, a lot of breakthroughs have been made from the research on the mechanical impact on the biological materials. Early works include the growth velocity field representation (GVFR). The procedures include the use of path lines and streamlines \cite{Truesdell}. Applications of the GVFR have been reported in \cite{Erickson, Erickson2, Richards}. Despite of the simplicity, GVFR is is criticized of lacking most of essential ingredients that constitute a theory of material behavior \cite{Cowin}.

Another approach is the multi-constituent open systems based on the theory of mixtures \cite{Bowen, Rajagopal}. This approach models solids and fluids as combinations of diverse constituents. It is useful for modeling evolving contributions of the different solid constituents that comprise a tissue or cell and influence its growth. The resulting equations are usually complicated. For instance Humphrey and Rajagopal \cite{Humphrey} suggested that full mixture equations for mass balance (including reaction-diffusion effects of vasoactive, mitogenic, synthetic, proteolytic and inflammatory molecules) be used in combination with classical equations for linear momentum balance written in terms of rule-of-mixture relations for the stress response.

Different from the mathematical biologists cited above, modern continuum mechanical community treats the growth as a change in mass without ignoring the related change in form \cite{Ambrosi}. The key problem is how to translate the change of mass into non-uniform changes in form. Progress was made in hard tissues first because they are easier than soft tissues for three reasons, as put by Menzel and Kuhl in \cite{Menzel}: First, since they do not undergo large deformations in vivo, a linear kinematic characterization is sufficiently accurate.  Second, although they are multiphase materials consisting of solid and fluid constituents, for most technical applications, a single phase characterization in terms of the solid constituent alone represents their behavior sufficiently well. Third, since they are relatively easy to preserve, and their ex vivo response closely matches their in vivo behavior, they are relatively easy to test. For instance, Cowin's work in \cite{Cowin} shows that in case of small deformation and rotation, the problem that the mapping between the reference configuration and the spatial configuration is not ``onto'' can be avoided because the two configurations can be considered as identical. A lot of insights are gained from the studies on the growth of hard tissues including the mechanism of how astronauts lose bones in space \cite{Kuhl2}, the reason why hip replacement and repair induce a local loss of bone density \cite{Ambrosi}, the relationship between osteoporosis and the local bone loss \cite{Pang} and so on.

The growth of soft tissues is much more difficult according to \cite{Menzel} for three reasons: First, soft tissues typically undergo large deformations and their accurate characterization requires a finite kinematic approach \cite{Rodriguez}. Second, their multi-phase character often plays a critical role, and their reduction to single-phase materials could sometimes be overly simplified \cite{Mow}. Third, they are difficult to preserve, and their ex vivo response might vary significantly from their in vivo behavior \cite{Krishnamurthy}. Despite of the difficulties, researchers have developed a successful strategy to characterize soft biological tissues. The key ideas of this strategy is the multiplicative decomposition of the deformation gradient. By decomposing the deformation gradient into an elastic part and a growth part, this approach identifies an intermediate configuration which does not have to be compatible \cite{Goriely, Menzel2, Rodriguez}. This kinematic approach resembles the finite strain plasticity, which decomposes the deformation into an elastic part and a plastic part. The change of form is governed by the coupled solution of balances of mass and linear momentum. Many successful applications are reported soon after the forerunners in \cite{Hsu, Skalak}. This method has been utilized to model isotropic volumetric growth of the arterial wall in response to balloon angioplasty and restenosis \cite{Himpel, Kuhl3}, to characterize growth of tendon \cite{Garikipati2}, tumors \cite{Ambrosi2}, vascular tissue \cite{Humphrey2, Humphrey3, Taber}, and cardiac tissue \cite{Kroon, Goktepe}. The achievements in cardiovascular system is extraordinary considering the extra challenge of anisotropy. The anisotropic growth under myocardial infarction in vivo is characterized in \cite{Tsamis}, the reorientation of the microstructural directions is studied in \cite{Kuhl4, Kuhl5}. It is even discovered that cardiac muscle grows eccentrically in response to volume overload \cite{Goktepe2}, but concentrically in case of pressure overload \cite{Rausch}.

\clearpage
\subsection{Motivation of the Proposed Research - 0.5 page}
Although the continuum mechanical approach has been more or less established, there are unsolved problems:
the first problem is the lack of realistic boundary conditions on the solid tissues, which often lead to the unrealistic deformations. For instance, the study in \cite{Kuhl3} on patient-specific human aorta yields growth towards the outside rather than the inside after the stent operation because of the non-physiological boundary conditions. Since soft tissues grow in the fluid environment, the fluid-solid interaction (FSI) could act as a natural bound on the tissues. Unfortunately few of the published studies is devoted to account for it due to the complexity of coupling fluid and solid.

Another open issue is the mass flux. In the modeling of the growth, the increased mass either comes from the mass source inside the tissues or the mass flux coming from the outside. In the latter case, mass flux is considered as an additional boundary condition. In reality, the mass flux comes from the fluid environment. With the modeling of FSI, we could furthermore consider mass flux in fluid equations in order to obtain a more realistic system.

The other problem that has not received enough attention is the time scale. Biological growth happens at the time scale of weeks, months or even years. To simulate this process, the time step used is usually very large. On the other hand, to simulate the effect of the blood flow on the tissue in the vascular environment, the time scale of interest is usually a cardiac cycle, which is at the scale of seconds. Therefore it is important to find a way to coordinate two different time scales so that both processes can be reasonably simulated. Towards this end, we adopt a time-lapse approach based on the theory of ``small on large''.

\section{OBJECTIVE AND EXPECTED SIGNIFICANCE - 1.5 page}

\subsection{Objectives }
\begin{itemize}
\item{Couple the solid growth with the fluid medium}
\item{Model the mass transfer from the fluid to the solid}
\item{Implement the time-lapse strategy}
\end{itemize}

\subsection{Expected Significance}

\section{RESULTS FROM PRIOR RELATED WORK - 2.5 pages}

\subsection{Framework for Fluid-Structure Interactions}

\subsection{Growth Models}

\section{PROPOSED WORK - 6 pages}

\subsection{Task 1: Growth Models - isotropic and anisotropic?}
\subsection{Task 2: Numerical Framework of Growth Models in Fluid Environment - mass transfer from fluid to solid?}
\subsection{Task 3: Mutli-time Scale in Growth Models}


\section{BROADER IMPACT - 3 pages}

\subsection{Technical Impact}

\subsection{Outreach Activities}


\section{QUALIFICATIONS OF PARTICIPANTS - 0.5 page}

\section{PROJECT MANAGEMENT \& TIMELINE - 1 page}
\section{RESULTS FROM PRIOR NSF SUPPORTS - 0.5 page}
%
%Dr. Zhang, as a co-PI, recently completed a NSF grant, ACI-1126125, entitled ``MRI: Acquisition of a Balanced Environment for Simulation'', 9/1/2011-8/31/2015 (PI: Christopher Carothers), with a total funded amount of \$2,657,633. 
%The goal for this MRI (major research instrument) award was the construction of the compute core of the cyberinstrument Blue Gene/Q. 
%%
%\textbf{Intellectual Merit}: This project supported building a powerful MRI cyberinstrument combining IBM's new Blue Gene/Q system with integrated large-scale high-performance storage and visualization components 
%that provided 
%a balanced computation and data storage cyberinstrument for massively parallel simulations and data analytics. 
%%
%\textbf{Broader Impacts}: This cyberinstrument provided a new major research capability that is now accessible to a diverse set of disciplinary and inter-disciplinary researchers, students and industry collaborators. The instrument's combination of scale, balance, and Internet accessibility made it particularly valuable for under-served researchers, such as those from states and minority-serving institutions who would not otherwise have access to such a powerful resource.
%%
%Fourteen articles were published from Dr. Zhang using the equipment built with the grant \cite{li-particletransport-2015,cwang2013,wang-zhang-mifem,zhang2013,zhang2014,chuwang2015,wang2012semi,wang2013modified,zhang2013advancements,yong-2013-jcp,yong2013,yong-slip2013,chuwang2014,zhang2014modeling}, while 2 are under review \cite{yang2016VFdraft,yang2016PMLdraft}. Research product from this grant also includes established open-source software tools to study biomedical applications involving fluid-structure interactions, currently published and maintained on Github. 
%
\clearpage
\pagenumbering{arabic}
%\bibliography{cancer-refs,zhang2013-2016,refs-lucy,fem,chuthesis,Cell_NP,porous}
\bibliographystyle{unsrt}
\bibliography{references}
\end{document}
