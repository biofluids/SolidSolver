\section{Introduction}
Hyperelastic models have been successfully applied to various types of biomaterials scaling from onion epidermis \cite{Qian} to breast tissues \cite{OHagen}, from carotid vessels \cite{Zidi, Zidi2, Bols} to breast tumors \cite{Oberai}, and a number of animal organs including brain \cite{Karimi, Samani, Gilchrist}, lung  \cite{Wall, Wall2}, liver and kidney \cite{Fu, Untaroiu, Willinger}. A detailed review of the applications of isotropic hyperelastic models on biological tissues is provided in \cite{Kupriyanova}.
According to \cite{Steinmann}, hyperelastic models can be classified as phenomenological or micro-mechanical. The latter are derived from statistical mechanics arguments on networks of idealized chain molecules while the former utilize more or less complex, frequently polynomial formulations in terms of strain invariants or principle stretches. Because of the popularity of hyperelastic models, many commercial softwares have hyperelastic models as a built-in material selection. For instance, Abaqus FEA and COMSOL offer Mooney-Rivlin model, Yeoh model, Neo-Hookean model, etc. They also allow users to provide subroutines to define new models.

Although hyperelastic models have been frequently used with finite element method to model nonlinear deformation behaviors, it is not straightforward to implement them because of the complicated procedure invlved in deriving stress and elasticity tensors. The stress tensor is used to form the equilibrium equation, and the elasticity tensor is the keystone to form the tangent stiffness matrix that is used to solve the equilibrium equation. There are few existing literature or references providing a systematic approach to evaluate stress and elasticity tensors. An indicial method to derive the stress and elasticity tensors for Mooney-Rivlin model is presented in \cite{Bower}, which in the author's opinion, is not easy to be transplanted to another model. In \cite{Belytschko}, a general method to find the tangent stiffness matrix for any hyperelastic model in index notation is introduced in an abstract form with no illustrated examples. A step-by-step method written in tensor notation is derived in \cite{Holzapfel}, which we refer to frequently in this paper. But again limited examples for different models are shown and there is no computational results to validate the finite element implementation. Some derivations for specific models can be found in some early publications \cite{Weiss, Nicholson}. A more recent paper \cite{Suchocki} presents the implementation of Knowles model within Abaqus, using tensor notation described in \cite{Holzapfel}. But instead of using the spatial tensor of elasticity, the author used Jaumann objective rate as it is used in Abaqus \cite{Abaqus}. In this paper, we will present a systematic approach to derive the stress and elasticity tensors for any given hyperelastic model.

This paper is motivated to address this obstacle and clarify some of the key concepts presented in the existing literature in this field, as explained earlier. We present a framework for the derivation and walk the readers through the entire process with detailed examples. With the detailed derivation and the neat workflow, readers should be able to quickly implement any hyperelastic model. This is not only useful to the researchers who develop their own finite element code but also to those who work with commercial or open-source software with user-defined functions to include new constitutive models, especially for the ever-increasing newly defined models to describe biomaterials.

The rest of this paper is organized as follows: in Section \ref{general}, we introduce the basics of the hyperelastic model followed by the general procedures to derive the stress and elasticity tensors. To demonstrate how they can be applied for specific models, we present three different examples: Mooney-Rivlin model, Yeoh model, and Holzapfel-Gasser-Ogden (HGO) model. Mooney-Rivlin model is simple but very effective in modeling large strain nonlinear behavior of incompressible materials such as rubber and biomaterial. Yeoh model contains only one invariant but with a second order term. It is not complicated but can correctly predict the behavior of elastomer material in the range of a greater extent of deformation than the Mooney-Rivlin model \cite{Gajewski}, and is able to characterize the stiffening phenomenon of vulcanized rubber. The HGO model is designed to model collagen fiber-reinforced biological materials. This anisotropic model is so widely used that many commercial and open-source finite element packages have included it as a standard or user-defined model,  such as in Abaqus \cite{Abaqus}, COMSOL \cite{COMSOL} and FEBio \cite{FEBio}. In Section \ref{experiments}, two sets of numerical experiments are presented: biaxial tension and $2$D vessel expansion. The experiments are performed using the three hyperelastic models, where the results are validated with analytical solutions and existing computational results whenever possible, and isotropic and anisotropic behaviors are observed and compared. Finally, the conclusions are drawn in Section \ref{conclusions}.




 



