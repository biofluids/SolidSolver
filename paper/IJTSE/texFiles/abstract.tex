\begin{abstract}

Hyperelastic models are of particular interest in modeling biomaterials. In order to implement them, one must derive the stress and elasticity tensors from the given potential energy function explicitly. However, it is often cumbersome to do so because researchers in biomechanics may not be well-exposed to systematic approaches to derive the stress and elasticity tensors as it is vaguely addressed in literature. To resolve this, we present a framework of the general approach to derive the stress and elasticity tensors for hyperelastic models. During the derivation we carefully elaborated the differences between formulas used in the displacement-based formulation and the displacement/pressure mixed formulation. Three hyperelastic models, Mooney-Rivlin, Yeoh and Holzapfel-Gasser-Ogden models that span from first-order to higher order and from isotropic to anisotropic materials, are served as examples. These detailed derivations are validated with numerical experiments that demonstrate excellent agreements with analytical and other computational solutions. Following this framework, one could implement with ease any hyperelastic model as user-defined functions in software packages or source code from scratch.

\end{abstract}