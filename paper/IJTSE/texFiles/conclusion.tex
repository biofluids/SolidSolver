\section{Concluding Remarks}
\label{conclusions}
In order to implement any model with finite element method efficiently, one has to find a general way to derive the stress and elasticity tensors. In this paper, we presented a systematic way to do that by decomposing the strain-energy function, stress tensor, and the elasticity tensor into volumetric and isochoric parts. We derived the stress and elasticity tensors for the Mooney-Rivlin, Yeoh, and HGO model as examples. These models are popular in biomechanics and cover both isotropic and anisotropic models. Incompressibility in biomaterials was also discussed, where the tangent stiffness matrix of the standard displacement-based formulation is often ill-conditioned. Towards this end, the volumetric parts of the derived formulas are rewritten as functions of the pressure so that mixed formulation can be implemented. Two numerical expereiments were examined that include biaxial tension, and vessel expansion. These experiments involved Neo-Hookean, Mooney-Rivlin, Yeoh and HGO models. Their comparisons to analytical solutions and existing numerical solutions yielded excellent agreement. In summary, this paper addressed the challenges in the derivation of the stress and elasticity tensors for hyperelastic models using the existing literature by providing a framework with detailed procedure. It will be useful to the researchers in computational biomechanics. Following the systematic approach, readers should be able to derive stress and elasticity tensors for any hyperelastic model as part of their own code or as user-define functions in finite element packages.
