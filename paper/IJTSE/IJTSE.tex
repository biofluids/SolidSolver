%
%\documentclass[Proceedings]{ascelike}
\documentclass{article}[12pt]
%
% Feb. 14, 2013
%
% Some useful packages...
%
\usepackage{graphicx}
%\usepackage{subfigure}
\usepackage{amsmath}
\usepackage{amsfonts}
\usepackage{amssymb}
\usepackage{amsbsy}
\usepackage{setspace}
\usepackage{fullpage}
\usepackage{cite}
\usepackage{float}
\usepackage{subcaption}
\usepackage{appendix}
\usepackage{bbm}
\usepackage{booktabs}
\usepackage{multirow}

\usepackage[T1]{fontenc}
\usepackage[utf8]{inputenc}
\usepackage{authblk}

%\usepackage{lineno}
\doublespacing
%\linenumbers
%\usepackage{times}^\mathrm{s}
%
%
% Place hyperlinks within the pdf file (works only with pdflatex, not latex)
% \usepackage[colorlinks=true,citecolor=red,linkcolor=black]{hyperref}
%
%
% NOTE: Don't include the \NameTag{<your name>} if you have selected 
%       the NoPageNumbers option: this leads to an inconsistency and
%       a warning, and the NameTag is ignored.
%\NameTag{Zhang, Nov. 7, 2013}
%
%
\input texFiles/defs
\begin{document}

% You will need to make the title all-caps
\title{A General Approach to Derive Stress and Elasticity Tensors for Hyperelastic Isotropic and Anisotropic Biomaterials}
\author[1]{Jie Cheng}
\author[1]{Lucy T. Zhang\thanks{zhanglucy@rpi.edu}}
\affil[1]{Department of Mechanical Aerospace and Nuclear Engineering, Rensselaer Polytechnic Institute, Troy, NY 12180, USA}


\maketitle
\input texFiles/abstract
\textbf{Keywords: \textit{hyperelastic models; anisotropy; Mooney-Rivlin; Yeoh; Holzapfel-Gasser-Ogden; mixed formulation}}
\input texFiles/introduction
\input texFiles/formulations
\input texFiles/experiments
\input texFiles/conclusion
\input texFiles/appendix

\bibliography{references}
\bibliographystyle{unsrt}
%\bibliographystyle{apalike}
\clearpage
\input tables/tables
\end{document}



