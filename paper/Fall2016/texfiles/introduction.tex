\section{Introduction}
The contribution from the mechanics community to the understanding of growth of living biological materials dates back to a hundred years ago. Thompson \cite{Thompson} and Huxley \cite{Huxley} brought up the idea of organism-level morphological view where growth is characterized as a change in form. The interaction between mechanics and biology is mutual: on one hand biology changes mechanical properties significantly. For instance stiffness changes with biological microstructure. It is found in \cite{Carter} that bone stiffness not only increases linearly with bone density, but exponentially with the number and thickness of the individual trabaeculae. On the other hand biological microstructure can be changed by the mechanical loading: the living materials can adapt to environmental changes. A well-known example is that physical training changes the muscle and bone microstructure.

In the recent half century, a lot of breakthroughs have been made from the research on the mechanical impact on the biological materials. Early works include the grwoth velocity field representation (GVFR). The procedures include the use of path lines and streamlines \cite{Truesdell}. Applications of the GVFR have been reported in \cite{Erickson, Erickson2, Richards} D espite of the simplicity, GVFR is is criticized of lacking most of essential ingredients that constitute a theory of material behavior, and definition of the boundaries of the object considered as well as any boundary or initial condition \cite{Cowin}.

Another approach is the multi-constituent open systems based on the theory of mixtures \cite{Bowen, Rajagopal}. It also has numerous adherents, especially the complexity of notations required to account for multi-constituent and their various reference configurations \cite{Cowin}. Recent paper by Humphrey and Rajagopal \cite{Humphrey} develop this approach further by applying it to flow-induced arterial adaptions. Ateshian \cite{Ateshian} extends it to the concepts of reactive mixtures and singular surface.

Different from the mathematical biologists cited above, modern continuum mechanical community treats the growth as a change in mass without ignoring the related change in form \cite{Ambrosi}. The key problem is how to translate the change of mass into non-uniform changes in form. Progress was made in hard tissues first because they are easier than soft tissues for three reasons, as put by Menzel and Kuhl in \cite{Menzel}: First, since they do not undergo large deformations in vivo, a linear kinematic characterization is sufficiently accurate.  Second, although they are multiphase materials consisting of solid and fluid constituents, for most technical applications, a single phase characterization in terms of the solid constituent alone represents their behavior sufficiently well. Third, since they are relatively easy to preserve, and their ex vivo response closely matches their in vivo behavior, they are relatively easy to test. For instance, Cowin's work in \cite{Cowin} shows that in case of small deformation and rotation, the problem that the mapping between the reference configuration and the spatial configuration is not "onto" can be avoided because the two configurations can be considered as identical. \cite{Kuhl} introduces the kinematics, time and spatial discretization, and the flowchart of the final implementation of the growth model of hard tissues. For a more detailed review on the bone generation, please refer to \cite{Isaksson}. A lot of insights are gained from these studies including the mechanism of how astronauts lose bones in space \cite{Kuhl2}, the reason why hip replacement and repair induce a local loss of bone density \cite{Ambrosi}, the relationship between osteoporosis and the local bone loss \cite{Pang} and so on.

The growth of soft tissues is much more difficult according to \cite{Menzel} for three reasons: First, soft tissues typically undergo large deformations and their accurate characterization requires a finite kinematic approach \cite{Rodriguez}. Second, their multi-phase character often plays a critical role, and their reduction to single-phase materials could sometimes be overly simplified \cite{Mow}. Third, they are difficult to preserve, and their ex vivo response might vary significantly from their in vivo behavior \cite{Krishnamurthy}. Despite of the difficulties, researchers have developed a successful strategy to characterize soft biological tissues. The key ideas of this strategy is the multiplicative decomposition of the deformation gradient. By decomposing the deformation gradient into an elastic part and a growth part, this approach identifies an intermediate configuration which does not have to be incompatible \cite{Goriely, Menzel2, Rodriguez}. This kinematic approach resembles the finite strain plasticity, which decomposes the deformation into an elastic part and a plastic part. The change of form is governed by the coupled solution of balances of mass and linear momentum. Many successful applications are reported soon after the forerunners in \cite{Hsu, Skalak}. The growth of various soft biological tissues are modeled in \cite{Garikipati, Lubarda}. This concept is first utilized to model isotropic volumetric growth of the arterial wall in response to balloon angioplasty and restenosis \cite{Himpel, Kuhl3}, to characterize growth of tendon \cite{Garikipati2}, tumors \cite{Ambrosi2}, vascular tissue \cite{Humphrey2, Humphrey3, Taber}, and cardiac tissue \cite{Kroon, Goktepe}. Many applications are health-related thus attract more attention. A typical example is concerning growing tumors. Computational simulation of transversely isotropic area growth successfully predicts area growth in response to controlled mechanical overstretch during tissue expansion \cite{Zollner}. In contrst to tumor growth, growth of skin is initiated on purpose to create extra tissue for defect correction in plastic and reconstructive surgery \cite{Zollner2}. The achievements in cardiovascular system is extraordinary considering the extra challenge of anisotropy. The anisotropic growth under myocardial infarction in vivo is characterized in \cite{Tsamis}, the reorientation of the microstructural directions is studied in \cite{Kuhl4, Kuhl5}. It is even discovered that tissue may grow anisotropically in response to different loading scenarios. For example, cardiac muscle grows isotropically \cite{Kroon}, eccentrically in response to volume overload \cite{Goktepe2}, or concentrically in case of pressure overload \cite{Rausch}.

Another term that is often used together with growth is "remodeling". Remodeling refers to the changing in properties such as the anisotropy, stiffness and strength, that are resulted from the changes in microstructures. One example that demonstrates the interaction between remodeling and growth is the growth of long bones. The cartilaginous growth plate near the ends of a long bone provides an increment of length to the bone before the growth plate closes (endochondral ossification). This additional length distorts the overall shape of the bone initiating a remodeling process that reshapes the bone. Growth and remodeling are separate processes in this case, but the former initiates the latter \cite{Ambrosi}. To distinguish remodeling from growth, we follow the convention of restricting the term remodeling to signify a change in the underlying microstructure while mass is held constant \cite{Kuhl4, Garikipati3}.

Other than growth and remodeling, morphogenesis is  also a developmental process of importance in biology. In fact it is the most dramatic change in these three developments. Problems include the development of the heart, blood vessels, brain, lungs gut, eye and musculoskeletal system \cite{Murray}. in form which happens in the embryo development. In the embryo, cells move either individually (mesenchyme) or in sheets (epithelia). The heart, brain, and gut begin as epithelia. Muscles and bone arise from clusters of mesenchymal cells \cite{Ambrosi}. A major model for mesenchymal morphogenesis is the Murray-Oster theory, which is based on continuum mechanics for a mixture of cells and matrix. The effects of cell traction, mitosis, matrix secretion, cell migration and differential adhesion are considered \cite{Murray2, Oster}.

This research will be focused on growth of soft tissues introduced above. We will adopt the idea of introducing an incompatible growth configuration and decomposing deformation gradient into an elastic and growth part, which has been well-accepted. The work is based on \cite{Himpel, Kuhl3, Goktepe2} where the growth is regulated by stress and the growth tensor is implemented as an internal variable within a finite element framework. This approach is easy to cooperate with any existing standard finite element solver and the extension from isotropic models to anisotropic models is extremely straightforward.