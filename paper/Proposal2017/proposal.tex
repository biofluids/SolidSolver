\documentclass[12pt]{article}
\usepackage{fullpage}
\usepackage{graphicx}
\usepackage{subfigure}
\usepackage{amsmath}
\usepackage{amsfonts}
\usepackage{amssymb}
\usepackage{amsbsy}
\usepackage{setspace}
\usepackage{fullpage}
\usepackage{wrapfig}
\usepackage{cite}
\usepackage{float}
\usepackage{enumitem} 
\usepackage{algorithm, algpseudocode}
%\usepackage[skip=2pt,font=footnotesize]{caption}

\begin{document}
\begin{center}
 \LARGE\textbf{
Building An Efficient Toolkit for Multiphysics and Multiscale Coupling}
\end{center}

\tableofcontents
\newpage
\clearpage
%\newcommand{\bb}[1]{\mathbb{#1}}
\newcommand{\bF}{\bold{F}}
\newcommand{\bC}{\bold{C}}
\newcommand{\bX}{\bold{X}}
\newcommand{\bx}{\bold{x}}
\newcommand{\bg}{\bold{g}}
\newcommand{\bl}{\bold{l}}
\newcommand{\bL}{\bold{L}}
\newcommand{\bv}{\bold{v}}
\newcommand{\bS}{\bold{S}}
\newcommand{\bM}{\bold{M}}
\newcommand{\bI}{\bold{I}}
\newcommand{\bA}{\bold{A}}
\newcommand{\bB}{\bold{B}}

\newcommand{\bFe}{\bold{F}_\mathrm{e}}
\newcommand{\bFg}{\bold{F}_\mathrm{g}}
\newcommand{\bCe}{\bold{C}_\mathrm{e}}
\newcommand{\bCg}{\bold{C}_\mathrm{g}}
\newcommand{\bLe}{\bold{L}_\mathrm{e}}
\newcommand{\bLg}{\bold{L}_\mathrm{g}}

\newcommand{\rme}{\mathrm{e}}
\newcommand{\rmg}{\mathrm{g}}
\newcommand{\rmt}{\mathrm{t}}

\newcommand{\bbC}{\mathbb{C}}
\newcommand{\bbc}{\mathbb{c}}
\newcommand{\grho}{\bar{\rho}_0}
\section{BACKGROUND AND MOTIVATION: 2 pages-Li }

\subsection{Background: }

\subsection{The State-of-the-Art: }


\subsection{Motivation of the Proposed Research}
Defining the bottleneck


\section{OBJECTIVE AND EXPECTED SIGNIFICANCE: 1 page-Zhang}

\subsection{Objectives}
The objective of this project is to \textbf{design and implement a toolkit that can efficiently coupling existing solvers for multiphysics and multiscale analysis}. This proposed work is accomplished by completing the following tasks:

\textbf{Task 1: Designing Coupling Algorithm} 

\textbf{Task 2: CPU \& GPU} 

\textbf{Task 3: Building Front and Back Ends with Existing Open Source Tools - w/ Kitware} 

\textbf{Task 4: Open Source Project: Expand and Build Sustainable User Base}

\textbf{Task 5: Outreach} 

\subsection{Expected Significance}

\section{RESULTS FROM PRIOR RELATED WORK: 2 pages - Li}

\subsection{Multiphysics Solvers}
\subsection{Multiscale Solvers}

%
\subsection{Parallelization}

\subsection{Building Open Source Community - Existing Users: 0.5 page - Zhang}

\section{PROPOSED WORK}

\subsection{Task 1: Coupling Algorithm: 3-pages - Zhang, Li }

\subsubsection{1.1 Design of the toolkit}
\subsubsection{1.2 Validation and test suites}
\subsubsection{1.3 Convergence studies}

\subsection{Task 2: Utilization of CPU \& GPU: 2-pages - Li, Carothers}
\subsubsection{2.1 Computer architecture}
\subsubsection{2.2 Scability studies}

\subsection{Task 3: Building Front \& Back Ends: 2-pages - Zhang, Kitware}
\subsubsection{3.1 Front end: incorporate cMake for commonly-used compilers and operating systems}
\subsubsection{3.2 Back end: output to Paraview formats}



\section{BROADER IMPACT}

\subsection{Task 4: Open Source Project: 1.5 page}
\subsubsection{4.1 Multiphysics and Multiscale Applications:  Li}
\subsubsection{4.2 Code Maintenance and Documentation:  Zhang}
\subsubsection{4.3 Building Community in Computer Science and Engineering: Li, Carothers}

\subsection{Task 5: Outreach Activities - Li, Zhang}


\vskip0.05in



\section{PROJECT MANAGEMENT \& TIMELINE - 1 pages - Zhang, Li}


\section{RESULTS FROM PRIOR NSF SUPPORTS- 0.5 pages - Zhang, Li, Carothers}

Dr. Zhang, as a co-PI, recently completed a NSF grant, ACI-1126125, entitled ``MRI: Acquisition of a Balanced Environment for Simulation'', 9/1/2011-8/31/2015 (PI: Christopher Carothers), with a total funded amount of \$2,657,633. 
The goal for this MRI (major research instrument) award was the construction of the compute core of the cyberinstrument Blue Gene/Q. 
%
\textbf{Intellectual Merit}: This project supported building a powerful MRI cyberinstrument combining IBM's new Blue Gene/Q system with integrated large-scale high-performance storage and visualization components 
that provided 
a balanced computation and data storage cyberinstrument for massively parallel simulations and data analytics. 
%
%a well-balanced massively parallel computing platform that is uniquely capable of supporting cutting edge research across a wide range of disciplines from semantic integration of the abundance of heterogeneous, multimodal, and multiscale data to improve personal health to modeling flow and disease in the arterial system. 
%
\textbf{Broader Impacts}: This cyberinstrument provided a new major research capability that is now accessible to a diverse set of disciplinary and inter-disciplinary researchers, students and industry collaborators. The instrument's combination of scale, balance, and Internet accessibility made it particularly valuable for under-served researchers, such as those from states and minority-serving institutions who would not otherwise have access to such a powerful resource.
%, nor the ability to pre- and post-process the massive data sets involved.
%
%Since this is a MRI grant, no salary or student support was requested. 
Fifteen (15)  articles were published from Dr. Zhang using the equipment built with the grant \cite{li-particletransport-2015,cwang2013,wang-zhang-mifem,zhang2013,zhang2014,chuwang2015,wang2012semi,wang2013modified,zhang2013advancements,yong-2013-jcp,yong2013,yong-slip2013,chuwang2014,zhang2014modeling,yang2016VFdraft}, while 1 is under review \cite{yang2016PMLdraft}. Research product from this grant also includes established open-source software tools to study biomedical applications involving fluid-structure interactions, currently published and maintained on GitHub. 

\clearpage

%
\clearpage
\pagenumbering{arabic}
\bibliographystyle{unsrt}
\bibliography{references,zhang2013-2016}
\end{document}
