\section{Examples}
Mooney-Rivlin is one of the most frequently used model in biomechanics. This model predicts the stress in rubber-like materials very well when the deformation is not too large. In this section, we will use Mooney-Rivlin model to demonstrate the procedures described in Section \ref{general}. 

%
\subsection{Mooney-Rivlin Model}
\subsubsection{The Decoupled Form of Mooney-Rivlin Model}
The decoupled form of compressible Mooney-Rivlin model is written as
\begin{equation}
\Psi = \Psi_{iso} + \Psi_{vol}
\end{equation}
\begin{equation} \label{vol}
\Psi_{vol} = \frac{\kappa}{2}(J - 1)^2
\end{equation}
\begin{equation} \label{iso}
\Psi_{iso} = \frac{\mu_1}{2}(\bar{I_1} - 3) + \frac{\mu_2}{2}(\bar{I_2} - 3)
\end{equation}
where $\mu_1$, $\mu_2$ and $\kappa$ are material constants. For small deformations, the shear modulus and bulk modulus can be approximated by $\mu_1+\mu_2$ and $\kappa$.

For incompressible Mooney-Rivlin model, we replace Equation \ref{vol} with
\begin{equation}
\Psi_{vol} = p(J - 1)
\end{equation}
where $p$ is a Lagrangian multiplier to enforce the incompressibility condition and can be interpreted as the hydrostatic pressure.  

%
\subsubsection{Stress}
Substitute Equation \ref{vol} into Equation \ref{Svol}, and recall the definition in Equation \ref{pressure}, we can derive the expression for the second Piola-Kirchhoff stress.
\begin{equation} \label{mSvol}
\bold{S}_{vol} = Jp\bold{C}^{-1} = \kappa{J(J-1)}\bold{C}^{-1} = \kappa{J^{1/3}(J-1)}\bold{\overline{C}}^{-1}
\end{equation}
In mixed formulation, $p$ is independent from $\bold{u}$, the volumetric part is simply
\begin{equation} \label{mSvol1}
\bold{S}_{vol} = Jp\bold{C}^{-1}
\end{equation}
To obtain the isochoric stress, first derive the fictitious stress in Equation \ref{fictitious}
\begin{equation} \label{mfictitious}
\begin{split}
\overline{\bold{S}} &= 2\frac{\partial\Psi_{iso}({\overline{\bold{C}})}}{\partial\overline{\bold{C}}} \\
&= 2\frac{\partial\Psi_{iso}}{\partial\bar{I_1}} \frac{\partial\bar{I_1}}{\partial\overline{\bold{C}}}  + 2\frac{\partial\Psi_{iso}}{\partial\bar{I_2}} \frac{\partial\bar{I_2}}{\partial\overline{\bold{C}}} \\
&= \mu_1{\bold{I}} + \mu_2({\bar{I_1}\bold{I} - \overline{\bold{C}}}) \\
&= (\mu_1+\mu_2\bar{I_1})\bold{I} - \mu_2{\overline{\bold{C}}}
\end{split}
\end{equation}
Multiply with the 4th order identity tensor
\begin{equation}
\begin{split}
\mathbb{P}:\overline{\bold{S}} &= ( \mathbb{I} - \frac{1}{3}\bold{C}^{-1} \otimes \bold{C} ) : \overline{\bold{S}} \\
&= \overline{\bold{S}} - \frac{1}{3}( \overline{\bold{C}}^{-1} \otimes \overline{\bold{C}} ) : \overline{\bold{S}} \\
&= \overline{\bold{S}} - \frac{1}{3} \overline{\bold{C}}^{-1} \otimes (\overline{\bold{C}} : \overline{\bold{S}}) \\
&= \overline{\bold{S}} - \frac{1}{3} \overline{\bold{C}}^{-1} \otimes \left[(\mu_1+\mu_2\bar{I_1})\bar{I_1} - \mu_2(\bar{I_1}^2-2\bar{I_2})\right]\\
&= \overline{\bold{S}} -  \frac{1}{3}(\mu_1\bar{I_1} + 2\mu_2\bar{I_2}) \overline{\bold{C}}^{-1} 
\end{split}
\end{equation}
Therefore the isochoric part of the second Piola-Kirchhoff stress is
\begin{equation} \label{mSiso}
\bold{S_{iso}} = J^{-2/3}\mathbb{P}:\overline{\bold{S}} = J^{-2/3}\left[    - \frac{1}{3}(\mu_1\bar{I_1} + 2\mu_2\bar{I_2}) \overline{\bold{C}}^{-1}  + (\mu_1 + \mu_2\bar{I_1})\bold{I} - \mu_2\overline{\bold{C}} \right]
\end{equation}
Equation \ref{mSvol} and \ref{mSiso} are the expression we need in the displacement-based formulation. In the mixed formulation, use \ref{mSvol1} instead of \ref{mSvol}.

%
\subsubsection{Elasticity Tensor}
Insert our model into the definition of $\tilde{p}$ in Equation \ref{tildep}
\begin{equation}
\tilde{p} = p + J\frac{dp}{dJ} + \kappa(J-1) + \kappa{J} = \kappa{(2J - 1)}
\end{equation}
Substituting $p$ and $\tilde{p}$ into Equation \ref{Cvol}, we have
\begin{equation}
\begin{split}
\mathbb{C}_{vol} &= J\kappa{(2J-1)} \bold{C}^{-1} \otimes \bold{C}^{-1} - 2J\kappa{(J-1)} \bold{C}^{-1} \odot \bold{C}^{-1}  \\
&= J^{-1/3}\kappa{(2J-1)} \overline{\bold{C}}^{-1} \otimes \overline{\bold{C}}^{-1} - 2J^{-1/3}\kappa{(J-1)} \overline{\bold{C}}^{-1} \odot \overline{\bold{C}}^{-1}
\end{split}
\end{equation}
In mixed formulation
\begin{equation} \label{mCvol}
\begin{split}
\mathbb{C}_{vol} &= Jp\bold{C}^{-1}\otimes\bold{C}^{-1} - 2Jp\bold{C}^{-1}\odot\bold{C}^{-1} \\
&= J^{-1/3}p\overline{\bold{C}}^{-1}\otimes\overline{\bold{C}}^{-1} - 2J^{-1/3}p\overline{\bold{C}}^{-1}\odot\overline{\bold{C}}^{-1}
\end{split}
\end{equation}
Notice $\tilde{p}$ is replaced with $p$ because ${dp}/{dJ}$ is $0$.

Next we follow Equation \ref{Ciso} and \ref{barcc} to \ref{tracec} to derive $\mathbb{C}_{iso}$. First, find the second order derivative of $\Psi_{iso}$ with respect to $\overline{\bold{C}}$.
\begin{equation}
\begin{split}
\frac{\partial{\Psi_{iso}}}{\partial{\overline{\bold{C}}}} &= \frac{\mu_1}{2}\frac{\partial{\bar{I_1}}}{\partial{\overline{\bold{C}}}} +  \frac{\mu_2}{2}\frac{\partial{\bar{I_2}}}{\partial{\overline{\bold{C}}}} \\
&= \frac{\mu_1}{2}\bold{I} + \frac{\mu_2}{2}(\bar{I_1}\bold{I} - \overline{\bold{C}})
\end{split}
\end{equation}
\begin{equation}
\frac{\partial^2\Psi_{iso}(\overline{\bold{C}})}{\partial\overline{\bold{C}}\partial\overline{\bold{C}}} = 
\frac{\mu_2}{2}(\bold{I} \otimes \frac{\partial{\bar{I_1}}}{\partial{\overline{\bold{C}}}} - \mathbb{I}) = \frac{\mu_2}{2}(\bold{I} \otimes \bold{I} - \mathbb{I})
\end{equation}
Substituting into Equation \ref{barcc}
\begin{equation}
\overline{\mathbb{C}} = 2J^{-4/3}\frac{\partial^2\Psi_{iso}(\overline{\bold{C}})}{\partial{\overline{\bold{C}}}{\partial{\overline{\bold{C}}}}} = 2\mu_2{J^{-4/3}}(\bold{I} \otimes \bold{I} - \mathbb{I})
\end{equation}
The first part of the right-hand-side Equation \ref{Ciso} is
\begin{equation} \label{part1}
\begin{split}
\mathbb{P} : {\bold{C}} : {\mathbb{P}}^T &= 2\mu_2{J^{-4/3}} (\mathbb{I} - \frac{1}{3}{\bold{C}}^{-1} \otimes {\bold{C}})
: (\bold{I} \otimes \bold{I} - \mathbb{I}) : (\mathbb{I} - \frac{1}{3}\overline{\bold{C}} \otimes {\bold{C}}^{-1}) \\
&= 2\mu_2{J^{-4/3}}[\mathbb{I} : (\bold{I} \otimes \bold{I}) : \mathbb{I} - \frac{1}{3}\mathbb{I} : (\bold{I} \otimes \bold{I}) : ({{\bold{C}}} \otimes {{\bold{C}}}^{-1}) - \mathbb{I} : \mathbb{I} : \mathbb{I} \\
&+ \mathbb{I} : \mathbb{I} : \frac{1}{3}({\bold{C}} \otimes {{\bold{C}}}^{-1}) - 
\frac{1}{3}({{\bold{C}}}^{-1} \otimes {\bold{C}} ) : (\bold{I} \otimes \bold{I}) : \mathbb{I}
- \frac{1}{3}({{\bold{C}}}^{-1} \otimes {\bold{C}} ) :  \mathbb{I} : \frac{1}{3}({{\bold{C}}} \otimes {{\bold{C}} }^{-1}) \\
&+ \frac{1}{3}({{\bold{C}}}^{-1} \otimes {\bold{C}} ) : (\bold{I} \otimes \bold{I}) : \frac{1}{3}({{\bold{C}}} \otimes {{\bold{C}}}^{-1} ) + \frac{1}{3}({{\bold{C}}} \otimes {{\bold{C}}}^{-1}) : \mathbb{I} : \mathbb{I}
] \\
&= 2\mu_2{J^{-4/3}}[\bold{I} \otimes \bold{I} - \frac{1}{3}{I_1}\bold{I} \otimes {{\bold{C}}}^{-1} - \mathbb{I} + \frac{1}{3}{\bold{C}} \otimes {{\bold{C}}}^{-1} -  \frac{1}{3}{I_1}{{\bold{C}}}^{-1} \otimes \bold{I} \\
&- \frac{1}{9}({\bold{C}} : {\bold{C}}){{\bold{C}}}^{-1} \otimes {{\bold{C}}} + \frac{1}{9}{{I_1}}^2{{\bold{C}}}^{-1} \otimes {{\bold{C}}}^{-1} +  \frac{1}{3}{{\bold{C}}}^{-1} \otimes {{\bold{C}}}] \\
&= 2\mu_2{J}^{-4/3}[\bold{I} \otimes \bold{I} - \mathbb{I} - \frac{1}{3}{I_1}({{\bold{C}}}^{-1} \otimes \bold{I} + \bold{I}\otimes{{\bold{C}}}^{-1} ) +
\frac{1}{3}({{\bold{C}}}^{-1} \otimes {\bold{C}} + {\bold{C}}\otimes{{\bold{C}}}^{-1} )  + \frac{2}{9}{I_2} {{\bold{C}}}^{-1} \otimes {{\bold{C}}}^{-1}] \\
&= 2\mu_2{J}^{-4/3}(\bold{I} \otimes \bold{I} - \mathbb{I}) - \frac{2}{3}\mu_2J^{-4/3}\bar{I_1}({\overline{\bold{C}}}^{-1} \otimes \bold{I} + \bold{I} \otimes {\overline{\bold{C}}}^{-1}) \\
&+
\frac{2}{3}\mu_2J^{-4/3}({\overline{\bold{C}}}^{-1} \otimes {\overline{\bold{C}}} + {\overline{\bold{C}}} \otimes {\overline{\bold{C}}}^{-1}) + \frac{4}{9}\mu_2\bar{I_2}J^{-4/3}({\overline{\bold{C}}}^{-1} \otimes {\overline{\bold{C}}}^{-1})
\end{split}
\end{equation}
Use Equation \ref{mfictitious} to derive the second part of the right-hand-side of Equation \ref{Ciso}
\begin{equation}
\begin{split}
Tr(J^{-2/3}\overline{\bold{S}}) &= J^{-2/3}{\overline{\bold{S}}} : {\overline{\bold{C}}} \\
&= [(\mu_1 + \mu_2{\bar{I_1}})\bold{I} - \mu_2{\overline{\bold{C}}}] : \overline{\bold{C}} \\
&= (\mu_1 + \mu_2\bar{I_1})\bar{I_1} - \mu_2{\overline{\bold{C}}} : {\overline{\bold{C}}} \\
&= (\mu_1 + \mu_2\bar{I_1})\bar{I_1} - \mu_2({\bar{I_1}}^2 - 2\bar{I_2}) \\
&= \mu_1\bar{I_1} + 2\mu_2\bar{I_2}
\end{split}
\end{equation}
Therefore the second part of the right-hand-side of Equation \ref{Ciso} is
\begin{equation} \label{part2}
\begin{split}
\frac{2}{3}Tr(J^{-2/3}\overline{\bold{S}})\tilde{\mathbb{P}} &= \frac{2}{3}(\mu_1\bar{I_1} + 2\mu_2\bar{I_2})({\bold{C}}^{-1} \odot {\bold{C}}^{-1} - \frac{1}{3}{\bold{C}}^{-1} \otimes {\bold{C}}^{-1}) \\
&= \frac{2}{3}J^{-4/3}(\mu_1\bar{I_1} + 2\mu_2\bar{I_2})({\overline{\bold{C}}}^{-1} \odot {{\overline{\bold{C}}}}^{-1} - \frac{1}{3}{{\overline{\bold{C}}}}^{-1} \otimes {\overline{{\bold{C}}}}^{-1})
\end{split}
\end{equation}
Plug in Equation \ref{mSiso} into Equation the third part of the right-hand-side of Equation \ref{Ciso}
\begin{equation} \label{part3}
\begin{split}
- \frac{2}{3}(\bold{C}^{-1}\otimes\bold{S}_{iso} + \bold{S}_{iso}\otimes \bold{C}^{-1})
&=
- \frac{2}{3}J^{-2/3} \{ {\bold{C}}^{-1} \otimes [-\frac{1}{3}(\mu_1\bar{I_1} + 2\mu_2\bar{I_2}){\overline{\bold{C}}}^{-1} + (\mu_1 + \mu_2\bar{I_1})\bold{I} - \mu_2{\overline{\bold{C}}}] \\
&+
[-\frac{1}{3}(\mu_1\bar{I_1} + 2\mu_2\bar{I_2}){\overline{\bold{C}}}^{-1} + (\mu_1 + \mu_2\bar{I_1})\bold{I} - \mu_2{\overline{\bold{C}}}] \otimes {\bold{C}}^{-1}\} \\
&=
- \frac{2}{3}J^{-4/3} [ -\frac{1}{3}(\mu_1\bar{I_1} + 2\mu_2\bar{I_2}) {\overline{\bold{C}}}^{-1} \otimes {\overline{\bold{C}}}^{-1} + (\mu_1 + \mu_2\bar{I_1}){\overline{\bold{C}}}^{-1}\otimes\bold{I} - \mu_2{\overline{\bold{C}}}^{-1} \otimes {\overline{\bold{C}}} \\
&-
\frac{1}{3}(\mu_1\bar{I_1} + 2\mu_2\bar{I_2}) {\overline{\bold{C}}}^{-1} \otimes {\overline{\bold{C}}}^{-1} + (\mu_1 + \mu_2\bar{I_1})\bold{I} \otimes {\overline{\bold{C}}}^{-1} - \mu_2{\overline{\bold{C}}} \otimes {\overline{\bold{C}}}^{-1}] \\
&=
 - \frac{2}{3}J^{-4/3} [ -\frac{2}{3}(\mu_1\bar{I_1} + 2\mu_2\bar{I_2})({\overline{\bold{C}}}^{-1} \otimes {\overline{\bold{C}}}^{-1}) + (\mu_1+\mu_2\bar{I_1})({\overline{\bold{C}}}^{-1} \otimes \bold{I} + \bold{I} \otimes {\overline{\bold{C}}}^{-1})\\
&- \mu_2({\overline{\bold{C}}}^{-1} \otimes {\overline{\bold{C}}}+{\overline{\bold{C}}} \otimes {\overline{\bold{C}}}^{-1}) ]
\end{split}
\end{equation}
Combining Equation \ref{part1}, \ref{part2} and \ref{part3}, we have
\begin{equation} \label{mCiso}
\begin{split}
\mathbb{C}_{iso} 
&= 
2\mu_2{J}^{-4/3}(\bold{I} \otimes \bold{I} - \mathbb{I}) - \frac{2}{3}\mu_2J^{-4/3}\bar{I_1}({\overline{\bold{C}}}^{-1} \otimes \bold{I} + \bold{I} \otimes {\overline{\bold{C}}}^{-1}) \\
&+
\frac{2}{3}\mu_2J^{-4/3}({\overline{\bold{C}}}^{-1} \otimes {\overline{\bold{C}}} + {\overline{\bold{C}}} \otimes {\overline{\bold{C}}}^{-1}) + \frac{4}{9}\mu_2\bar{I_2}J^{-4/3}({\overline{\bold{C}}}^{-1} \otimes {\overline{\bold{C}}}^{-1}) \\
&+
\frac{2}{3}J^{-4/3}(\mu_1\bar{I_1} + 2\mu_2\bar{I_2})({\overline{\bold{C}}}^{-1} \odot {{\overline{\bold{C}}}}^{-1} - \frac{1}{3}{{\overline{\bold{C}}}}^{-1} \otimes {\overline{{\bold{C}}}}^{-1}) \\
&+
\frac{4}{9}J^{-4/3} (\mu_1\bar{I_1} + 2\mu_2\bar{I_2})({\overline{\bold{C}}}^{-1} \otimes {\overline{\bold{C}}}^{-1}) - \frac{2}{3}J^{-4/3}(\mu_1+\mu_2\bar{I_1})({\overline{\bold{C}}}^{-1} \otimes \bold{I} + \bold{I} \otimes {\overline{\bold{C}}}^{-1})\\
&+ \frac{2}{3}J^{-4/3}\mu_2({\overline{\bold{C}}}^{-1} \otimes {\overline{\bold{C}}}+{\overline{\bold{C}}} \otimes {\overline{\bold{C}}}^{-1}) ] \\
&=
2\mu_2{J}^{-4/3}(\bold{I} \otimes \bold{I} - \mathbb{I}) + \frac{2}{3}(\mu_1\bar{I_1} + 2\mu_2\bar{I_2})J^{-4/3}({\overline{\bold{C}}}^{-1} \odot {\overline{\bold{C}}}^{-1} - \frac{1}{3}{\overline{\bold{C}}}^{-1} \otimes {\overline{\bold{C}}}^{-1}) \\
&-
\frac{2}{3}J^{-4/3}(\mu_1 + 2\mu_2\bar{I_1})({\overline{\bold{C}}}^{-1} \otimes \bold{I} + \bold{I} \otimes {\overline{\bold{C}}}^{-1}) + \frac{4}{3}\mu_2J^{-4/3}({\overline{\bold{C}}}^{-1} \otimes {\overline{\bold{C}}} + {\overline{\bold{C}}} \otimes {\overline{\bold{C}}}^{-1}) \\
&+
\frac{4}{9}J^{-4/3}(\mu_1\bar{I_1} + 3\mu_2\bar{I_2}){\overline{\bold{C}}}^{-1} \otimes {\overline{\bold{C}}}^{-1} \\
&=
2\mu_2{J}^{-4/3}(\bold{I} \otimes \bold{I} - \mathbb{I}) - \frac{2}{3}J^{-4/3}(\mu_1 + 2\mu_2\bar{I_1})({\overline{\bold{C}}}^{-1} \otimes \bold{I} + \bold{I} \otimes {\overline{\bold{C}}}^{-1}) + \frac{4}{3}\mu_2J^{-4/3}({\overline{\bold{C}}}^{-1} \otimes {\overline{\bold{C}}} + {\overline{\bold{C}}} \otimes {\overline{\bold{C}}}^{-1}) \\
&+
\frac{2}{9}J^{-4/3}(\mu_1\bar{I_1} + 4\mu_2\bar{I_2}) {\overline{\bold{C}}}^{-1} \otimes {\overline{\bold{C}}}^{-1}) + \frac{2}{3}J^{-4/3}(\mu_1\bar{I_1} + 2\mu_2\bar{I_2}){\overline{\bold{C}}}^{-1} \odot {\overline{\bold{C}}}^{-1} 
\end{split}
\end{equation} 

%
\subsubsection{Matrix Form}

%
\subsection{Another Model}







