\section{Examples}
In this section, we will derive the stress and elasticity tensor in detail for two models: Yeoh model and Mooney-Rivlin model, as demonstrations of the procedures described in Section \ref{general}. Both of them are designed to model incompressible rubber. In practice, a volumetric part is added to allow a small compressibility to make them more realistic.  Both models use invariants as variables and have simple forms. Mooney-Rivlin model is linear in both $\bar{I_1}$ and $\bar{I_2}$ while in Yeoh model $\bar{I_2}$ is neglected but higher order terms of $\bar{I_1}$ are added to capture the stiffening effect of rubber in the large strain domain.

%
\subsection{Mooney-Rivlin Model}
Mooney-Rivlin is one of the most popular models used in biomechanics, especially in case of large deformation. In case of small deformation, this behavior is close to linear material.
\subsubsection{The Decoupled Form of Mooney-Rivlin Model}
The decoupled form of compressible Mooney-Rivlin model is written as
\begin{equation}
\Psi = \Psi_{iso} + \Psi_{vol}
\end{equation}
\begin{equation} \label{vol1}
\Psi_{vol} = \frac{\kappa}{2}(J - 1)^2
\end{equation}
\begin{equation} \label{iso}
\Psi_{iso} = \frac{\mu_1}{2}(\bar{I_1} - 3) + \frac{\mu_2}{2}(\bar{I_2} - 3)
\end{equation}
where $\mu_1$, $\mu_2$ and $\kappa$ are material constants. For small deformations, the shear modulus and bulk modulus can be approximated by $\mu_1+\mu_2$ and $\kappa$.

In mixed formulation we replace Equation \ref{vol1} with
\begin{equation} \label{vol2}
\Psi_{vol} = p(J - 1)
\end{equation}
where $p$ is a Lagrangian multiplier to enforce the incompressibility condition and can be interpreted as the hydrostatic pressure.  

%
\subsubsection{Stress}
Substitute Equation \ref{vol} into Equation \ref{Svol}, and recall the definition in Equation \ref{pressure}, we can derive the expression for the PK2 stress.
\begin{equation} \label{mSvol}
\bold{S}_{vol} = Jp\bold{C}^{-1} = \kappa{J(J-1)}\bold{C}^{-1} = \kappa{J^{1/3}(J-1)}\bold{\overline{C}}^{-1}
\end{equation}
In mixed formulation, $p$ is independent from $\bold{u}$, the volumetric part is simply
\begin{equation} \label{mSvol1}
\bold{S}_{vol} = Jp\bold{C}^{-1}
\end{equation}
To obtain the isochoric stress, first derive the fictitious stress in Equation \ref{fictitious}
\begin{equation} \label{mfictitious}
\begin{split}
\overline{\bold{S}} &= 2\frac{\partial\Psi_{iso}({\overline{\bold{C}})}}{\partial\overline{\bold{C}}} \\
&= 2\frac{\partial\Psi_{iso}}{\partial\bar{I_1}} \frac{\partial\bar{I_1}}{\partial\overline{\bold{C}}}  + 2\frac{\partial\Psi_{iso}}{\partial\bar{I_2}} \frac{\partial\bar{I_2}}{\partial\overline{\bold{C}}} \\
&= \mu_1{\bold{I}} + \mu_2({\bar{I_1}\bold{I} - \overline{\bold{C}}}) \\
&= (\mu_1+\mu_2\bar{I_1})\bold{I} - \mu_2{\overline{\bold{C}}}
\end{split}
\end{equation}
Multiply with the 4th order projection tensor
\begin{equation}
\begin{split}
\mathbb{P}:\overline{\bold{S}} &= ( \mathbb{I} - \frac{1}{3}\bold{C}^{-1} \otimes \bold{C} ) : \overline{\bold{S}} \\
&= \overline{\bold{S}} - \frac{1}{3}( \overline{\bold{C}}^{-1} \otimes \overline{\bold{C}} ) : \overline{\bold{S}} \\
&= \overline{\bold{S}} - \frac{1}{3} \overline{\bold{C}}^{-1}  (\overline{\bold{C}} : \overline{\bold{S}}) \\
&= \overline{\bold{S}} - \frac{1}{3} \overline{\bold{C}}^{-1}  \left[(\mu_1+\mu_2\bar{I_1})\bar{I_1} - \mu_2(\bar{I_1}^2-2\bar{I_2})\right]\\
&= \overline{\bold{S}} -  \frac{1}{3}(\mu_1\bar{I_1} + 2\mu_2\bar{I_2}) \overline{\bold{C}}^{-1} 
\end{split}
\end{equation}
Therefore the isochoric part of the PK2 stress is
\begin{equation} \label{mSiso}
\bold{S_{iso}} = J^{-2/3}\mathbb{P}:\overline{\bold{S}} = J^{-2/3}\left[    - \frac{1}{3}(\mu_1\bar{I_1} + 2\mu_2\bar{I_2}) \overline{\bold{C}}^{-1}  + (\mu_1 + \mu_2\bar{I_1})\bold{I} - \mu_2\overline{\bold{C}} \right]
\end{equation}
Equation \ref{mSvol} and \ref{mSiso} are the expression we need in the displacement-based formulation. In the mixed formulation, use \ref{mSvol1} instead of \ref{mSvol}.

%
\subsubsection{Elasticity Tensor}
Insert our model into the definition of $\tilde{p}$ in Equation \ref{tildep}
\begin{equation}
\tilde{p} = p + J\frac{dp}{dJ} + \kappa(J-1) + \kappa{J} = \kappa{(2J - 1)}
\end{equation}
Substituting $p$ and $\tilde{p}$ into Equation \ref{Cvol}, we have
\begin{equation} \label{mCvol1}
\begin{split}
\mathbb{C}_{vol} &= J\kappa{(2J-1)} \bold{C}^{-1} \otimes \bold{C}^{-1} - 2J\kappa{(J-1)} \bold{C}^{-1} \odot \bold{C}^{-1}  \\
&= J^{-1/3}\kappa{(2J-1)} \overline{\bold{C}}^{-1} \otimes \overline{\bold{C}}^{-1} - 2J^{-1/3}\kappa{(J-1)} \overline{\bold{C}}^{-1} \odot \overline{\bold{C}}^{-1}
\end{split}
\end{equation}
In mixed formulation
\begin{equation} \label{mCvol2}
\begin{split}
\mathbb{C}_{vol} &= Jp\bold{C}^{-1}\otimes\bold{C}^{-1} - 2Jp\bold{C}^{-1}\odot\bold{C}^{-1} \\
&= J^{-1/3}p\overline{\bold{C}}^{-1}\otimes\overline{\bold{C}}^{-1} - 2J^{-1/3}p\overline{\bold{C}}^{-1}\odot\overline{\bold{C}}^{-1}
\end{split}
\end{equation}
Notice $\tilde{p}$ is replaced with $p$ because ${dp}/{dJ}$ is $0$.

Next we follow Equation \ref{Ciso} and \ref{barcc} to \ref{tracec} to derive $\mathbb{C}_{iso}$. First, find the second order derivative of $\Psi_{iso}$ with respect to $\overline{\bold{C}}$.
\begin{equation}
\begin{split}
\frac{\partial{\Psi_{iso}}}{\partial{\overline{\bold{C}}}} &= \frac{\mu_1}{2}\frac{\partial{\bar{I_1}}}{\partial{\overline{\bold{C}}}} +  \frac{\mu_2}{2}\frac{\partial{\bar{I_2}}}{\partial{\overline{\bold{C}}}} \\
&= \frac{\mu_1}{2}\bold{I} + \frac{\mu_2}{2}(\bar{I_1}\bold{I} - \overline{\bold{C}})
\end{split}
\end{equation}
\begin{equation}
\frac{\partial^2\Psi_{iso}(\overline{\bold{C}})}{\partial\overline{\bold{C}}\partial\overline{\bold{C}}} = 
\frac{\mu_2}{2}(\bold{I} \otimes \frac{\partial{\bar{I_1}}}{\partial{\overline{\bold{C}}}} - \mathbb{I}) = \frac{\mu_2}{2}(\bold{I} \otimes \bold{I} - \mathbb{I})
\end{equation}
Substituting into Equation \ref{barcc}
\begin{equation}
\overline{\mathbb{C}} = 4J^{-4/3}\frac{\partial^2\Psi_{iso}(\overline{\bold{C}})}{\partial{\overline{\bold{C}}}{\partial{\overline{\bold{C}}}}} = 2\mu_2{J^{-4/3}}(\bold{I} \otimes \bold{I} - \mathbb{I})
\end{equation}
The first part of the right-hand-side Equation \ref{Ciso} is
\begin{equation} \label{part1}
\begin{split}
\mathbb{P} : {\overline{\mathbb{C}}} : {\mathbb{P}}^T &= 2\mu_2{J^{-4/3}} (\mathbb{I} - \frac{1}{3}{\bold{C}}^{-1} \otimes {\bold{C}})
: (\bold{I} \otimes \bold{I} - \mathbb{I}) : (\mathbb{I} - \frac{1}{3}\overline{\bold{C}} \otimes {\bold{C}}^{-1}) \\
&= 2\mu_2{J^{-4/3}}[\mathbb{I} : (\bold{I} \otimes \bold{I}) : \mathbb{I} - \frac{1}{3}\mathbb{I} : (\bold{I} \otimes \bold{I}) : ({{\bold{C}}} \otimes {{\bold{C}}}^{-1}) - \mathbb{I} : \mathbb{I} : \mathbb{I} \\
&+ \mathbb{I} : \mathbb{I} : \frac{1}{3}({\bold{C}} \otimes {{\bold{C}}}^{-1}) - 
\frac{1}{3}({{\bold{C}}}^{-1} \otimes {\bold{C}} ) : (\bold{I} \otimes \bold{I}) : \mathbb{I}
- \frac{1}{3}({{\bold{C}}}^{-1} \otimes {\bold{C}} ) :  \mathbb{I} : \frac{1}{3}({{\bold{C}}} \otimes {{\bold{C}} }^{-1}) \\
&+ \frac{1}{3}({{\bold{C}}}^{-1} \otimes {\bold{C}} ) : (\bold{I} \otimes \bold{I}) : \frac{1}{3}({{\bold{C}}} \otimes {{\bold{C}}}^{-1} ) + \frac{1}{3}({{\bold{C}}} \otimes {{\bold{C}}}^{-1}) : \mathbb{I} : \mathbb{I}
] \\
&= 2\mu_2{J^{-4/3}}[\bold{I} \otimes \bold{I} - \frac{1}{3}{I_1}\bold{I} \otimes {{\bold{C}}}^{-1} - \mathbb{I} + \frac{1}{3}{\bold{C}} \otimes {{\bold{C}}}^{-1} -  \frac{1}{3}{I_1}{{\bold{C}}}^{-1} \otimes \bold{I} \\
&- \frac{1}{9}({\bold{C}} : {\bold{C}}){{\bold{C}}}^{-1} \otimes {{\bold{C}}} + \frac{1}{9}{{I_1}}^2{{\bold{C}}}^{-1} \otimes {{\bold{C}}}^{-1} +  \frac{1}{3}{{\bold{C}}}^{-1} \otimes {{\bold{C}}}] \\
&= 2\mu_2{J}^{-4/3}[\bold{I} \otimes \bold{I} - \mathbb{I} - \frac{1}{3}{I_1}({{\bold{C}}}^{-1} \otimes \bold{I} + \bold{I}\otimes{{\bold{C}}}^{-1} ) +
\frac{1}{3}({{\bold{C}}}^{-1} \otimes {\bold{C}} + {\bold{C}}\otimes{{\bold{C}}}^{-1} )  + \frac{2}{9}{I_2} {{\bold{C}}}^{-1} \otimes {{\bold{C}}}^{-1}] \\
&= 2\mu_2{J}^{-4/3}(\bold{I} \otimes \bold{I} - \mathbb{I}) - \frac{2}{3}\mu_2J^{-4/3}\bar{I_1}({\overline{\bold{C}}}^{-1} \otimes \bold{I} + \bold{I} \otimes {\overline{\bold{C}}}^{-1}) \\
&+
\frac{2}{3}\mu_2J^{-4/3}({\overline{\bold{C}}}^{-1} \otimes {\overline{\bold{C}}} + {\overline{\bold{C}}} \otimes {\overline{\bold{C}}}^{-1}) + \frac{4}{9}\mu_2\bar{I_2}J^{-4/3}({\overline{\bold{C}}}^{-1} \otimes {\overline{\bold{C}}}^{-1})
\end{split}
\end{equation}
Use Equation \ref{mfictitious} to derive the second part of the right-hand-side of Equation \ref{Ciso}
\begin{equation}
\begin{split}
Tr(J^{-2/3}\overline{\bold{S}}) &= J^{-2/3}{\overline{\bold{S}}} : {\overline{\bold{C}}} \\
&= [(\mu_1 + \mu_2{\bar{I_1}})\bold{I} - \mu_2{\overline{\bold{C}}}] : \overline{\bold{C}} \\
&= (\mu_1 + \mu_2\bar{I_1})\bar{I_1} - \mu_2{\overline{\bold{C}}} : {\overline{\bold{C}}} \\
&= (\mu_1 + \mu_2\bar{I_1})\bar{I_1} - \mu_2({\bar{I_1}}^2 - 2\bar{I_2}) \\
&= \mu_1\bar{I_1} + 2\mu_2\bar{I_2}
\end{split}
\end{equation}
Therefore the second part of the right-hand-side of Equation \ref{Ciso} is
\begin{equation} \label{part2}
\begin{split}
\frac{2}{3}Tr(J^{-2/3}\overline{\bold{S}})\tilde{\mathbb{P}} &= \frac{2}{3}(\mu_1\bar{I_1} + 2\mu_2\bar{I_2})({\bold{C}}^{-1} \odot {\bold{C}}^{-1} - \frac{1}{3}{\bold{C}}^{-1} \otimes {\bold{C}}^{-1}) \\
&= \frac{2}{3}J^{-4/3}(\mu_1\bar{I_1} + 2\mu_2\bar{I_2})({\overline{\bold{C}}}^{-1} \odot {{\overline{\bold{C}}}}^{-1} - \frac{1}{3}{{\overline{\bold{C}}}}^{-1} \otimes {\overline{{\bold{C}}}}^{-1})
\end{split}
\end{equation}
Plug in Equation \ref{mSiso} into Equation the third part of the right-hand-side of Equation \ref{Ciso}
\begin{equation} \label{part3}
\begin{split}
- \frac{2}{3}(\bold{C}^{-1}\otimes\bold{S}_{iso} + \bold{S}_{iso}\otimes \bold{C}^{-1})
&=
- \frac{2}{3}J^{-2/3} \{ {\bold{C}}^{-1} \otimes [-\frac{1}{3}(\mu_1\bar{I_1} + 2\mu_2\bar{I_2}){\overline{\bold{C}}}^{-1} + (\mu_1 + \mu_2\bar{I_1})\bold{I} - \mu_2{\overline{\bold{C}}}] \\
&+
[-\frac{1}{3}(\mu_1\bar{I_1} + 2\mu_2\bar{I_2}){\overline{\bold{C}}}^{-1} + (\mu_1 + \mu_2\bar{I_1})\bold{I} - \mu_2{\overline{\bold{C}}}] \otimes {\bold{C}}^{-1}\} \\
&=
- \frac{2}{3}J^{-4/3} [ -\frac{1}{3}(\mu_1\bar{I_1} + 2\mu_2\bar{I_2}) {\overline{\bold{C}}}^{-1} \otimes {\overline{\bold{C}}}^{-1} + (\mu_1 + \mu_2\bar{I_1}){\overline{\bold{C}}}^{-1}\otimes\bold{I} - \mu_2{\overline{\bold{C}}}^{-1} \otimes {\overline{\bold{C}}} \\
&-
\frac{1}{3}(\mu_1\bar{I_1} + 2\mu_2\bar{I_2}) {\overline{\bold{C}}}^{-1} \otimes {\overline{\bold{C}}}^{-1} + (\mu_1 + \mu_2\bar{I_1})\bold{I} \otimes {\overline{\bold{C}}}^{-1} - \mu_2{\overline{\bold{C}}} \otimes {\overline{\bold{C}}}^{-1}] \\
&=
 - \frac{2}{3}J^{-4/3} [ -\frac{2}{3}(\mu_1\bar{I_1} + 2\mu_2\bar{I_2})({\overline{\bold{C}}}^{-1} \otimes {\overline{\bold{C}}}^{-1}) + (\mu_1+\mu_2\bar{I_1})({\overline{\bold{C}}}^{-1} \otimes \bold{I} + \bold{I} \otimes {\overline{\bold{C}}}^{-1})\\
&- \mu_2({\overline{\bold{C}}}^{-1} \otimes {\overline{\bold{C}}}+{\overline{\bold{C}}} \otimes {\overline{\bold{C}}}^{-1}) ]
\end{split}
\end{equation}
Combining Equation \ref{part1}, \ref{part2} and \ref{part3}, we have
\begin{equation} \label{mCiso}
\begin{split}
\mathbb{C}_{iso} 
&= 
2\mu_2{J}^{-4/3}(\bold{I} \otimes \bold{I} - \mathbb{I}) - \frac{2}{3}\mu_2J^{-4/3}\bar{I_1}({\overline{\bold{C}}}^{-1} \otimes \bold{I} + \bold{I} \otimes {\overline{\bold{C}}}^{-1}) \\
&+
\frac{2}{3}\mu_2J^{-4/3}({\overline{\bold{C}}}^{-1} \otimes {\overline{\bold{C}}} + {\overline{\bold{C}}} \otimes {\overline{\bold{C}}}^{-1}) + \frac{4}{9}\mu_2\bar{I_2}J^{-4/3}({\overline{\bold{C}}}^{-1} \otimes {\overline{\bold{C}}}^{-1}) \\
&+
\frac{2}{3}J^{-4/3}(\mu_1\bar{I_1} + 2\mu_2\bar{I_2})({\overline{\bold{C}}}^{-1} \odot {{\overline{\bold{C}}}}^{-1} - \frac{1}{3}{{\overline{\bold{C}}}}^{-1} \otimes {\overline{{\bold{C}}}}^{-1}) \\
&+
\frac{4}{9}J^{-4/3} (\mu_1\bar{I_1} + 2\mu_2\bar{I_2})({\overline{\bold{C}}}^{-1} \otimes {\overline{\bold{C}}}^{-1}) - \frac{2}{3}J^{-4/3}(\mu_1+\mu_2\bar{I_1})({\overline{\bold{C}}}^{-1} \otimes \bold{I} + \bold{I} \otimes {\overline{\bold{C}}}^{-1})\\
&+ \frac{2}{3}J^{-4/3}\mu_2({\overline{\bold{C}}}^{-1} \otimes {\overline{\bold{C}}}+{\overline{\bold{C}}} \otimes {\overline{\bold{C}}}^{-1}) ] \\
&=
2\mu_2{J}^{-4/3}(\bold{I} \otimes \bold{I} - \mathbb{I}) + \frac{2}{3}(\mu_1\bar{I_1} + 2\mu_2\bar{I_2})J^{-4/3}({\overline{\bold{C}}}^{-1} \odot {\overline{\bold{C}}}^{-1} - \frac{1}{3}{\overline{\bold{C}}}^{-1} \otimes {\overline{\bold{C}}}^{-1}) \\
&-
\frac{2}{3}J^{-4/3}(\mu_1 + 2\mu_2\bar{I_1})({\overline{\bold{C}}}^{-1} \otimes \bold{I} + \bold{I} \otimes {\overline{\bold{C}}}^{-1}) + \frac{4}{3}\mu_2J^{-4/3}({\overline{\bold{C}}}^{-1} \otimes {\overline{\bold{C}}} + {\overline{\bold{C}}} \otimes {\overline{\bold{C}}}^{-1}) \\
&+
\frac{4}{9}J^{-4/3}(\mu_1\bar{I_1} + 3\mu_2\bar{I_2}){\overline{\bold{C}}}^{-1} \otimes {\overline{\bold{C}}}^{-1} \\
&=
2\mu_2{J}^{-4/3}(\bold{I} \otimes \bold{I} - \mathbb{I}) - \frac{2}{3}J^{-4/3}(\mu_1 + 2\mu_2\bar{I_1})({\overline{\bold{C}}}^{-1} \otimes \bold{I} + \bold{I} \otimes {\overline{\bold{C}}}^{-1}) + \frac{4}{3}\mu_2J^{-4/3}({\overline{\bold{C}}}^{-1} \otimes {\overline{\bold{C}}} + {\overline{\bold{C}}} \otimes {\overline{\bold{C}}}^{-1}) \\
&+
\frac{2}{9}J^{-4/3}(\mu_1\bar{I_1} + 4\mu_2\bar{I_2}) {\overline{\bold{C}}}^{-1} \otimes {\overline{\bold{C}}}^{-1}) + \frac{2}{3}J^{-4/3}(\mu_1\bar{I_1} + 2\mu_2\bar{I_2}){\overline{\bold{C}}}^{-1} \odot {\overline{\bold{C}}}^{-1} 
\end{split}
\end{equation} 
Equation \ref{mCvol1}/\ref{mCvol2} and \ref{mCiso} completes the derivation of elasticity tensor in reference configuration.

%
\subsection{Yeoh Model}
Yeoh model is motivated to simulate the mechanical behavior of carbon-black filled rubber vulcanizates with the typical stiffening effect in large strain domain. It's also widely used in mechanical engineering. We use the same volumetric part to account for compressibility as in Equation \ref{vol1} or \ref{vol2}. Consequently the volumetric part of both stress and elasticity tensor are the same as Mooney-Rivlin. Therefore, we will only show the derivation of the isochoric components.

%
\subsubsection{The Decoupled Form of Yeoh Model}
The isochoric part is written as
\begin{equation} \label{isopart}
\Psi_{iso} = c_1(\bar{I_1} - 3) + c_2(\bar{I_2} - 3)^2 + c_3(\bar{I_3} - 3)^3
\end{equation}
where $c_1$, $c_2$, $c_3$ are material constants. The shear modules is approximated by $\mu = 2c_1 + 4c_2(\bar{I_1} - 3) + 6c_3(\bar{I_1} - 3)^2$.

%
\subsubsection{Stress}
Following the same way, first derive the fictitious stress
\begin{equation} \label{yfictitious}
\begin{split}
\overline{\bold{S}} &= 2\frac{\partial\Psi_{iso}({\overline{\bold{C}})}}{\partial\overline{\bold{C}}} \\
&= [2c_1 + 4c_2(\bar{I_1} - 3) + 6c_3(\bar{I_1} - 3)^2] \bold{I} \\
\end{split}
\end{equation}
Then the isochoric part of the PK2 stress is obtained
\begin{equation} \label{ySiso}
\begin{split}
{\bold{S}}_{iso} &= J^{-2/3}(\mathbb{I} - \frac{1}{3}\bold{C}^{-1} \otimes \bold{C}) : \overline{\bold{S}}\\
&= J^{-2/3} [\overline{\bold{S}} - \frac{1}{3}({\overline{\bold{C}}}^{-1} \otimes  {\overline{\bold{C}}}) :  \overline{\bold{S}}] \\
&= J^{-2/3}[\overline{\bold{S}} - \frac{1}{3} \overline{\bold{C}}^{-1}  (\overline{\bold{C}} : \overline{\bold{S}})] \\
&= J^{-2/3}\{\overline{\bold{S}} - \frac{1}{3} \overline{\bold{C}}^{-1} [2c_1 + 4c_2(\bar{I_1} - 3) + 6c_3(\bar{I_1} - 3)^2] \bar{I_1}\}\\
&= J^{-2/3}[2c_1 + 4c_2(\bar{I_1} - 3) + 6c_3(\bar{I_1} - 3)^2] (\bold{I} - \frac{1}{3}\bar{I_1}{\overline{\bold{C}}}^{-1})
\end{split}
\end{equation}

%
\subsubsection{Elasticity Tensor}
The second order derivative of $\Psi_{iso}$ with respect to $\overline{\bold{C}}$ is
\begin{equation}
\begin{split}
\frac{\partial^2\Psi_{iso}(\overline{\bold{C}})}{\partial\overline{\bold{C}}\partial\overline{\bold{C}}} &= 
\bold{I} \otimes \frac{\partial[c_1 + 2c_2({\bar{I}}_1-3) + 3c_3({\bar{I}}_1-3)^2 ]}{\partial{\overline{\bold{C}}}} \\
&= \bold{I} \otimes [2c_2\bold{I} + 6c_3({\bar{I}}_1-3)\bold{I}]\\
&= [2c_2 + 6c_3({\bar{I}}_1-3)] \bold{I} \otimes \bold{I}
\end{split}
\end{equation}
The fictitious elasticity tensor is
\begin{equation}
\overline{\mathbb{C}} = 4J^{-4/3}\frac{\partial^2\Psi_{iso}(\overline{\bold{C}})}{\partial{\overline{\bold{C}}}{\partial{\overline{\bold{C}}}}} = 8J^{-4/3}[c_2 + 3c_3({\bar{I}}_1 - 3) ]\bold{I} \otimes \bold{I}
\end{equation}
The first part of the right-hand-side of Equation \ref{Ciso} is
\begin{equation} \label{ypart1}
\begin{split}
\mathbb{P} : \overline{\mathbb{C}} : {\mathbb{P}}^T &= 8J^{-4/3}[c_2 + 3c_3({\bar{I}}_1 - 3 )](\mathbb{I} - \frac{1}{3}{\overline{\bold{C}}}^{-1} \otimes {\overline{\bold{C}}}) : (\bold{I} \otimes \bold{I}): (\mathbb{I} - \frac{1}{3}{\overline{\bold{C}}} \otimes {\overline{\bold{C}}}^{-1}) \\
&= 8J^{-4/3}[c_2 + 3c_3({\bar{I}}_1 - 3 )][\mathbb{I}:(\bold{I} \otimes \bold{I}):\mathbb{I} - \mathbb{I}:(\bold{I} \otimes \bold{I}): \frac{1}{3} ({\overline{\bold{C}}} \otimes {\overline{\bold{C}}}^{-1}) \\
&- \frac{1}{3} ({\overline{\bold{C}}}^{-1} \otimes {\overline{\bold{C}}}) : (\bold{I} \otimes \bold{I}) : \mathbb{I} +  \frac{1}{9} ({\overline{\bold{C}}}^{-1} \otimes {\overline{\bold{C}}}) : (\bold{I} \otimes \bold{I}) : ({\overline{\bold{C}}} \otimes {\overline{\bold{C}}}^{-1}) ] \\
&= 8J^{-4/3}[c_2 + 3c_3({\bar{I}}_1 - 3 )] [\bold{I} \otimes \bold{I} - \frac{{\bar{I}}_1}{3}(\bold{I} \otimes {\overline{\bold{C}}}^{-1} +{\overline{\bold{C}}}^{-1}  \otimes  \bold{I}) + \frac{{{\bar{I}}_1}^2}{9} {\overline{\bold{C}}}^{-1} \otimes {\overline{\bold{C}}}^{-1} ]
\end{split}
\end{equation}
Using the fictitious PK2 stress derived in Equation \ref{yfictitious}, we can obtain the second part of the right-hand-side of Equation \ref{Ciso}
\begin{equation} \label{ypart2}
\begin{split}
\frac{2}{3}Tr(J^{-2/3}\overline{\bold{S}})\tilde{\mathbb{P}} &= \frac{2}{3}J^{-2/3}(\overline{\bold{S}} : \bold{C})\tilde{\mathbb{P}} \\ 
&= \frac{2}{3} (\overline{\bold{S}} : \overline{\bold{C}})\tilde{\mathbb{P}} \\
&= \frac{2}{3} [2c_1 + 4c_2(\bar{I_1} - 3) + 6c_3(\bar{I_1} - 3)^2] (\bold{I} : \overline{\bold{C}})
\tilde{\mathbb{P}} \\
&= [\frac{4}{3}c_1{\bar{I}}_1 + \frac{8}{3}c_2({\bar{I}}_1 - 3){\bar{I}}_1 + 4c_3({\bar{I}}_1 - 3)^2{\bar{I}}_1]({\overline{\bold{C}}}^{-1} \odot {\overline{\bold{C}}}^{-1} - \frac{1}{3}{\overline{\bold{C}}}^{-1} \otimes {\overline{\bold{C}}}^{-1}) \\
&= J^{-4/3} \{ [\frac{4}{3}c_1{\bar{I}}_1 + \frac{8}{3}c_2({\bar{I}}_1 - 3){\bar{I}}_1 + 4c_3({\bar{I}}_1 - 3)^2{\bar{I}}_1]{\overline{\bold{C}}}^{-1} \odot {\overline{\bold{C}}}^{-1} \\
&-  [\frac{4}{9}c_1{\bar{I}}_1 + \frac{8}{9}c_2({\bar{I}}_1 - 3){\bar{I}}_1 + \frac{4}{3}c_3({\bar{I}}_1 - 3)^2{\bar{I}}_1]{\overline{\bold{C}}}^{-1} \otimes {\overline{\bold{C}}}^{-1} \} 
\end{split}
\end{equation}
The third part of the right-hand-side of Equation \ref{Ciso} is
\begin{equation} \label{ypart3}
\begin{split}
-\frac{2}{3}({\bold{C}}^{-1} \otimes {\bold{S}}_{iso} + {\bold{S}}_{iso} \otimes {{\bold{C}}^{-1}}^{-1}) 
&= 
-\frac{2}{3}J^{-4/3}({\overline{\bold{C}}}^{-1} \otimes {\overline{\bold{S}}}_{iso} + {\overline{\bold{S}}}_{iso} \otimes {\overline{{\bold{C}}}^{-1}}) \\
&=
-\frac{2}{3}J^{-4/3}[2c_1 + 4c_2({\bar{I}}_1 - 3) + 6c_3({\bar{I}}_1 - 3)^2]
[{\overline{\bold{C}}}^{-1} \otimes (\bold{I} - \frac{1}{3}{\bar{I}_1}{\overline{\bold{C}}}^{-1}) + 
(\bold{I} - \frac{1}{3}{\bar{I}_1}{\overline{\bold{C}}}^{-1}) \otimes {\overline{\bold{C}}}^{-1}]\\
&=
J^{-4/3} \{ [-\frac{4}{3}c_1 - \frac{8}{3}c_2({\bar{I}}_1 - 3) - 4c_3({\bar{I}}_1 - 3)^2] ({\overline{\bold{C}}}^{-1} \otimes \bold{I} + \bold{I} \otimes {\overline{\bold{C}}}^{-1}) \\
&+ [\frac{8}{9}c_1{\bar{I}}_1 + \frac{16}{9}c_2({\bar{I}}_1 - 3){\bar{I}}_1 + \frac{8}{3}c_3({\bar{I}}_1 - 3)^2{\bar{I}}_1 ] {\overline{\bold{C}}}^{-1} \otimes {\overline{\bold{C}}}^{-1}
\}
\end{split}
\end{equation}
Combining \ref{ypart1}, \ref{ypart2} and \ref{ypart3}, we have the isochoric part of elasticity tensor
\begin{equation} \label{yCiso}
\begin{split}
{\mathbb{C}}_{iso} &= J^{-4/3} \{
[8c_2+24c_3({\bar{I}}_1 - 3)] \bold{I} \otimes \bold{I} \\
&- [\frac{4}{3}c_1 + (\frac{16}{3}{\bar{I}}_1 - 3)c_2 + 12({\bar{I}}_1 - 1)({\bar{I}}_1 - 3)c_3]({\overline{\bold{C}}}^{-1} \otimes \bold{I} + \bold{I} \otimes {\overline{\bold{C}}}^{-1})\\
&+ [\frac{4}{9}{\bar{I}}_1c_1 + (\frac{16}{9}{{\bar{I}}_1}^2 - \frac{8}{3}{\bar{I}}_1)c_2 + 4{\bar{I}}_1({\bar{I}}_1 - 1)({\bar{I}}_1 - 3)c_3] {\overline{\bold{C}}}^{-1} \otimes {\overline{\bold{C}}}^{-1} \\
&+ [\frac{4}{3}c_1{\bar{I}}_1 + \frac{8}{3}c_2({\bar{I}}_1 - 3){\bar{I}}_1 + 4c_3({\bar{I}}_1 - 3)^2{\bar{I}}_1]{\overline{\bold{C}}}^{-1} \odot {\overline{\bold{C}}}^{-1}
\}
\end{split}
\end{equation}
Equation \ref{yCiso} and Equation \ref{mCvol1} or \ref{mCvol2} gives the elasticity tensor in reference configuration.

In these two examples we show the derivation of stress and elasticity tensor for isotropic hyperelastic models written as functions of invariants. Both potential functions are polynomials. Some other isotropic models are exponential functions and some are combination of both forms. The procedures can be easily applied to these models. For models expressed in principal stretches, the method is still the same, we only have to find the derivatives of strain energy function with respect to principal stretches.

Yet not all biomaterials are isotropic. For example, heart muscle has strong directional properties, therefore it is regarded as anisotropic. Materials like this are modeled as a combination of a ground substance and one or more families of fibers which are continuously arranged in the ground material. In literature, this kind of materials are referred as transversely isotropic materials. In the next subsection, we will derive the stress and elasticity tensor for a composite material with two families of fibers.

%
\subsection{Holzapfel-Gasser-Ogden Model}
The Holzapfel-Gasser-Ogden (HGO) model is widely used to model soft biological tissues, for example, arterial tissue. Such materials are usually almost incompressible, so they can modeled in decoupled form to allow a slight compressibility. HGO model uses neo-Hookean model as the isotropic ground material, and the anisotropic part consists two families of fibers whose directions are represented by unit vectors $\bold{a}_{04}$ and $\bold{a}_{06}$ in the reference configuration. As the material deforms, these vectors deform accordingly and are expressed as unit vectors $\bold{a}_4$ and $\bold{a}_6$ in the current configuration. $\bold{a}_{0i}$ and $\bold{a}_{i}$ ($i = 4, 6$) are related by 
\begin{equation} \label{vector}
\lambda\bold{a}_i = \bold{F}\bold{a}_{0i}
\end{equation} 
where $i = 4, 6$, $\lambda$ is the stretch of the original fiber.
