\section{Concluding Remarks}
\label{conclusions}
This review paper shows a comprehensive overview and finite element implementation of hyperelastic models for biomaterials. In order to implement any model with finite element method efficiently, one has to find a general way to derive the stress and elasticity tensors. In this paper, we presented a systematic way to do that by decomposing the strain-energy function, stress tensor, and the elasticity tensor into volumetric and isochoric parts. We derived the stress and elasticity tensors for the Mooney-Rivlin, Yeoh, and HGO model as examples. These models are popular in biomechanics and cover both isotropic and anisotropic models. Incompressibility in biomaterials was also discussed, where the tangent stiffness matrix of the standard displacement-based formulation is often ill-conditioned. A detailed work of the displacement/pressure mixed formulation was presented. Three numerical expereiments were examined that include biaxial tension, vessel expansion, and $2$-layer vessel expansion. These experiments involved Neo-Hookean, Mooney-Rivlin, Yeoh and HGO models. Their comparisons to analytical solutions and existing numerical solutions yielded excellent agreement. The $2$-layer vessel example also demonstrated the effects of the residual stress in vessels with anisotropy. In summary, this paper addressed the challenges in the implementation of hyperelastic models using the existing literature by providing a framework with detailed procedure. It will be useful to the researchers in computational biomechanics. Following the systematic approach to derive stress and elasticity tensors, as well as the detailed illustration of the finite element implementation, readers should be able to implement any hyperelastic model with displacement-based formulation, mixed formulation, or as user-define functions in finite element packages. 

