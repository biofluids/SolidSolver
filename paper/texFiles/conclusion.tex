\section{Concluding Remarks}
Hyperelastic models are often used to model biological materials. There are two key difficulties in solving hyperelastic problems using finite element method. One is deriving the stress tensor and the elasticity tensor, the other is the illness of the tangent matrix in near incompressibility. The first problem has been solved in particular with many different models, however, to the author's knowledge, there is no paper thoroughly walk through a standard way to derive the stress and elasticity formulas and in many papers the elasticity tensors are totally ignored, leaving the readers confused. Regarding the second problem, although displacement/pressure mixed formulation has been proposed as early as 1990s, few papers are written to explain how to implement this formulation in matrix form. 

Motivated in giving the readers a clear and followable way to solve any hyperelastic material with mixed formulation, we reviewed the general procedures to derive the stress and elasticity tensor. With three different example, we show how to substitute the specific models into the general formulations. Both isotropic and anisotropic models are covered in these examples. We also reviewed the theory of principle of virtual work and principle of stationary potential energy. With these theories, we introduced the origin of mixed formulation and further wrote it in matrix form. Along the way we discussed the pressure load which is common in biomechanics. Pressure load is more difficult than the obvious given traction because it involves the deformation of the loaded surface. To complete the finite element formulation in solving hyperelastic material problems, we linearized the principle of stationary potential energy. The published papers seldom covers the linearization of pressure load, which we also covered in the linearization process. 

Using the procedures we summarized in this paper, we are able to solve any hyperelastic model, no matter of isotropic or anisotropic. To validate our method, two sets of tests are performed. The first one is the biaxial tension test which is often used to calibrate the material parameters in practice. For isotropic models, the computational results are exactly the same as 
the analytical solutions; for anisotropic model, which lacks an analytical solution, the results are as expected. The second set of tests is the cylindrical pressure vessel, which is a common situation in biomechanics. As a benchmark, we computed a series of cases for Mooney-Rivlin model under different pressures. The computational deformation and stress are in very good agreement with the theoretical solutions. Then we also demonstrated the effectiveness of our derivations for Yeoh and HGO models based on the benchmark. 

At last, we briefly introduced the finite element formulation in the updated Lagrange form to be consulted by the readers work in current coordinates.   
