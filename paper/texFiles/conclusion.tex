\section{Concluding Remarks}
Hyperelastic models are widely used for biomaterials. A variety of both isotropic and anisotropic models have been proposed, which have different advantages. In order to implement any model with finite element method efficiently, one has to find a general way to derive the stress and elasticity tensors. In this paper, we review a systematic way to do that by decomposing the strain-energy function, stress tensor, and the elasticity tensor into volumetric and isochoric parts. Once the derivatives of the strain-energy function with respect to the strain are obtained using chain rule, plug them into the equations presented in Section \ref{general}, the stress and elasticity tensors will immediately be derived. We derive the stress and elasticity tensors for Mooney-Rivlin, Yeoh, and HGO model as examples. These models are popular in biomechanics and cover both isotropic and anisotropic models. Readers can implement any hyperelastic model following the same procedures. The workflow is not only useful to the researchers who develop their own finite element code, but also to those who work with commercial or open-source packages and want to implement new models through user-define functions.

There is another practical problem troubling researchers in biomechanics. Biomaterials are often nearly incompressible, in which case the tangent stiffness matrix of the standard displacement-based formulation is often ill-conditioned. Even if the compressibility is relatively large, in many circumstances locking can be a serious problem in displacement-based finite element formulation. To help researchers

We also review the basic principles to derive finite element formulations in Section \ref{formulation}. Based on the principle of stationary potential energy, we review the displacement/pressure mixed formulation in detail. 