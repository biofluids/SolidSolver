\section{Concluding Remarks}
\label{conclusions}
Hyperelastic models are widely used for biomaterials. A variety of both isotropic and anisotropic models have been proposed, which have different advantages. In order to implement any model with finite element method efficiently, one has to find a general way to derive the stress and elasticity tensors. In this paper, we review a systematic way to do that by decomposing the strain-energy function, stress tensor, and the elasticity tensor into volumetric and isochoric parts. Once the derivatives of the strain-energy function with respect to the strain are obtained using chain rule, plug them into the equations presented in Section \ref{general}, the stress and elasticity tensors will immediately be derived. We derive the stress and elasticity tensors for Mooney-Rivlin, Yeoh, and HGO model as examples. These models are popular in biomechanics and cover both isotropic and anisotropic models. Readers can implement any hyperelastic model following the same procedures. The workflow is not only useful to the researchers who develop their own finite element code, but also to those who work with commercial or open-source packages and want to implement new models through user-define functions.

There is another practical problem troubling researchers in biomechanics. Biomaterials are often nearly incompressible, in which case the tangent stiffness matrix of the standard displacement-based formulation is often ill-conditioned. Even if the compressibility is relatively large, in many circumstances locking can be a serious problem in displacement-based finite element formulation. The cure to this problem is displacement/pressure mixed formulation. Therefore mixed formulation is essential to computational biomechanics. We review the derivation and implementation of mixed formulation in Section \ref{formulation}, based on the basic principle of stationary potential energy. In this section, the implementation of hydrostatic pressure including its linearization is also introduced since hydrostatic pressure load is common in biomechanics.

In Section \ref{experiments} we present three sets of numerical experiments to validate our implementation. First we examine all these three models with biaxial tension test. For the isotropic models, we compute the stress-stretch curve and compare with the analytical solutions under the assumption that the material is incompressible. The computational results correspond to the analytical solutions perfectly, and stiffening effect is observed in Yeoh model as expected; for the anisotropic model, we compute the stress-stretch curve with different fiber directions. The computational results meet our expectation too. The second experiment is designed for isotropic models. The expansion of a $2$D cylindrical vessel is considered with plain strain assumption. As a benchmark, the computational results of Mooney-Rivlin model are compared with the analytical solution in detail. A wide range of internal pressures is studied. It is shown that the numerical solutions at all nodes agree with analytical solutions very well. Furthermore, we model a specific kind of rubber with Neo-Hookean, Mooney-Rivlin and Yeoh models and run the experiment again. Different models differ with each other mainly in axial stress. The performance of Yeoh model and Neo-Hookean are especially similar since they both use only one invariant and the strain is not large enough for Yeoh model to demonstrate the stiffening effect. At last, we model an artery vessel with HGO model. The artery is modeled as a $3$D cylinder with $2$ layers. Two families of fiber are distributed on the curvilinear circumferential surface. The effect of pre-stretching is studied. The pre-stretching significantly changes the displacement and stress distribution in the vessel except for the radial stress. Overall pre-stretching causes the difference of the inner wall and the outer wall to be greater. In this study, the computational results are compared to commercial software COMSOL, and agreement is achieved.

This paper should be useful to the researchers in computational biomechanics. Following the systematic approach to derive stress and elasticity tensors, as well as the detailed illustration of the finite element implementation, readers should be able to implement any hyperelastic model with displacement-based formulation, mixed formulation, or as user-define functions in finite element packages. 