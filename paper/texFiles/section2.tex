\section{Stress and Elasticity Tensors in Hyperelastic Models} \label{general}
In this section, we briefly review the hyperelastic models and their decomposition, and the general approach to derive stress and elasticity tensors, followed by three examples of different models: Mooney-Rivlin model, Yeoh model and Holzapfel-Gasser-Ogden model. The first two are isotropic models written as polynomial functions of invariants. There are also isotropic models written as exponential functions or combination of both forms \cite{Fung5, Demiray, Westmann}. 
Yet not all biomaterials are isotropic. For example, the stomach wall tissues in pigs are found to be direction-dependent \cite{Zhao}. Heart muscles have strong directional properties as well \cite{Ramesh}. Materials like these are modeled as a combination of a ground substance and one or more families of fibers which are continuously arranged in the ground material. In literature, this kind of materials are referred as transversely isotropic materials. HGO model is one of the anisotropic models with two families of fibers. To model anisotropy, two pseudo-invariants are introduced. However, the procedure of the derivation of the stress and elasticity tensors remain the same for both isotropic and anisotropic materials.

%
\subsection{Hyperelastic Models}
Generally in continuum mechanics we use constitutive equations to describe the stress components in terms of other functions such as strain. This functional relationship distinguishes different types of materials. For hyperelastic material, we postulate there exists a Helmholtz free-energy function $\Psi$, which is defined per unit reference volume. If $\Psi$ is uniquely determined by the deformation tensor $\bold{F}$ or other strain tensor, it is called strain-energy function. For example, all isotropic models are functions of the right Cauchy-Green tensor, which is defined as $\C = {\F}^T{\F}$. If the strain is given, the stress can be uniquely determined from the strain-energy function. Our discussion will be focused on rubber-like materials as often the case for biomaterials which are modeled as incompressible or nearly incompressible. In this case, the strain-energy function is postulated to have a unique decoupled form:

\begin{equation} \label{energy_split}
\Psi(\C) = \Psi_\mathrm{vol}(J) + \Psi_\mathrm{iso}(\Cb)
\end{equation}
where $\Psi_\mathrm{vol}(J)$ and $\Psi_\mathrm{iso}(\Cb)$ are volumetric and isochoric responses of the material, respectively. $J$ is the Jacobian or the volume ratio defined as the determinant of the deformation tensor $\bold{F}$, and $\overline{\bold{C}}$ is the modified right Cauchy-Green tensor defined as 
$\overline{\bold{C}}  = J^{-2/3} \bold{C}$. Likely the modified counterpart of the deformation is written as $\overline{\bold{F}} = J^{-1/3} \bold{F}$. The modified tensors are associated with the volume-preserving, because their determinants are $1$. The concept of decomposition of the deformation and strain tensors is necessary for the mixed formulation because we need to deal with the volumetric and isotropic parts individually.

In the displacement-based formulation, the volumetric part of the strain-energy function is a penalty to allow for a small compressibility:
\begin{equation} \label{penalty}
\Psi_\mathrm{vol} = \kappa G(J)
\end{equation}
where $\kappa$ is the penalty parameter and can be interpreted as the bulk modulus, $G(J)$ is the penalty function and may adopt the simple form:
\begin{equation} \label{penalty2}
G(J) = \frac{1}{2}(J - 1)^2
\end{equation}

In the displacement/pressure mixed formulation, the volumetric part of the strain-energy function acts as a Lagrange condition to enforce the incompressibility $J -1 = 0$:
\begin{equation} \label{Lagrange}
\Psi_\mathrm{vol} = p(J-1)
\end{equation}
where $p$ is a Lagrange multiplier and can be identified as the hydrostatic pressure. For nearly incompressible materials, $p$ can be determined from the deformation. However, for incompressible materials, it has to be determined from the equilibrium equation and any imposed boundary conditions. 

Furthermore, if the material is isotropic, the strain-energy function can be expressed as a function of the three invariants of the right (or left) Cauchy-Green tensor. 
These invariants are 
$I_1 = \mathrm{tr}(\bold{C}) = \mathrm{tr}(\bold{B})$, $I_2 = \frac{1}{2}(I_1^2 - \mathrm{tr}(\bold{C}^2)) = \frac{1}{2}(I_1^2 - \mathrm{tr}(\bold{B}^2))$, and $I_3 = \mathrm{det}(\bold{C}) = \mathrm{det}(\bold{B}) = J^2$, where $\bold{B}$ is the left Cauchy-Green tensor defined as $\bold{B} = \F{\F}^T$. 
They also have their modified counterparts: $\bar{I}_1 = J^{-2/3}I_1$, $\bar{I}_2 = J^{-4/3}I_2$, and $\bar{I}_3 = 1$. In fact, most isotropic models can be expressed as:
\begin{equation}
\Psi = \Psi_\mathrm{vol}(J) + \Psi_\mathrm{iso}(\bar{I}_1, \bar{I}_2)
\end{equation}

%
\subsection{Stress Evaluation and Elasticity Tensor} \label{general_stress}
Based on the postulation of strain-energy function, it follows that the work done on hyperelastic materials in a dynamic process within a time interval $[t_1, t_2]$ is the difference between the strain-energy in the two states. It can also be proven that the work can be written as the integration of the tensor product of conjugate stress-strain rate pairs:

\begin{equation}
\Psi(\F_2) - \Psi(\F_1) = \delta{W_\mathrm{int}} = \int_{t_1}^{t_2}\bold{P}:\dot{\bold{F}}dt = \int_{t_1}^{t_2}\bold{S}:\dot{\bold{E}}dt = 
\int_{t_1}^{t_2}\boldsymbol{\sigma}:\dot{\bold{e}}dt
\end{equation}
where $\bold{P}$ is the first Piola-Kirchhoff stress, $\bold{S}$ is the second Piola-Kirchhoff (PK2) stress tensor, $\bold{E}$ is the Green-Lagrange strain tensor defined as $\bold{E} = \frac{1}{2}(\C - \I)$, and $\bold{e}$ is Euler-Almansi strain tensor which is defined as $\bold{e} = \frac{1}{2}( \bold{I} - {\bold{F}}^{-T}{\bold{F}}^{-1} )$.

If the material is homogeneous, the PK2 stress $\bold{S}$ can be determined by: 
\begin{equation}
\bold{S}  = \frac{\partial\Psi(\bold{E})}{\partial{\bold{E}}} = 2\frac{\partial\Psi(\bold{C})}{\partial{\bold{C}}}\end{equation}
Not surprisingly, the PK2 stress can also be written in a decoupled form with volumetric and isochoric parts:

\begin{subequations}
\label{S}
\begin{align}
\bold{S} &=  2\frac{\partial{\Psi(\overline{\bold{C}})}}{\partial{\bold{C}}} = \bold{S}_\mathrm{vol}  + \bold{S}_\mathrm{iso} 
\label {Stotal} \\
\bold{S}_\mathrm{vol} &= 2\frac{\partial{\Psi_\mathrm{vol}(J)}}{\partial{\bold{C}}} = Jp{\bold{C}} ^{-1} = J^{1/3}p{\overline{\bold{C}}}^{-1} 
\label{Svol} \\
\bold{S}_\mathrm{iso}  &= 2\frac{\partial{\Psi_\mathrm{iso}({\overline{\bold{C}})}}}{\partial{\bold{C}}} = J^{-2/3}\mathbb{P}:\overline{\bold{S}}
\label{Siso}
\end{align}
\end{subequations}
where  $\overline{\bold{S}}$ is the fictitious PK2 stress defined as
$\overline{\bold{S}} = 2{\partial\Psi_\mathrm{iso}({\overline{\bold{C}})}} / {\partial\overline{\bold{C}}}$
and $\mathbb{P}$ is the projection tensor defined as $\mathbb{P} = \mathbb{I} - \frac{1}{3}\bold{C}^{-1} \otimes \bold{C} = \mathbb{I} - \frac{1}{3}\overline{\bold{C}}^{-1} \otimes \overline{\bold{C}} $ in which $\otimes$ is the dyadic multiplication symbol and $\mathbb{I}$ is the fourth order identity tensor, i.e.,  $\mathbb{P}_{ijkl} =  \frac{1}{2}(\delta_{ik}\delta_{jl} + \delta_{il}\delta_{jk}) - \frac{1}{3} {\overline{{C}}^{-1}}_{ij} {\overline{C}}_{kl}$.

Note that if the material is completely incompressible, the hydrostatic pressure $p$ is independent from the displacement $\bold{u}$; otherwise $p$ can be determined from the displacement even when it is treated as an independent variable by:

\begin{equation} \label{pressure}
p = \frac{d\Psi_\mathrm{vol}(J)}{dJ} = \kappa\frac{dG(J)}{dJ}
\end{equation}
For the material that is not completely incompressible, either displacement-based formulation or mixed formulation can be used. If we explicitly substitute $p$ with Equation \ref{pressure}, we end up with a displacement-based formulation as $\bold{u}$ is the only unknown to be solved; If we leave $p$ as an unknown and use Equation \ref{pressure} as a relaxed constraint in the Lagrangian multiplier method, we will have a displacement/pressure mixed formulation.

%
The solutions in nonlinear finite element methods are often obtained incrementally with Newton's method. The elasticity tensors is crucial in the implementation of Newton's method.
In the material description or the reference configuration, the elasticity tensor $\CC$ is defined as the gradient of the PK2 stress to its work conjugate strain, the Green-Lagrange strain:

\begin{equation}
\mathbb{C} = \frac{\partial{\bold{S}(\bold{E})}}{\partial{\bold{E}}} =  2\frac{\partial{\bold{S}(\bold{C})}} {\partial{\bold{C}}} = 4 \frac{\partial^2{\Psi(\bold{C})}}{{\partial{\bold{C}}}{\partial{\bold{C}}}} \label{def_elasticity_tensor}
\end{equation}

Similar to $\Psi$ and $\PK$, the elasticity tensor can also be rewritten in the decoupled form: 

\begin{subequations} 
\label{C}
\begin{align}
\mathbb{C} &= 2\frac{\partial{\bold{S}}}{\partial{\bold{C}}} = \mathbb{C}_\mathrm{vol} + \mathbb{C}_\mathrm{iso} 
\label{Ctotal} \\
\mathbb{C}_\mathrm{vol} &= 2\frac{\partial{\bold{S}_\mathrm{vol}}}{\partial{\bold{C}}} =
J^{-1/3}\tilde{p}{\overline{\bold{C}}}^{-1}\otimes {\overline{\bold{C}}}^{-1} - 2J^{-1/3}p{\overline{\bold{C}}}^{-1}\odot {\overline{\bold{C}}}^{-1}  \label{Cvol} \\
\mathbb{C}_\mathrm{iso} &= 2\frac{\partial{\bold{S}_\mathrm{iso}}}{\partial{\bold{C}}} =
\mathbb{P} : \overline{\mathbb{C}} : \mathbb{P}^T + \frac{2}{3}\mathrm{Tr}(J^{-2/3}\overline{\bold{S}})\tilde{\mathbb{P}} - \frac{2}{3}J^{-2/3}(\overline{\bold{C}}^{-1}\otimes\bold{S}_\mathrm{iso} + \bold{S}_\mathrm{iso}\otimes {\overline{\bold{C}}}^{-1})
\label{Ciso} 
\end{align}
\end{subequations}
where $\tilde{p}$ is defined as $\tilde{p} = p + J{dp}/{dJ}$. Note that in the mixed formulation $\tilde{p} = p$ because $p$ is independent of $J$. $\odot$ is an operator defined as the derivative of the inverse of a second order tensor with respect to itself, i.e. 
${\overline{\bold{C}}}^{-1}\odot {\overline{\bold{C}}}^{-1} = - {\partial{{\overline{\bold{C}}}^{-1}}} / {\partial{{\overline{\bold{C}}}}}$; 
$\overline{\mathbb{C}}$ is the fourth-order fictitious elasticity tensor defined as 
$\overline{\mathbb{C}} = 2J^{-2/3}\dfrac{\partial{\overline{\bold{S}}}}{\partial{\overline{\bold{C}}}} = 4J^{-4/3} \dfrac{\partial^2\Psi_\mathrm{iso}(\overline{\bold{C}})} {\partial{\overline{\bold{C}}}{\partial{\overline{\bold{C}}}}} $;
$\mathrm{Tr}(\bullet)$ is the trace defined as $\mathrm{Tr}(\bullet) = (\bullet) : \bold{C}$;
and $\tilde{\mathbb{P}}$ is the modified projection tensor of fourth-order defined as 
$\tilde{\mathbb{P}} = \bold{C}^{-1} \odot \bold{C}^{-1} -  \frac{1}{3}\bold{C}^{-1} \otimes \bold{C}^{-1} = J^{-4/3}(
{\overline{\bold{C}}}^{-1} \odot {\overline{\bold{C}}}^{-1} -  \frac{1}{3} {\overline{\bold{C}}}^{-1} \otimes {\overline{\bold{C}}}^{-1})$.

Therefore, once the stress expressions $\PK_\mathrm{vol}$ and $\PK_\mathrm{iso}$ are obtained, we can evaluate the elasticity tensor $\CC$ by substituting Equation \ref{S} into Equation \ref{C}.

%
\subsection {Examples}
\subsubsection{Mooney-Rivlin Model}
Originally derived by Mooney \cite{Mooney} and Rivlin \cite{Rivlin}, Mooney-Rivlin model is a polynomial function of $\bar{I}_1$ and $\bar{I}_2$. The first-order Mooney-Rivlin is the most widely used to model biological tissues because of its simplicity. However it was found that the first-order Mooney-Rivlin model is sufficient to characterize the nonlinear behavior of many tissues. Successful applications include animal organs \cite{Wall}, muscular tissue \cite{Bols2}, vessels \cite{Navidbakhsh} etc. As a linear function of the invariants, it is possible to find analytical solutions for some simple cases. The Mooney-Rivlin model is written as:

\begin{subequations}
\label{Mooney}
\begin{align}
\Psi_\mathrm{vol} &= \frac{\kappa}{2}(J - 1)^2 \label{vol11} \\
\Psi_\mathrm{iso} &= \frac{\mu_1}{2}(\bar{I_1} - 3) + \frac{\mu_2}{2}(\bar{I_2} - 3) \label{iso1}
\end{align}
\end{subequations}
where $\mu_1$, $\mu_2$ and $\kappa$ are material constants. For small deformations, the shear modulus and bulk modulus can be approximated by $\mu_1+\mu_2$ and $\kappa$. When using mixed formulation Equation \ref{vol11} is rewritten as Equation \ref{Lagrange} where $\Psi_\mathrm{vol} = p(J - 1) $.

To derive the expression for the PK2 stress, we will first decide $S_\mathrm{vol}$ and $S_\mathrm{iso}$ separately. The volumetric part of the PK2 stress that is defined in Equation \ref{Svol}, can be shown in the following forms:
\begin{subequations}
\label{Svol1}
\begin{align}
\bold{S}_\mathrm{vol} &= J^{1/3}p{\overline{\bold{C}}}^{-1} \label{Svol11}\\
		      &= {J^{1/3}(J-1)}\kappa {\overline{\bold{C}}}^{-1} \label{Svol12}
\end{align}
\end{subequations}
Equation \ref{Svol11} and \ref{Svol12} are used in the mixed and displacement-based formulations, respectively.

To obtain the isochoric stress $\bold{S}_\mathrm{iso}$ defined in Equation \ref{Siso}, we need to first find the fictitious stress:
\begin{equation} \label{Sbar1}
\begin{split}
\overline{\bold{S}} &= 2\frac{\partial\Psi_\mathrm{iso}({\overline{\bold{C}})}}{\partial\overline{\bold{C}}} \\
&= 2\frac{\partial\Psi_\mathrm{iso}}{\partial\bar{I_1}} \frac{\partial\bar{I_1}}{\partial\overline{\bold{C}}}  + 2\frac{\partial\Psi_\mathrm{iso}}{\partial\bar{I_2}} \frac{\partial\bar{I_2}}{\partial\overline{\bold{C}}} \\
&= \mu_1{\bold{I}} + \mu_2({\bar{I_1}\bold{I} - \overline{\bold{C}}}) \\
&= (\mu_1+\mu_2\bar{I_1})\bold{I} - \mu_2{\overline{\bold{C}}}
\end{split}
\end{equation}
With the fourth-order projection tensor $\mathbb{P}$, the isochoric PK2 stress becomes:

\begin{equation} \label{Siso1}
\begin{split}
\bold{S}_\mathrm{iso}
&= J^{-2/3}\mathbb{P}:\overline{\bold{S}} \\
&= J^{-2/3}\left( \mathbb{I} - \frac{1}{3}{\overline{\bold{C}}}^{-1} \otimes \overline{\bold{C}} \right) : \overline{\bold{S}} \\
&= J^{-2/3} \left[\overline{\bold{S}} - \frac{1}{3}( \overline{\bold{C}}^{-1} \otimes \overline{\bold{C}} ) : \overline{\bold{S}} \right]  \\
&= J^{-2/3}\left[ \overline{\bold{S}} - \frac{1}{3}(\overline{\bold{C}} : \overline{\bold{S}})\overline{\bold{C}}^{-1} \right]  \\
&= J^{-2/3}\left\{ \overline{\bold{S}} - \frac{1}{3} [(\mu_1+\mu_2\bar{I_1})\bar{I_1} - \mu_2(\bar{I_1}^2-2\bar{I_2})]\overline{\bold{C}}^{-1} \right\} \\
&= J^{-2/3}\left[\overline{\bold{S}} -  \frac{1}{3}(\mu_1\bar{I_1} + 2\mu_2\bar{I_2}) \overline{\bold{C}}^{-1}\right] \\
&= J^{-2/3} \left[    - \frac{1}{3}(\mu_1\bar{I_1} + 2\mu_2\bar{I_2}) \overline{\bold{C}}^{-1}  + (\mu_1 + \mu_2\bar{I_1})\bold{I} - \mu_2\overline{\bold{C}} \right]
\end{split}
\end{equation}

In the mixed formulation, the volumetric part and isochoric part of the PK2 stress are Equation \ref{Svol11} and \ref{Siso1}, respectively since $p$ is independent of $\bold{u}$. While in the displacement-based formulation, the two parts are Equation \ref{Svol12} and Equation \ref{Siso1}, respectively.

Once the stress is evaluated, the next step is to evaluate the elasticity tensor $\mathbb{C}$ as defined in Equation \ref{C}. Recall that the volumetric elasticity tensor in Equation \ref{Cvol} is a function of $\tilde{p}$ where $\tilde{p}$ is:

\begin{equation}
\begin{split}
\tilde{p} &= p + J\frac{dp}{dJ} \\
             &= \kappa{(2J - 1)}
\end{split}
\end{equation}
Substituting $\tilde{p}$ into Equation \ref{Cvol} and simply evaluating $p$ from Equation \ref{pressure}, $\mathbb{C}_{vol}$ becomes:

\begin{subequations} \label{Cvol1}
\begin{align}
\mathbb{C}_\mathrm{vol} &= J^{-1/3}\tilde{p}{\overline{\bold{C}}}^{-1}\otimes{\overline{\bold{C}}}^{-1} - 2J^{-1/3}p{\overline{\bold{C}}}^{-1}\odot{\overline{\bold{C}}}^{-1} \label{Cvol11} \\
&=  J^{-1/3}(2J-1)\kappa \overline{\bold{C}}^{-1} \otimes \overline{\bold{C}}^{-1} - 2J^{-1/3}(J-1)\kappa \overline{\bold{C}}^{-1} \odot \overline{\bold{C}}^{-1} \label{Cvol12}
\end{align}
\end{subequations} 
Equation \ref{Cvol11} is for the mixed formulation, with $\tilde{p} = p$ because $dp/dJ = 0$, while Equation \ref{Cvol12} is for the displacement-based formulation.

Next we evaluate $\mathbb{C}_\mathrm{iso}$ following Equation \ref{Ciso}. The first term on the right-hand side in Equation \ref{Ciso} requires the evaluation of the fourth-order fictitious elasticity tensor $\mathbb{C}$, which starts with ${\partial{\Psi_\mathrm{iso}}}/{\partial{\overline{\bold{C}}}}$:

\begin{equation}
\begin{split}
\frac{\partial{\Psi_\mathrm{iso}(\Cb)}}{\partial{\overline{\bold{C}}}} &= \frac{\mu_1}{2}\frac{\partial{\bar{I_1}}}{\partial{\overline{\bold{C}}}} +  \frac{\mu_2}{2}\frac{\partial{\bar{I_2}}}{\partial{\overline{\bold{C}}}} \\
&= \frac{\mu_1}{2}\bold{I} + \frac{\mu_2}{2}(\bar{I_1}\bold{I} - \overline{\bold{C}})
\end{split}
\end{equation}

\begin{equation}
\frac{\partial^2\Psi_\mathrm{iso}(\overline{\bold{C}})}{\partial\overline{\bold{C}}\partial\overline{\bold{C}}} = 
\frac{\mu_2}{2}\left(\bold{I} \otimes \frac{\partial{\bar{I_1}}}{\partial{\overline{\bold{C}}}} - \mathbb{I}\right) = \frac{\mu_2}{2}(\bold{I} \otimes \bold{I} - \mathbb{I})
\end{equation}
Therefore the fourth-order fictitious elasticity tensor becomes:
\begin{equation}
\overline{\mathbb{C}} = 4J^{-4/3}\frac{\partial^2\Psi_\mathrm{iso}(\overline{\bold{C}})}{\partial{\overline{\bold{C}}}{\partial{\overline{\bold{C}}}}} = 2J^{-4/3}\mu_2(\bold{I} \otimes \bold{I} - \mathbb{I})
\end{equation}
with the projection tensor $\mathbb{P}$, the first term of the right-hand side in Equation \ref{Ciso} is:

\begin{equation} \label{part11}
\begin{split}
\mathbb{P} : {\overline{\mathbb{C}}} : {\mathbb{P}}^T 
= {} & 2J^{-4/3}\mu_2 \left(\mathbb{I} - \frac{1}{3}{\overline{\bold{C}}}^{-1} \otimes \overline{\bold{C}}\right)
: (\bold{I} \otimes \bold{I} - \mathbb{I}) : \left(\mathbb{I} - \frac{1}{3}\overline{\bold{C}} \otimes {\overline{\bold{C}}}^{-1}\right) \\
= {} & 2J^{-4/3}\mu_2 \bigg[\mathbb{I} : (\bold{I} \otimes \bold{I}) : \mathbb{I} - \frac{1}{3}\mathbb{I} : (\bold{I} \otimes \bold{I}) : (\overline{{\bold{C}}} \otimes {\overline{\bold{C}}}^{-1}) - \mathbb{I} : \mathbb{I} : \mathbb{I} \\
&+ \mathbb{I} : \mathbb{I} : \frac{1}{3}(\overline{\bold{C}} \otimes {\overline{\bold{C}}}^{-1}) - 
\frac{1}{3}({\overline{\bold{C}}}^{-1} \otimes \overline{\bold{C}} ) : (\bold{I} \otimes \bold{I}) : \mathbb{I}
- \frac{1}{3}({\overline{\bold{C}}}^{-1} \otimes \overline{\bold{C}} ) :  \mathbb{I} : \frac{1}{3}({\overline{\bold{C}}} \otimes {\overline{\bold{C}} }^{-1}) \\
&+ \frac{1}{3}(\overline{{\bold{C}}}^{-1} \otimes \overline{\bold{C}} ) : (\bold{I} \otimes \bold{I}) : \frac{1}{3}({\overline{\bold{C}}} \otimes {\overline{\bold{C}}}^{-1} ) + \frac{1}{3}({\overline{\bold{C}}} \otimes {\overline{\bold{C}}}^{-1}) : \mathbb{I} : \mathbb{I}
\bigg] \\
= {} & 2J^{-4/3}\mu_2 \bigg[\bold{I} \otimes \bold{I} - \frac{1}{3}\bar{I}_1(\bold{I} \otimes {\overline{\bold{C}}}^{-1}) - \mathbb{I} + \frac{1}{3}(\overline{\bold{C}} \otimes {\overline{\bold{C}}}^{-1}) -  \frac{1}{3}\bar{I}_1({\overline{\bold{C}}}^{-1} \otimes \bold{I})\\
&- \frac{1}{9}(\overline{\bold{C}} : \overline{\bold{C}})({\overline{\bold{C}}}^{-1} \otimes {\overline{\bold{C}}}^{-1})+ \frac{1}{9}{\bar{I}_1}^2({\overline{\bold{C}}}^{-1} \otimes {\overline{\bold{C}}})^{-1} +  \frac{1}{3}({\overline{\bold{C}}}^{-1} \otimes {\overline{\bold{C}}}) \bigg] \\
= {} & 2{J}^{-4/3}\mu_2 \bigg[ \bold{I} \otimes \bold{I} - \mathbb{I} - \frac{1}{3}\bar{I}_1({\overline{\bold{C}}}^{-1} \otimes \bold{I} + \bold{I}\otimes{\overline{\bold{C}}}^{-1} ) +
\frac{1}{3}({\overline{\bold{C}}}^{-1} \otimes \overline{\bold{C}} + \overline{\bold{C}}\otimes{\overline{\bold{C}}}^{-1} )  + \frac{2}{9}\bar{I}_2 ({\overline{\bold{C}}}^{-1} \otimes {\overline{\bold{C}}}^{-1}) \bigg] \\
= {} & 2{J}^{-4/3}\mu_2(\bold{I} \otimes \bold{I} - \mathbb{I}) - \frac{2}{3}J^{-4/3}\mu_2\bar{I}_1({\overline{\bold{C}}}^{-1} \otimes \bold{I} + \bold{I} \otimes {\overline{\bold{C}}}^{-1}) \\
&+
\frac{2}{3}J^{-4/3}\mu_2({\overline{\bold{C}}}^{-1} \otimes {\overline{\bold{C}}} + {\overline{\bold{C}}} \otimes {\overline{\bold{C}}}^{-1}) + \frac{4}{9}J^{-4/3}\mu_2\bar{I_2}({\overline{\bold{C}}}^{-1} \otimes {\overline{\bold{C}}}^{-1})
\end{split}
\end{equation}
The second term of the right-hand side of Equation \ref{Ciso} requires the trace of $J^{-2/3}\overline{\bold{S}}$:

\begin{equation}
\begin{split}
\mathrm{Tr}(J^{-2/3}\overline{\bold{S}}) &= J^{-2/3}{\overline{\bold{S}}} : {\bold{C}} \\
&= \Sb : \Cb \\
&= [(\mu_1 + \mu_2{\bar{I_1}})\bold{I} - \mu_2{\overline{\bold{C}}}] : \overline{\bold{C}} \\
&= (\mu_1 + \mu_2\bar{I_1})\bar{I_1} - \mu_2{\overline{\bold{C}}} : {\overline{\bold{C}}} \\
&= (\mu_1 + \mu_2\bar{I_1})\bar{I_1} - \mu_2({\bar{I_1}}^2 - 2\bar{I_2}) \\
&= \mu_1\bar{I_1} + 2\mu_2\bar{I_2}
\end{split}
\end{equation}
Along with the modified projection tensor $\tilde{\mathbb{P}}$, the second term of the right-hand side of Equation \ref{Ciso} becomes:

\begin{equation} \label{part21}
\begin{split}
\frac{2}{3}\mathrm{Tr}(J^{-2/3}\overline{\bold{S}})\tilde{\mathbb{P}}
&= \frac{2}{3}J^{-4/3}(\mu_1\bar{I_1} + 2\mu_2\bar{I_2})\left({\overline{\bold{C}}}^{-1} \odot {{\overline{\bold{C}}}}^{-1} - \frac{1}{3}{{\overline{\bold{C}}}}^{-1} \otimes {\overline{{\bold{C}}}}^{-1}\right)
\end{split}
\end{equation}
Finally, with Equation \ref{Siso1} for $\bold{S}_\mathrm{iso}$, the third term of the right-hand side of Equation \ref{Ciso} becomes:
\begin{equation} \label{part31}
\begin{split}
- \frac{2}{3}(\bold{C}^{-1}\otimes\bold{S}_\mathrm{iso} + \bold{S}_\mathrm{iso}\otimes \bold{C}^{-1})
= {} &
- \frac{2}{3}J^{-4/3} \bigg\{ {\overline{\bold{C}}}^{-1} \otimes \left[-\frac{1}{3}(\mu_1\bar{I_1} + 2\mu_2\bar{I_2}){\overline{\bold{C}}}^{-1} + (\mu_1 + \mu_2\bar{I_1})\bold{I} - \mu_2{\overline{\bold{C}}} \right] \\
&+
\left[-\frac{1}{3}(\mu_1\bar{I_1} + 2\mu_2\bar{I_2}){\overline{\bold{C}}}^{-1} + (\mu_1 + \mu_2\bar{I_1})\bold{I} - \mu_2{\overline{\bold{C}}} \right] \otimes {\overline{\bold{C}}}^{-1} \bigg\} \\
= {} &
- \frac{2}{3}J^{-4/3} \bigg[ -\frac{1}{3}(\mu_1\bar{I_1} + 2\mu_2\bar{I_2}) ({\overline{\bold{C}}}^{-1} \otimes {\overline{\bold{C}}}^{-1}) + (\mu_1 + \mu_2\bar{I_1})({\overline{\bold{C}}}^{-1}\otimes\bold{I}) \\ 
& - \mu_2({\overline{\bold{C}}}^{-1} \otimes {\overline{\bold{C}}}) 
- \frac{1}{3}(\mu_1\bar{I_1} + 2\mu_2\bar{I_2}) ({\overline{\bold{C}}}^{-1} \otimes {\overline{\bold{C}}}^{-1}) \\
&+ (\mu_1 + \mu_2\bar{I_1}) (\bold{I} \otimes {\overline{\bold{C}}}^{-1}) - \mu_2 ({\overline{\bold{C}}} \otimes {\overline{\bold{C}}}^{-1}) \bigg] \\
= {} &
 - \frac{2}{3}J^{-4/3} \bigg[ -\frac{2}{3}(\mu_1\bar{I_1} + 2\mu_2\bar{I_2})({\overline{\bold{C}}}^{-1} \otimes {\overline{\bold{C}}}^{-1}) + (\mu_1+\mu_2\bar{I_1})({\overline{\bold{C}}}^{-1} \otimes \bold{I} + \bold{I} \otimes {\overline{\bold{C}}}^{-1})\\
&- \mu_2({\overline{\bold{C}}}^{-1} \otimes {\overline{\bold{C}}}+{\overline{\bold{C}}} \otimes {\overline{\bold{C}}}^{-1}) \bigg]
\end{split}
\end{equation}
Combining Equations \ref{part11}, \ref{part21} and \ref{part31}, $\mathbb{C}_\mathrm{iso}$ is:
\begin{equation} \label{Ciso1}
\begin{split}
\mathbb{C}_\mathrm{iso} 
= {} & 
2{J}^{-4/3}\mu_2(\bold{I} \otimes \bold{I} - \mathbb{I}) - \frac{2}{3}J^{-4/3}\mu_2\bar{I_1}({\overline{\bold{C}}}^{-1} \otimes \bold{I} + \bold{I} \otimes {\overline{\bold{C}}}^{-1}) \\
&+
\frac{2}{3}J^{-4/3}\mu_2({\overline{\bold{C}}}^{-1} \otimes {\overline{\bold{C}}} + {\overline{\bold{C}}} \otimes {\overline{\bold{C}}}^{-1}) + \frac{4}{9}J^{-4/3}\mu_2\bar{I_2}({\overline{\bold{C}}}^{-1} \otimes {\overline{\bold{C}}}^{-1}) \\
&+
\frac{2}{3}J^{-4/3}(\mu_1\bar{I_1} + 2\mu_2\bar{I_2}) \left ({\overline{\bold{C}}}^{-1} \odot {{\overline{\bold{C}}}}^{-1} - \frac{1}{3}{{\overline{\bold{C}}}}^{-1} \otimes {\overline{{\bold{C}}}}^{-1} \right) \\
&+
\frac{4}{9}J^{-4/3} (\mu_1\bar{I_1} + 2\mu_2\bar{I_2})({\overline{\bold{C}}}^{-1} \otimes {\overline{\bold{C}}}^{-1}) - \frac{2}{3}J^{-4/3}(\mu_1+\mu_2\bar{I_1})({\overline{\bold{C}}}^{-1} \otimes \bold{I} + \bold{I} \otimes {\overline{\bold{C}}}^{-1})\\
&+ \frac{2}{3}J^{-4/3}\mu_2({\overline{\bold{C}}}^{-1} \otimes {\overline{\bold{C}}}+{\overline{\bold{C}}} \otimes {\overline{\bold{C}}}^{-1})  \\
= {} &
2{J}^{-4/3}\mu_2(\bold{I} \otimes \bold{I} - \mathbb{I}) + \frac{2}{3}J^{-4/3}(\mu_1\bar{I_1} + 2\mu_2\bar{I_2})\left({\overline{\bold{C}}}^{-1} \odot {\overline{\bold{C}}}^{-1} - \frac{1}{3}{\overline{\bold{C}}}^{-1} \otimes {\overline{\bold{C}}}^{-1}\right) \\
&-
\frac{2}{3}J^{-4/3}(\mu_1 + 2\mu_2\bar{I_1})({\overline{\bold{C}}}^{-1} \otimes \bold{I} + \bold{I} \otimes {\overline{\bold{C}}}^{-1}) + \frac{4}{3}J^{-4/3}\mu_2({\overline{\bold{C}}}^{-1} \otimes {\overline{\bold{C}}} + {\overline{\bold{C}}} \otimes {\overline{\bold{C}}}^{-1}) \\
&+
\frac{4}{9}J^{-4/3}(\mu_1\bar{I_1} + 3\mu_2\bar{I_2}){\overline{\bold{C}}}^{-1} \otimes {\overline{\bold{C}}}^{-1} \\
= {} &
2{J}^{-4/3}\mu_2(\bold{I} \otimes \bold{I} - \mathbb{I}) - \frac{2}{3}J^{-4/3}(\mu_1 + 2\mu_2\bar{I_1})({\overline{\bold{C}}}^{-1} \otimes \bold{I} + \bold{I} \otimes {\overline{\bold{C}}}^{-1}) + \frac{4}{3}J^{-4/3}\mu_2({\overline{\bold{C}}}^{-1} \otimes {\overline{\bold{C}}} + {\overline{\bold{C}}} \otimes {\overline{\bold{C}}}^{-1}) \\
&+
\frac{2}{9}J^{-4/3}(\mu_1\bar{I_1} + 4\mu_2\bar{I_2}) ({\overline{\bold{C}}}^{-1} \otimes {\overline{\bold{C}}}^{-1}) + \frac{2}{3}J^{-4/3}(\mu_1\bar{I_1} + 2\mu_2\bar{I_2})({\overline{\bold{C}}}^{-1} \odot {\overline{\bold{C}}}^{-1}) 
\end{split}
\end{equation} 
Equations \ref{Cvol1} and \ref{Ciso1} complete the derivations of elasticity tensor for the volumetric and isochoric parts, respectively, in the reference configuration.

%
\subsubsection{Yeoh Model}
First introduced in 1990 \cite{Yeoh} and modified in 1993 \cite{Yeoh2}, Yeoh model was motivated to simulate the mechanical behavior of carbon-black filled rubber vulcanizates with the typical stiffening effect in large strain domain. Yeoh model is a polynomial of $\bar{I}_1$ only.  To capture the stiffening effect of rubber in the large strain domain, it is usually truncated up to the third-order. Yeoh model is another popular choice in biomechanics which has been be applied to human breast tissue \cite{OHagen}, porcine muscular tissue \cite{Bols2}, rat lung parenchyma \cite{Wall, Wall2} etc. It is even proven to be the best one in capturing nonlinear behaviors of some specific tissues compared to Neo-Hookean and Mooney-Rivlin models \cite{Zaeimdar}.

The volumetric part to account for compressibility in Yeoh model is the same as in Mooney-Rivlin model. Consequently the volumetric parts of both stress and elasticity tensors are the same as Equations \ref{Svol1} and \ref{Cvol1}. Therefore, we will only show the derivation of the isochoric component.

The isochoric part in Yeoh model includes higher order terms. It is written as

\begin{equation} \label{iso2}
\Psi_\mathrm{iso} = c_1(\bar{I_1} - 3) + c_2(\bar{I_1} - 3)^2 + c_3(\bar{I_1} - 3)^3
\end{equation}
where $c_1$, $c_2$, $c_3$ are material constants with constraints $c_1 > 0$, $c2 < 0$, and $c_3 > 0$. The initial shear modulus is approximated as $\mu = 2c_1$.

Following the same way, we first derive the fictitious stress:
\begin{equation} \label{Sbar2}
\begin{split}
\overline{\bold{S}} &= 2\frac{\partial\Psi_\mathrm{iso}({\overline{\bold{C}})}}{\partial\overline{\bold{C}}} \\
&= 2\frac{\partial\Psi_\mathrm{iso}}{\partial\bar{I_1}} \frac{\partial \bar{I}_1}{\partial \Cb} \\
&= [2c_1 + 4c_2(\bar{I_1} - 3) + 6c_3(\bar{I_1} - 3)^2] \bold{I} \\
\end{split}
\end{equation}
Then the isochoric part of the PK2 stress is obtained as:
\begin{equation} \label{Siso2}
\begin{split}
{\bold{S}}_\mathrm{iso} 
&= J^{-2/3} \mathbb{P} : \Sb \\
&= J^{-2/3}\left(\mathbb{I} - \frac{1}{3}{\overline{\bold{C}}}^{-1} \otimes \overline{\bold{C}}\right) : \overline{\bold{S}}\\
&= J^{-2/3} \left[ \overline{\bold{S}} - \frac{1}{3} (\overline{\bold{C}} : \overline{\bold{S}}) \overline{\bold{C}}^{-1}  \right] \\
&= J^{-2/3} \left\{\overline{\bold{S}} - \frac{1}{3} [2c_1 + 4c_2(\bar{I_1} - 3) + 6c_3(\bar{I_1} - 3)^2] \bar{I_1}\overline{\bold{C}}^{-1} \right\} \\
&= J^{-2/3}[2c_1 + 4c_2(\bar{I_1} - 3) + 6c_3(\bar{I_1} - 3)^2] \left(\bold{I} - \frac{1}{3}\bar{I_1}{\overline{\bold{C}}}^{-1} \right)
\end{split}
\end{equation}
Similarly, to derive the first term on the right-hand side of Equation \ref{Ciso} we must begin with $\partial{\Psi_\mathrm{iso}}/\partial{\overline{\bold{C}}}$:

\begin{equation}
\begin{split}
\frac{\partial{\Psi_\mathrm{iso}(\Cb)}}{\partial{\overline{\bold{C}}}} 
&= [c_1 + 2c_2(\bar{I_1} - 3) + 3c_3(\bar{I_1} - 3)^2] \frac{\partial{\bar{I}_1}}{\partial{\overline{\bold{C}}}} \\
&= [c_1 + 2c_2(\bar{I_1} - 3) + 3c_3(\bar{I_1} - 3)^2] \bold{I} 
\end{split}
\end{equation}
The second order derivative is:
\begin{equation}
\begin{split}
\frac{\partial^2\Psi_\mathrm{iso}(\overline{\bold{C}})}{\partial\overline{\bold{C}}\partial\overline{\bold{C}}} &= 
\bold{I} \otimes \frac{\partial[c_1 + 2c_2({\bar{I}}_1-3) + 3c_3({\bar{I}}_1-3)^2 ]}{\partial{\overline{\bold{C}}}} \\
&= \bold{I} \otimes [2c_2\bold{I} + 6c_3({\bar{I}}_1-3)\bold{I}]\\
&= [2c_2 + 6c_3({\bar{I}}_1-3)] \bold{I} \otimes \bold{I}
\end{split}
\end{equation}
Therefore the fictitious elasticity tensor is
\begin{equation}
\overline{\mathbb{C}} = 4J^{-4/3}\frac{\partial^2\Psi_\mathrm{iso}(\overline{\bold{C}})}{\partial{\overline{\bold{C}}}{\partial{\overline{\bold{C}}}}} = 8J^{-4/3}[c_2 + 3c_3({\bar{I}}_1 - 3) ]\bold{I} \otimes \bold{I}
\end{equation}
The first term on the right-hand side of Equation \ref{Ciso} can be found as

\begin{equation} \label{part12}
\begin{split}
\mathbb{P} : \overline{\mathbb{C}} : {\mathbb{P}}^T 
= {}& 8J^{-4/3}[c_2 + 3c_3({\bar{I}}_1 - 3 )] \left( \mathbb{I} - \frac{1}{3}{\overline{\bold{C}}}^{-1} \otimes {\overline{\bold{C}}} \right) : (\bold{I} \otimes \bold{I}): \left(\mathbb{I} - \frac{1}{3}{\overline{\bold{C}}} \otimes {\overline{\bold{C}}}^{-1}\right) \\
= {}& 8J^{-4/3}[c_2 + 3c_3({\bar{I}}_1 - 3 )] \bigg[\mathbb{I}:(\bold{I} \otimes \bold{I}):\mathbb{I} - \mathbb{I}:(\bold{I} \otimes \bold{I}): \frac{1}{3} ({\overline{\bold{C}}} \otimes {\overline{\bold{C}}}^{-1}) \\
&- \frac{1}{3} ({\overline{\bold{C}}}^{-1} \otimes {\overline{\bold{C}}}) : (\bold{I} \otimes \bold{I}) : \mathbb{I} +  \frac{1}{9} ({\overline{\bold{C}}}^{-1} \otimes {\overline{\bold{C}}}) : (\bold{I} \otimes \bold{I}) : ({\overline{\bold{C}}} \otimes {\overline{\bold{C}}}^{-1}) \bigg] \\
= {}& 8J^{-4/3}[c_2 + 3c_3({\bar{I}}_1 - 3 )] \left[ \bold{I} \otimes \bold{I} - \frac{1}{3}\bar{I}_1({\overline{\bold{C}}}^{-1} \otimes \bold{I}) - \frac{1}{3}\bar{I}_1 (\bold{I} \otimes {\overline{\bold{C}}}^{-1}) + \frac{1}{9}{\bar{I}_1}^2({\overline{\bold{C}}}^{-1} \otimes {\overline{\bold{C}}}^{-1}) \right] \\
= {}& 8J^{-4/3} [c_2 + 3c_3({\bar{I}}_1 - 3 )] \left[ \bold{I} \otimes \bold{I} - \frac{{\bar{I}}_1}{3}(\bold{I} \otimes {\overline{\bold{C}}}^{-1} +{\overline{\bold{C}}}^{-1}  \otimes  \bold{I}) + \frac{{{\bar{I}}_1}^2}{9} ({\overline{\bold{C}}}^{-1} \otimes {\overline{\bold{C}}}^{-1}) \right] \\
= {}&  J^{-4/3}[8c_2 + 24c_3({\bar{I}}_1 - 3 )] (\bold{I} \otimes \bold{I} ) -  J^{-4/3} \left[ \frac{8}{3}c_2{\bar{I}}_1 + 8c_3({\bar{I}}_1 - 3 ){\bar{I}}_1 \right] (\bold{I} \otimes {\overline{\bold{C}}}^{-1} +{\overline{\bold{C}}}^{-1}  \otimes  \bold{I}) \\
&+
J^{-4/3} \left[ \frac{8}{9}c_2{\bar{I}_1}^2 + \frac{8}{3}c_3({\bar{I}}_1 - 3 ){\bar{I}_1}^2 \right] ({\overline{\bold{C}}}^{-1} \otimes {\overline{\bold{C}}}^{-1})
\end{split}
\end{equation}
Using the fictitious PK2 stress derived in Equation \ref{Sbar2}, we can obtain the trace of $J^{-2/3}\overline{\bold{S}}$

\begin{equation}
\begin{split}
\mathrm{Tr}(J^{-2/3}\overline{\bold{S}}) &= J^{-2/3}\overline{\bold{S}}:\bold{C} \\
&= \overline{\bold{S}} : \overline{\bold{C}} \\
&=  [2c_1 + 4c_2(\bar{I}_1 - 3) + 6c_3(\bar{I}_1 - 3)^2] (\bold{I} : \overline{\bold{C}}) \\
&= 2c_1\bar{I}_1 + 4c_2(\bar{I_1} - 3)\bar{I}_1 + 6c_3(\bar{I}_1 - 3)^2\bar{I}_1
\end{split}
\end{equation}
So that the second term on the right-hand side of Equation \ref{Ciso} is

\begin{equation} \label{part22}
\begin{split}
\frac{2}{3}\mathrm{Tr}(J^{-2/3}\overline{\bold{S}}) \tilde{\mathbb{P}}
= {} & 
\frac{2}{3}J^{-4/3} [2c_1\bar{I}_1 + 4c_2(\bar{I_1} - 3)\bar{I}_1 + 6c_3(\bar{I_1} - 3)^2\bar{I}_1] \left({\overline{\bold{C}}}^{-1} \odot {\overline{\bold{C}}}^{-1} - \frac{1}{3}{\overline{\bold{C}}}^{-1} \otimes {\overline{\bold{C}}}^{-1} \right) \\
= {} &
J^{-4/3} \left[\frac{4}{3}c_1\bar{I}_1 + \frac{8}{3}c_2(\bar{I_1} - 3)\bar{I}_1 + 4c_3(\bar{I_1} - 3)^2\bar{I}_1\right] \left({\overline{\bold{C}}}^{-1} \odot {\overline{\bold{C}}}^{-1} - \frac{1}{3}{\overline{\bold{C}}}^{-1} \otimes {\overline{\bold{C}}}^{-1} \right) 
\end{split}
\end{equation}
The third term on the right-hand-side of Equation \ref{Ciso} is obtained by using $\bold{S}_\mathrm{iso}$ in Equation \ref{Siso2}

\begin{equation} \label{part32}
\begin{split}
&-\frac{2}{3}J^{-4/3}({\overline{\bold{C}}}^{-1} \otimes {\overline{\bold{S}}}_\mathrm{iso} + {\overline{\bold{S}}}_\mathrm{iso} \otimes {\overline{{\bold{C}}}^{-1}})\\
= {} &
-\frac{2}{3}J^{-4/3}[2c_1 + 4c_2({\bar{I}}_1 - 3) + 6c_3({\bar{I}}_1 - 3)^2]
\left[{\overline{\bold{C}}}^{-1} \otimes \left(\bold{I} - \frac{1}{3}{\bar{I}_1}{\overline{\bold{C}}}^{-1}\right) + 
\left(\bold{I} - \frac{1}{3}{\bar{I}_1}{\overline{\bold{C}}}^{-1}\right) \otimes {\overline{\bold{C}}}^{-1}\right]\\
= {} &
J^{-4/3}  \left[-\frac{4}{3}c_1 - \frac{8}{3}c_2({\bar{I}}_1 - 3) - 4c_3({\bar{I}}_1 - 3)^2\right] ({\overline{\bold{C}}}^{-1} \otimes \bold{I} + \bold{I} \otimes {\overline{\bold{C}}}^{-1}) \\
&+ J^{-4/3}  \left[\frac{8}{9}c_1{\bar{I}}_1 + \frac{16}{9}c_2({\bar{I}}_1 - 3){\bar{I}}_1 + \frac{8}{3}c_3({\bar{I}}_1 - 3)^2{\bar{I}}_1 \right] ({\overline{\bold{C}}}^{-1} \otimes {\overline{\bold{C}}}^{-1})
\end{split}
\end{equation}
Combining Equations \ref{part12}, \ref{part22} and \ref{part32}, we have the isochoric part of the elasticity tensor:
\begin{equation} \label{Ciso2}
\begin{split}
{\mathbb{C}}_\mathrm{iso} = {} & J^{-4/3} \bigg\{
[8c_2+24c_3({\bar{I}}_1 - 3)] (\bold{I} \otimes \bold{I}) \\
&- \left[\frac{4}{3}c_1 + c_2\left(\frac{16}{3}{\bar{I}}_1 - 8\right) + 12c_3({\bar{I}}_1 - 1)({\bar{I}}_1 - 3)\right]({\overline{\bold{C}}}^{-1} \otimes \bold{I} + \bold{I} \otimes {\overline{\bold{C}}}^{-1})\\
&+ \left[\frac{4}{9}c_1{\bar{I}}_1 + c_2\left(\frac{16}{9}{{\bar{I}}_1}^2 - \frac{8}{3}{\bar{I}}_1\right) + 4c_3{\bar{I}}_1({\bar{I}}_1 - 1)({\bar{I}}_1 - 3)\right] ({\overline{\bold{C}}}^{-1} \otimes {\overline{\bold{C}}}^{-1}) \\
&+ \left[\frac{4}{3}c_1{\bar{I}}_1 + \frac{8}{3}c_2({\bar{I}}_1 - 3){\bar{I}}_1 + 4c_3({\bar{I}}_1 - 3)^2{\bar{I}}_1\right]({\overline{\bold{C}}}^{-1} \odot {\overline{\bold{C}}}^{-1})
\bigg\}
\end{split}
\end{equation}
Equation \ref{Cvol1} and Equation \ref{Ciso2} yield the elasticity tensor in the reference configuration for Yeoh model.

%
\subsubsection{Holzapfel-Gasser-Ogden Model}
The Holzapfel-Gasser-Ogden (HGO) model was introduced in 2000 \cite{Holzapfel2} to model the layering structure of arterial tissues. The idea was to formulate a constitutive model which incorporates some histological structure of arterial walls (i.e. fiber direction). The volumetric part is the same as Mooney-Rivlin and Yeoh models. What distinguishes HGO model from the phenomenological models is that it accounts for both the non-collagenous matrix material which is active at low pressures (modeled as isotropic) and the collagenous fibers which becomes active at high pressure (modeled as anisotropic). HGO model uses Neo-Hookean model as the isotropic ground material, and the anisotropic part consists two families of fibers represented by two pseudo-invariants $\bar{I}_4$ and $\bar{I}_6$. The directions of these fibers are represented by unit vectors $\bold{a}_{04}$ and $\bold{a}_{06}$ in the reference configuration. Specifically, the isochoric part of HGO model is consist of isotropic part $\Psi_\mathrm{isotropic}$ and anisotropic part $\Psi_\mathrm{aniso}$:

\begin{subequations} \label{HGO}
\begin{align}
\Psi_\mathrm{iso}(\bold{C}, \bold{a}_{04}, \bold{a}_{06}) &= \Psi_\mathrm{isotropic}(\overline{\bold{C}}) + \Psi_\mathrm{aniso}(\overline{\bold{C}}, \bold{a}_{04}, \bold{a}_{06}) 
\end{align}
In particular
\begin{align}
\Psi_\mathrm{isotropic} &= \frac{\mu_1}{2}(\bar{I}_1 - 3)  \label{isotropic} \\
\Psi_\mathrm{aniso}(\overline{\bold{C}}, \bold{a}_{04}, \bold{a}_{06}) &= \frac{k_1}{2k_2} \sum_{i = 4, 6} \{\mathrm{exp}[k_2(\bar{I_i} - 1)^2] - 1\} \label{anisotropic} 
\end{align}
where $k_1 > 0$ is a stress-like material parameter and $k_2 > 0$ is a dimensionless parameter.
\end{subequations}

The volumetric part of the stress and elasticity tensors of the HGO model are the same as Mooney-Rivlin and Yeoh models derived in Equations \ref{Svol1} and \ref{Cvol1}. The isotropic part of the isochoric HGO model is simply the Neo-Hookean model, which is a special case of Mooney-Rivlin model with $\mu_2 = 0$. Therefore, to obtain the isotropic contribution to the isochoric PK2 stress, we only need to set $\mu_2$ to $0$ in Equation \ref{Siso1}:
\begin{equation} \label{Sisotropic}
\bold{S}_\mathrm{isotropic} = J^{-2/3}\mu_1\left(-\frac{1}{3}\bar{I}_1\overline{\bold{C}}^{-1} + \bold{I}\right)
\end{equation}

Similarly, by setting $\mu_2$ to $0$ in the isochoric elasticity tensor from Mooney-Rivlin model in Equation \ref{Ciso1}, the isotropic part of the isochoric elasticity tensor for the HGO model becomes:

\begin{equation} \label{Ciso32}
\begin{split}
\mathbb{C}_\mathrm{isotropic} = {}&
- \frac{2}{3}J^{-4/3}\mu_1({\overline{\bold{C}}}^{-1} \otimes \bold{I} + \bold{I} \otimes {\overline{\bold{C}}}^{-1}) 
+
\frac{2}{9}J^{-4/3}\mu_1\bar{I_1}  {\overline{\bold{C}}}^{-1} \otimes {\overline{\bold{C}}}^{-1} \\
&+ \frac{2}{3}J^{-4/3} \mu_1\bar{I_1} {\overline{\bold{C}}}^{-1} \odot {\overline{\bold{C}}}^{-1} 
\end{split}
\end{equation}

To derive the anisotropic parts of stress and elasticity tensors, the deformation of the fibers needs to be introduced. As the material deforms, the vectors of the fibers deform accordingly and are expressed as unit vectors $\bold{a}_4$ and $\bold{a}_6$ in the current configuration. $\bold{a}_{0i}$ and $\bold{a}_{i}$ ($i = 4, 6$) are related by 
\begin{equation} \label{vector}
\lambda_i\bold{a}_i = \bold{F}\bold{a}_{0i}
\end{equation} 
where $i = 4, 6$, $\lambda_i$ is the stretch of the original fiber and no summation is applied on $i$.
Since $|\bold{a}_i| = 1$, we find the value of stretch $\lambda_i$ through:
\begin{equation}
{\lambda_i}^2 = \bold{a}_{0i} \cdot \bold{F}^T\bold{F}\bold{a}_{0i} = \bold{a}_{0i} \cdot \bold{C}\bold{a}_{0i}, \quad i = 4, 6
\end{equation}
\\
To express the anisotropic term, two pseudo-invariants are defined:
\begin{equation} \label{pseudo-invariants}
I_i(\bold{C}, \bold{a}_{0i}) = \bold{a}_{0i} \cdot \bold{C}\bold{a}_{0i} = {\lambda_i}^2, \quad i = 4, 6
\end{equation}
Similarly, the modified invariants are defined as:
\begin{equation}
\bar{I_i} = J^{-2/3}I_i, \quad i = 4, 6
\end{equation}
For the convenience of the derivation, we define two second order tensors to represent the dyadic products of $\bold{a}_{0i}$:
\begin{equation} \label{A0i}
\bold{A}_{0i} = \bold{a}_{0i} \otimes \bold{a}_{0i}, \quad i = 4, 6
\end{equation}
The derivatives of the pseudo-invariants are easy to find:
\begin{equation} \label{helper1}
\frac{\partial\bar I_i}{\partial \overline{\bold{C}}} = \bold{A}_{0i}, \quad i = 4, 6
\end{equation}


Based on the definition of these pseudo-invariants and the symmetric structure of the HGO model itself, it is obvious to see that $\bar{I_4}$ and $\bar{I_6}$ should be symmetric in all the related expressions. 

Next, we derive the stress and elasticity tensors for the anisotropic part of the isochoric HGO model in a similar approach as presented in Section \ref{general_stress}. The anisotropic part will be combined with the isochoric isotropic part and the volumetric part to form the complete stress and elasticity tensors of the HGO model.

As the first step, Equation \ref{helper1} is used to obtain the anisotropic part of the fictitious PK2 stress:

\begin{equation} \label{Sbar3}
\begin{split}
\overline{\bold{S}}_\mathrm{aniso} &=  2\frac{\partial\Psi_\mathrm{aniso}({\overline{\bold{C}})}}{\partial\overline{\bold{C}}} \\
&= 2\sum_{i = 4, 6}\frac{\partial{\Psi_\mathrm{aniso}}}{\partial{\bar{I}_i}}\bold{A}_{0i} 
\end{split}
\end{equation}
Recall the definitions in Equations \ref{vector} to \ref{A0i}, we can prove that

\begin{equation}
\overline{\bold{C}} : \bold{A}_{0i} = \overline{\bold{C}} : (\bold{a}_{0i} \otimes \bold{a}_{0i}) = \bold{a}_{0i} \cdot (\overline{\bold{C}} \bold{a}_{0i}) = \bar{I}_i, \quad i = 4, 6
\end{equation}
Consequently, 

\begin{equation} \label{helper2}
\overline{\bold{C}} : \overline{\bold{S}}_\mathrm{aniso} = 2\sum_{i = 4, 6} \bar{I}_i \frac{\partial{\Psi_\mathrm{aniso}}}{\partial{\bar{I}_i}}  
\end{equation}
Using Equation \ref{helper2}, the isochoric anisotropic PK2 stress is obtained through Equation \ref{Siso}

\begin{equation} \label{Sanisotropic}
\begin{split}
\bold{S}_\mathrm{aniso}
&= J^{-2/3}(\mathbb{I} - \frac{1}{3}{\overline{\bold{C}}}^{-1} \otimes \overline{\bold{C}}) : \overline{\bold{S}}_\mathrm{aniso} \\
&= J^{-2/3}[\overline{\bold{S}}_\mathrm{aniso} - \frac{1}{3}(\overline{\bold{C}} : \overline{\bold{S}}_\mathrm{aniso}){\overline{\bold{C}}}^{-1}] \\
&= 2J^{-2/3} \sum_{i = 4, 6}\left[\frac{\partial{\Psi_\mathrm{aniso}}}{\partial{\bar{I}_i}}  \left(\bold{A}_{0i} - \frac{1}{3}\bar{I}_i\overline{\bold{C}}^{-1}\right)\right]
\end{split}
\end{equation}
where
${\partial\Psi_\mathrm{aniso}}/{\partial{\bar{I}_i}} = k_1(\bar{I}_i - 1)e^{k_2(\bar{I}_i - 1)^2}$.
Combining Equations \ref{Sisotropic} and \ref{Sanisotropic} we have the isochoric PK2 stress for the HGO model:
\begin{equation} \label{HGOSiso}
\bold{S}_\mathrm{iso} 
= J^{-2/3}\mu_1 \left( -\frac{1}{3}\bar{I}_1\overline{\bold{C}}^{-1} + \bold{I} \right) 
+2J^{-2/3} \sum_{i = 4, 6} \left[ \frac{\partial{\Psi_\mathrm{aniso}}}{\partial{\bar{I}_i}}  \left(\bold{A}_{0i} - \frac{1}{3}\bar{I}_i\overline{\bold{C}}^{-1} \right) \right]
\end{equation}

Next we derive the anisotropic part of the isochoric elasticity tensor of the HGO model. For convenience, we do not explicitly write out ${\partial\Psi_\mathrm{aniso}}/{\partial{\bar{I}_i}}$ and $\partial^2{\Psi_\mathrm{aniso}}/\partial{\bar{I}_i}^2$ where $i = 4, 6$. The first and second derivatives of $\Psi_\mathrm{aniso}$ with respect to $\overline{\bold{C}}$ can be easily converted to that with respect to $\bar{I}_{i}$ ($i = 4, 6$) as follows:

\begin{subequations} \label{convert}
\begin{align}
\frac{\partial\Psi_\mathrm{aniso}}{\partial\overline{\bold{C}}} 
&= \sum_{i = 4, 6}\frac{\partial\Psi_\mathrm{aniso}}{\partial\bar{I}_i}\bold{A}_{0i} \label{convert1} \\
\frac{\partial^2\Psi_\mathrm{aniso}}{\partial{\overline{\bold{C}}}\partial{\overline{\bold{C}}}} 
&= \sum_{i = 4, 6}\frac{\partial^2\Psi_\mathrm{aniso}}{\partial{\bar{I}_i}^2}\bold{A}_{0i} \otimes \bold{A}_{0i} \label{convert2}
\end{align}
\end{subequations}
Using Equation \ref{convert2}, the anisotropic contribution to the fictitious elasticity tensor $\overline{\mathbb{C}}_\mathrm{aniso}$ is:

\begin{equation}
\begin{split}
\overline{\mathbb{C}}_\mathrm{aniso} 
&= 4J^{-4/3}\frac{\partial^2\Psi_\mathrm{aniso}}{\partial\overline{\bold{C}}\partial\overline{\bold{C}}} \\
&= 4J^{-4/3}\sum_{i = 4, 6}\frac{\partial^2\Psi_\mathrm{aniso}}{\partial{\bar{I}_i}^2}\bold{A}_{0i} \otimes \bold{A}_{0i}
\end{split}
\end{equation}
Therefore, the first term on the right-hand side of Equation \ref{Ciso} becomes:

\begin{equation} \label{part13}
\begin{split}
\mathbb{P} : \overline{\mathbb{C}}_\mathrm{aniso} : \mathbb{P}^T 
&= \sum_{i = 4, 6}J^{-4/3}\frac{\partial^2\Psi_\mathrm{aniso}}{\partial{\bar{I}_i}^2} \left( \mathbb{I} - \frac{1}{3}\overline{\bold{C}}^{-1} \otimes \overline{\bold{C}} \right):(\bold{A}_{0i} \otimes \bold{A}_{0i}):\left( \mathbb{I} - \frac{1}{3}\overline{\bold{C}} \otimes \overline{\bold{C}}^{-1} \right) \\
&= \sum_{i = 4, 6} 4J^{-4/3}\frac{\partial^2\Psi_\mathrm{aniso}}{\partial{\bar{I}_i}^2}
\left[ \bold{A}_{0i} - \frac{1}{3}(\overline{\bold{C}} : \bold{A}_{0i})\overline{\bold{C}}^{-1} \right] \otimes
\left[ \bold{A}_{0i} - \frac{1}{3}(\overline{\bold{C}} : \bold{A}_{0i})\overline{\bold{C}}^{-1} \right] \\
&= \sum_{i = 4, 6} 4J^{-4/3}\frac{\partial^2\Psi_\mathrm{aniso}}{\partial{\bar{I}_i}^2}
\left[ \bold{A}_{0i} \otimes \bold{A}_{0i} - \frac{1}{3}\bar{I}_i
(\overline{\bold{C}}^{-1} \otimes \bold{A}_{0i} + \bold{A}_{0i} \otimes \overline{\bold{C}}^{-1}) 
+ \frac{1}{9}\bar{I}_i^2 \overline{\bold{C}}^{-1} \otimes \overline{\bold{C}}^{-1} \right]
\end{split}
\end{equation}
Recall Equation \ref{helper2}, the second term on the right-hand side of Equation \ref{Ciso} is:

\begin{equation} \label{part23}
\begin{split}
\frac{2}{3}\mathrm{Tr}(J^{-2/3}\overline{\bold{S}}_\mathrm{aniso})\tilde{\mathbb{P}} &= \frac{2}{3}(\overline{\bold{S}}_\mathrm{aniso} : \overline{\bold{C}}) \tilde{\mathbb{P}} \\
&= \frac{2}{3}J^{-4/3}(\overline{\bold{S}}_\mathrm{aniso} : \overline{\bold{C}}) 
\left( \overline{\bold{C}}^{-1} \odot \overline{\bold{C}}^{-1} - \frac{1}{3}\overline{\bold{C}}^{-1}\otimes\overline{\bold{C}}^{-1} \right) \\
&= \sum_{i = 4, 6}\frac{4}{3}J^{-4/3}\bar{I}_i\frac{\partial{\Psi_\mathrm{aniso}}}{\partial\bar{I}_i}
\left( \overline{\bold{C}}^{-1} \odot \overline{\bold{C}}^{-1} - \frac{1}{3}\overline{\bold{C}}^{-1}\otimes\overline{\bold{C}}^{-1} \right) \\
\end{split}
\end{equation}
Substituting Equation \ref{Sanisotropic} into the third term on the right-hand side of Equation \ref{Ciso}, it becomes:

\begin{equation} \label{part33}
\begin{split}
&-\frac{2}{3}J^{-2/3}({\overline{\bold{C}}}^{-1} \otimes \bold{S}_\mathrm{aniso} + \bold{S}_\mathrm{aniso} \otimes {\overline{\bold{C}}}^{-1}) \\
= {} &
\sum_{i = 4, 6} -\frac{4}{3}J^{-4/3}\frac{\partial\Psi_\mathrm{aniso}}{\partial\bar{I}_i}
\left[ \overline{\bold{C}}^{-1} \otimes \left( \bold{A}_{0i} - \frac{1}{3}\bar{I}_i \overline{\bold{C}}^{-1} \right)
+ \left( \bold{A}_{0i} - \frac{1}{3}\bar{I}_i \overline{\bold{C}}^{-1} \right) \otimes \overline{\bold{C}}^{-1} \right] \\
= {} & \sum_{i = 4, 6} -\frac{4}{3}J^{-4/3}\frac{\partial\Psi_\mathrm{aniso}}{\partial\bar{I}_i}
\left( \overline{\bold{C}}^{-1} \otimes \bold{A}_{0i} + \bold{A}_{0i} \otimes \overline{\bold{C}}^{-1} 
- \frac{2}{3} \bar{I}_i \overline{\bold{C}}^{-1} \otimes \overline{\bold{C}}^{-1} \right) \\
\end{split}
\end{equation}
Adding up Equations \ref{part13}, \ref{part23} and \ref{part33} we have the anisotropic part of the isochoric elasticity tensor for the HGO model:

\begin{equation} \label{Ciso31}
\begin{split}
\mathbb{C}_\mathrm{aniso} 
&= \sum_{i = 4, 6} J^{-4/3} \bigg[ 4\frac{\partial^2\Psi_\mathrm{aniso}}{\partial{\bar{I}_i}^2} (\bold{A}_{0i} \otimes \bold{A}_{0i}) - \frac{4}{3} \left( \bar{I}_i\frac{\partial^2\Psi_\mathrm{aniso}}{\partial{\bar{I}_i}^2} + \frac{\partial\Psi_\mathrm{aniso}}{\partial{\bar{I}_i}} \right) (\overline{\bold{C}}^{-1} \otimes \bold{A}_{0i} + \bold{A}_{0i} \otimes \overline{\bold{C}}^{-1}) \\
&+ \frac{4}{9} \left( {\bar{I}_i}^2\frac{\partial^2\Psi_\mathrm{aniso}}{\partial{\bar{I}_i}^2} + \bar{I}_i\frac{\partial\Psi_\mathrm{aniso}}{\partial{\bar{I}_i}} \right) (\overline{\bold{C}}^{-1} \otimes \overline{\bold{C}}^{-1}) 
+ \frac{4}{3}\bar{I}_i \frac{\partial\Psi_\mathrm{aniso}}{\partial{\bar{I}_i}}( \overline{\bold{C}}^{-1} \odot \overline{\bold{C}}^{-1}) \bigg]
\end{split}
\end{equation}
where the $\partial\Psi_{aniso}/\partial{\bar{I}_i}$ iand $\partial^2\Psi_{aniso}/{\partial{\bar{I}_i}^2}$ ($i = 4, 6$) are given in Equation \ref{convert}.

Combining Equation \ref{Ciso32} with Equation \ref{Ciso31} the complete isochoric part of the elasticity tensor of HGO model is:

\begin{equation} \label{Ciso3}
\begin{split}
\mathbb{C}_\mathrm{iso} = {}& 
- \frac{2}{3}J^{-4/3}\mu_1({\overline{\bold{C}}}^{-1} \otimes \bold{I} + \bold{I} \otimes {\overline{\bold{C}}}^{-1})  +
\frac{2}{9}J^{-4/3}\mu_1\bar{I_1} ( {\overline{\bold{C}}}^{-1} \otimes {\overline{\bold{C}}}^{-1}) + \frac{2}{3}J^{-4/3} \mu_1\bar{I_1} ({\overline{\bold{C}}}^{-1} \odot {\overline{\bold{C}}}^{-1})\\
&+ \sum_{i = 4, 6} J^{-4/3} \bigg[ 4\frac{\partial^2\Psi_\mathrm{aniso}}{\partial{\bar{I}_i}^2} (\bold{A}_{0i} \otimes \bold{A}_{0i}) - \frac{4}{3} \left( \bar{I}_i\frac{\partial^2\Psi_\mathrm{aniso}}{\partial{\bar{I}_i}^2} + \frac{\partial\Psi_\mathrm{aniso}}{\partial{\bar{I}_i}} \right) (\overline{\bold{C}}^{-1} \otimes \bold{A}_{0i} + \bold{A}_{0i} \otimes \overline{\bold{C}}^{-1}) \\
&+ \frac{4}{9} \left( {\bar{I}_i}^2\frac{\partial^2\Psi_\mathrm{aniso}}{\partial{\bar{I}_i}^2} - \bar{I}_i\frac{\partial\Psi_\mathrm{aniso}}{\partial{\bar{I}_i}} \right) (\overline{\bold{C}}^{-1} \otimes \overline{\bold{C}}^{-1} )
+ \frac{4}{3}\bar{I}_i \frac{\partial\Psi_\mathrm{aniso}}{\partial{\bar{I}_i}} (\overline{\bold{C}}^{-1} \odot \overline{\bold{C}}^{-1}) \bigg] 
\end{split}
\end{equation}
The volumetric part of the elasticity tensor is still the same as Equation \ref{Cvol1}.

Table \ref{summary1} lists a summary of the derived tensors in the reference configuration for all $3$ models, namely Mooney-Rivlin, Yeoh and HGO. An updated Lagrangian formulation can be derived by using a push-forward approach based on the reference configuration, The detailed derivation can be found in Appendix \ref{updated}. The corresponding summary of equations in the current configuration is shown in Table \ref{summary2}.






