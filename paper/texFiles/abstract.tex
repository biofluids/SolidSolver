\begin{abstract}

Hyperelastic models are of particular interest in modeling biomaterials. Researchers in biomechanics may not be well-exposed to systematic approaches to implement them with finite element method as they are vaguely addressed in literature. To resolve this, we present a framework of the general approach to derive the stress and elasticity tensors for hyperelastic models, and the derivation and implementation of the displacement/pressure mixed formulation. Additionally, we emphasize the treatment of the pressure load as it is commonly encountered in biomechanics and presents more complications than explicit traction boundary conditions. Three hyperelastic models, Mooney-Rivlin, Yeoh and Holzapfel-Gasser-Ogden models that span from first-order to higher order and from isotropic to anistropic materals, are served as examples. These detailed derivations and implementations are validated with numerical experiments that demonstrate excellent agreements with analytical and other computational solutions. More importantly, a step-by-step framework is built for modeling various types of hyperelastic materials using finite element. Following this framework, one could implement with ease any hyperelastic model as user-defined function in software packages or source code from scratch.

\end{abstract}