\begin{abstract}

Hyperelastic models are of particular interest in biomechanics. Unfortunately there are two obstacles impeding researchers from implementing them with finite element methods. The first one is the derivation of the stress and elasticity tensors and the other one is the displacement/pressure mixed formulation. Although both problems have been well solved, the researchers in biomechanics may not be well-exposed to these approaches because the systematic approaches are vaguely addressed in literature. To resolve this, we review the general approach to derive the stress and elasticity tensors for hyperelastic models, and the derivation and implementation of mixed formulation. Additionally, we emphasize the treatment of pressure load because it is more common and complicated than explicit traction boundary conditions in biomechanics. These approaches are validated with numerical experiments. More importantly, a step-by-step framework is built for the derivation and implementation, illustrated with examples. This framework is applicable for both isotropic and anisotropic models, therefore one could implement any hyperelastic model with the same approach.

\end{abstract}