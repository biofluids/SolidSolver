\section{Introduction}
Most biomaterials exhibit significant nonlinear deformation behavior, and hyperelastic models are often used to model this behavior. So far, hyperelastic models have been successfully applied on different kinds of biological materials. For example, it is reported in \cite{Qian}, the cell wall of onion epidermis could be modeled with anisotropic hyperelastic model, agreeing well with experimental data. In  \cite{OHagen}, a hyperelastic model is used to model Breast tissue and it is found that the material parameters of pathological tissues are significantly different from the healthy tissue. Differences in parameters are also found when the hyperelastic model is used to measure malignant tumors and benign tumors \cite{Oberai}. Another type of applications involves vessels. \cite{Zidi} shows that rat carotid artery can be modeled with a hyperelastic model, and agreement with experimental results is achieved. Human carotid artery is also modeled with hyperelastic model in \cite{Zidi2}. More recently, in \cite{Bols}, a hyperelastic model is used to simulate the stress distribution of the patient-specific vessels.
Other than these, numerous applications to the brain \cite{Karimi, Samani, Gilchrist}; the lung \cite{Wall, Wall2}; liver and kidney \cite{Fu, Untaroiu, Willinger} are reported. For a detailed review, please refer to \cite{Kupriyanova}.

According to \cite{Steinmann}, hyperelastic models can be classified as phenomenological or micro-mechanical. The latter are derived from statistical mechanics arguments on networks of idealized chain molecules while the former utilise more or less complex, frequently polynomial formulations in terms of strain invariants or principle stretches. We restrict the scope of this paper to the phenomenological models using principle invariants as they are the majority of the hyperelastic models and have more common characteristics. Among these models we pick out three typical ones as examples: Mooney-Rivlin model, Yeoh model and HGO model.



