\section{Introduction}
Hyperelastic models have been successfully applied to various types of biomaterial scaling from onion epidermis \cite{Qian} to breast tissues \cite{OHagen}, from carotid vessels \cite{Zidi, Zidi2, Bols} to breast tumors \cite{Oberai}, and a number of animal organs including brain \cite{Karimi, Samani, Gilchrist}, lung  \cite{Wall, Wall2}, liver and kidney \cite{Fu, Untaroiu, Willinger}. A detailed review of the applications of isotropic hyperelastic models on biological tissues is provided in \cite{Kupriyanova}.
According to \cite{Steinmann}, hyperelastic models can be classified as phenomenological or micro-mechanical. The latter are derived from statistical mechanics arguments on networks of idealized chain molecules while the former utilize more or less complex, frequently polynomial formulations in terms of strain invariants or principle stretches. Because of the popularity of hyperelastic models, many commercial software have hyperelastic models as a built-in material selection. For instance, Abaqus FEA offers Mooney-Rivlin model, Yeoh model, neo-Hookean model, etc. It also allows the user to provide a subroutine to define new models. But there are certain limitations with commercial software. One of them is that it is not allowed to model a completely incompressible material for the purpose of the robustness of the computation \cite{Abaqus}. However, as we will demonstrate in this work, with some simple modifications complete incompressibility in finite element method can be easily achieved based on nearly incompressible models.

Although hyperelastic models have been frequently used with finite element method to model the nonlinear deformation behavior, it is not straightforward to implement them because two key issues are vaguely addressed in literature. The first obstacle to implement hyperelastic models is the derivations of the stress and elasticity tensors. The stress tensor is used to form the equilibrium equation, and the elasticity tensor is the keystone to form the tangent stiffness matrix that is used to solve the equilibrium equation. There are few papers or books providing a systematic approach to evaluate stress and elasticity tensors. An indicial method to derive the stress and elasticity tensors for Mooney-Rivlin model is presented in \cite{Bower}, which in the author's opinion, is not easy to be transplanted to another model. In \cite{Belytschko}, a general method to find the tangent stiffness matrix for any hyperelastic model in index notation is introduced in an abstract form with no detailed example presented. A step-by-step method written in non-indicial form is derived in \cite{Holzapfel}, which we refer to frequently in this paper. But again limited examples for different models are shown and there is no computational results of the finite element implementation at all. Some derivations for specific models can be found in some early publications \cite{Weiss, Nicholson}. A more recent paper \cite{Suchocki} presents the implementation of Knowles model within Abaqus, using the non-indicial approach described in \cite{Holzapfel}. But instead of using the spatial tensor of elasticity, the author used Jaumann objective rate because it is used in Abaqus \cite{Abaqus}. In this review paper, we will present a systematic approach to derive the stress and elasticity tensors of any given hyperelastic model.


The second obstacle is the implementation of the mixed formulation. The mixed formulation in this context  refers to the displacement/pressure mixed formulation. As known to many, the tangent stiffness matrix becomes ill-conditioned if the compressibility of the material is small. The mixed formulation is able to ameliorate the condition of the tangent stiffness matrix as well as avoid locking in cases like vessel expansion. Furthermore, for completely incompressible material, the hydrostatic pressure is arbitrary. Therefore the pressure must be an independent unknown. With the mixed formulation, we are able to solve nearly or completely incompressible material easily. 
The mixed formulation was initially discussed in the context of linear elasticity by Fraeijs de Veubeke \cite{Veubeke} and Herrmann \cite{Herrmann}. The first attempt to apply mixed formulation to nonlinear materials dates back to the late 1960s \cite{Oden}. While the first successful application to rubberlike materials was done by Borst \cite{Borst}, in which several practical problems are solved. A number of more recent books discuss the topic of the mixed formulation as well, including \cite{Bathe}, \cite{Holzapfel}, \cite{Zienkiewicz}, etc. Unfortunately few of these works provides a workflow of the detailed implementation, which makes the mixed formulation a challenge to learn for beginners. Despite of its popularity, the displacement/pressure mixed formulation can be considered as an application of a more general principle: the Hu-Washizu principle, which allows simultaneous variation of displacements, strain, and stress \cite{Hu}. Within the framework of Hu-Washizu principle, the displacement/strain formulation is also developed \cite{Cervera, Rifai}, where the variational approach is used. This paper presents the derivation and implementation of the variational principle in detail, so that readers can easily follow the procedures to implement various types of mixed formulations. For example, the mixed formulation can be employed in approximating the variational inequalities of deformations of different bodies to solve contact problems \cite{Taylor}. Another example is that in non-Newtonian fluid mechanics, constitutive equation can be cast in a weighted-residual form by extracting the stress tensor out of the standard velocity/pressure formulation \cite{Baaijens}. 

This paper is motivated to address these two obstacles and clarify some of the frustrating issues presented in the existing literature in this field, as stated above. We present a framework for the derivation and the finite element implementation of hyperelastic models and walk the readers through the entire process with detailed examples. With the detailed derivation and the neat workflow, readers should be able to quickly implement any hyperelastic model with the standard displacement-based formulation or the mixed formulation. This is not only useful to the researchers who develop their own finite element code but also to those who work with commercial or open-source software with user-defined functions to include new constitutive models, especially for the ever-increasing newly defined models to describe biomaterials.

The rest of this paper is organized as follows: in Section \ref{general}, we introduce the basics of the hyperelastic model followed by the general procedures to derive the stress and elasticity tensors. To demonstrate how they can be applied for specific models, we present three different examples: Mooney-Rivlin model, Yeoh model, and Holzapfel-Gasser-Ogden (HGO) model. Mooney-Rivlin model is simple but very effective in modeling large strain nonlinear behavior of incompressible materials such as rubber and biomaterial. Yeoh model contains only one invariant but with a second order term. It is not complicated but can correctly predict the behavior of elastomer material in the range of a greater extent of deformation than the Mooney-Rivlin model \cite{Gajewski}, and is able to characterize the stiffening phenomenon of vulcanized rubber. The HGO model is designed to model collagen fiber-reinforced biological materials. This anisotropic model is so widely used that many commercial and open-source finite element codes have included it as standard or user-defined model,  such as Abaqus \cite{Abaqus}, COMSOL \cite{COMSOL} and FEBio \cite{FEBio}. In Section \ref{formulation}, we briefly review the theory of the principle of virtual work and the principle of stationary potential energy. With these theories, the origin of the mixed formulation is introduced and is further developed in the matrix form. Along the way we also discuss the pressure load which is common in biomechanics while more difficult to implement than the obvious given traction because it involves the deformation of the loaded surface. To complete the finite element formulation in solving hyperelastic material problems, the linearization of the principle of stationary potential energy is presented. In Section \ref{experiments}, three sets of numerical experiments are presented: biaxial tension, $2$D vessel expansion, and $2$-layer vessel expansion in $3$D. The experiments are performed using the three hyperelastic models, where the results are validated with analytical solutions and existing computational results whenever possible, and isotropic and anisotropic behaviors are observed and compared. Finally, the conclusions are drawn in Section \ref{conclusions}.




 



