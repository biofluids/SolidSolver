\section{Appendices}
In the updated Lagrangian Formulation, the Kirchhoff stress $\boldsymbol{\tau}$ or the Cauchy stress $\boldsymbol{\sigma}$ is used instead of the PK2 stress $\bold{S}$, which means the principle of virtual work or principle of stationary potential energy and their linearizations are also written in the current configuration. In the appendices we will introduce the corresponding formulations in the current configuration.

%
\subsection{Stress and Elasticity Tensor in the Current Configuration}
We work with the Kirchhoff stress $\boldsymbol{\tau}$ in the updated Lagrangian formulation, which is obtained by a push-forward operation of the PK2 stress $\bold{S}$.

\begin{equation}
\boldsymbol{\tau} = \bold{F}\bold{S}{\bold{F}}^T \label{current_stress}
\end{equation}
Recall the definition of the elasticity tensor $\mathbb{C}$ in Equation \ref{def_elasticity_tensor}, the spatial tensor of elasticity $\mathbb{c}$ is defined as

\begin{equation}
\pounds_{\bold{v}}({\boldsymbol{\tau}}^{\sharp}) = \mathbb{c} : \bold{d}
\end{equation} 
where $\pounds_{\bold{v}}({\boldsymbol{\tau}}^{\sharp})$ is the objective Oldroyd stress rate defined as $\pounds_{\bold{v}}({\boldsymbol{\tau}}^{\sharp}) = \dot{\boldsymbol{\tau}} - \bold{l}\boldsymbol{\tau} - \boldsymbol{\tau}\bold{l}^T$, $\bold{l} = \nabla_{\bold{x}}\bold{v}$ and $\bold{d} = (\bold{l} + {\bold{l}}^T)/2$. 
The spatial tensor of elasticity $\mathbb{c}$ can be transformed from the elasticity tensor $\mathbb{C}$ through a push-forward operation
\begin{equation}
c_{ijkl} = F_{iI}F_{jJ}F_{kK}F_{lL}C_{IJKL} \label{current_elasticity}
\end{equation}

Next we will derive the Kirchhoff stress $\boldsymbol{\tau}$ and the spatial tensor of elasticity $\mathbb{c}$ for Mooney-Rivlin model, Yeoh model and HGO model respectively. All these models have the same volumetric part, so we deal with it first. Substitute Equation \ref{Svol1} into Equation \ref{current_stress}

\begin{subequations}
\label{Tauvol1}
\begin{align}
\boldsymbol{\tau}_{vol} &= \bold{F}\bold{S}_{vol}{\bold{F}}^T \\
				    &= Jp\bold{I} \label{Tauvol11}\\
		      		    &= \kappa J(J-1) \bold{I} \label{Tauvol12}
\end{align}
\end{subequations}
Substituting Equation \ref{Cvol1} into Equation \ref{current_elasticity} gives the volumetric spatial tensor of elasticity, for the convenience, here we use the index notation

\begin{subequations} \label{spatial_cvol1}
\begin{align}
(c_{vol})_{ijkl} &= F_{iI}F_{jJ}F_{kK}F_{lL}(C_{vol})_{IJKL} \\
&= J\tilde{p}\delta_{ij}\delta_{kl} - Jp(\delta_{ik}\delta_{jl} + \delta_{il}\delta_{jk}) \label{spatial_cvol11} \\
&= \kappa[J(2J-1)\delta_{ij}\delta_{kl} - J(J-1)(\delta_{ik}\delta_{jl} + \delta_{il}\delta_{jk})] \label{spatial_cvol12}
\end{align}
\end{subequations} 

Equation \ref{Tauvol1} and \ref{spatial_cvol1} complete the volumetric part of the Kirchhoff stress and spatial tensor of elasticity for all these models. Next we will derive the isochoric parts individually.

The isochoric Kirchhoff stress $\boldsymbol{\tau}_{iso}$ for Mooney-Rivlin model is transformed from Equation \ref{Siso1}:
\begin{equation} \label{Tauiso1}
\begin{split}
\boldsymbol{\tau}_{iso} &= \bold{F}\bold{S}_{iso}{\bold{F}}^T \\
	   			    &= \bold{F}  J^{-2/3} [    - \frac{1}{3}(\mu_1\bar{I_1} + 2\mu_2\bar{I_2}) \overline{\bold{C}}^{-1}  + (\mu_1 + \mu_2\bar{I_1})\bold{I} - \mu_2\overline{\bold{C}} ]   \bold{F}^T \\
				    &= - \frac{1}{3}(\mu_1\bar{I_1} + 2\mu_2\bar{I_2})\bold{I} - \mu_2{\overline{\bold{B}}}^2 + (\mu_1 + \mu_2\bar{I_1}){\overline{\bold{B}}}
\end{split}
\end{equation}
where $\overline{\bold{B}}$ is the modified right Cauchy-Green tensor defined as $\overline{\bold{B}} = \overline{\bold{F}}\overline{\bold{F}}^T = \overline{\bold{C}}^T$
And the isochoric spatial tensor of elasticity $\mathbb{c}_{iso}$ is obtained from Equation \ref{Ciso1}:

\begin{equation} \label{spatial_ciso1}
\begin{split}
(c_{iso})_{ijkl} &=  F_{iI}F_{jJ}F_{kK}F_{lL}(C_{iso})_{IJKL} \\
&= 
2\mu_2[{\overline{\bold{B}}}_{ij} {\overline{\bold{B}}}_{kl} - \frac{1}{2}({\overline{\bold{B}}}_{ik}{\overline{\bold{B}}}_{jl} + {\overline{\bold{B}}}_{il}{\overline{\bold{B}}}_{jk}) ] 
- \frac{2}{3}(\mu_1 + 2\mu_2\bar{I_1})({\overline{\bold{B}}}_{ij}\delta_{kl} + {\overline{\bold{B}}}_{kl}\delta_{ij}) 
+ \frac{4}{3}\mu_2({\overline{\bold{B}}}^2_{ij}\delta_{kl} + {\overline{\bold{B}}}^2_{kl}\delta_{ij}) \\
&+
\frac{2}{9}(\mu_1\bar{I_1} + 4\mu_2\bar{I_2}) \delta_{ij}\delta_{kl}
+ \frac{1}{3}(\mu_1\bar{I_1} + 2\mu_2\bar{I_2})(\delta_{ik}\delta_{jl} + \delta_{il}\delta_{jk})
\end{split}
\end{equation} 

Similarly, the isochoric stress $\boldsymbol{\tau}_{iso}$ and the isochoric spatial tensor of elasticity $\mathbb{c}_{iso}$ for Yeoh model are transformed from Equation \ref{Siso2} and \ref{Ciso2}:

\begin{equation} \label{Tauiso2}
\begin{split}
\boldsymbol{\tau}_{iso} 
&= \bold{F}\bold{S}_{iso}{\bold{F}}^T \\
&= \bold{F}  J^{-2/3}[2c_1 + 4c_2(\bar{I_1} - 3) + 6c_3(\bar{I_1} - 3)^2] (\bold{I} - \frac{1}{3}\bar{I_1}{\overline{\bold{C}}}^{-1})  \bold{F}^T \\
&= [2c_1 + 4c_2(\bar{I_1} - 3) + 6c_3(\bar{I_1} - 3)^2](\overline{\bold{B}} - \frac{1}{3}\bar{I}_1\bold{I})
\end{split}
\end{equation}

\begin{equation} \label{spatial_ciso2}
\begin{split}
(c_{iso})_{ijkl} &=  F_{iI}F_{jJ}F_{kK}F_{lL}(C_{iso})_{IJKL} \\
&= 
[8c_2+24c_3({\bar{I}}_1 - 3)] {\overline{\bold{B}}}_{ij}{\overline{\bold{B}}}_{kl} 
- [\frac{4}{3}c_1 + (\frac{16}{3}{\bar{I}}_1 - 8)c_2 + 12({\bar{I}}_1 - 1)({\bar{I}}_1 - 3)c_3]({\overline{\bold{B}}}_{ij}\delta_{kl} + {\overline{\bold{B}}}_{kl}\delta_{ij}) \\
&+ [\frac{4}{9}{\bar{I}}_1c_1 + (\frac{16}{9}{{\bar{I}}_1}^2 - \frac{8}{3}{\bar{I}}_1)c_2 + 4{\bar{I}}_1({\bar{I}}_1 - 1)({\bar{I}}_1 - 3)c_3] \delta_{ij}\delta_{kl} \\
&+ [\frac{2}{3}c_1{\bar{I}}_1 + \frac{4}{3}c_2({\bar{I}}_1 - 3){\bar{I}}_1 + 2c_3({\bar{I}}_1 - 3)^2{\bar{I}}_1](\delta_{ik}\delta_{jl} + \delta_{il}\delta_{jk})
\end{split}
\end{equation} 


%
\subsection{Virtual Internal Work and Its Linearization in the Current Configuration}



