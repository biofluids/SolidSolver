\section{Tests}
In this section, we present two set of test cases to validate the results we obtained with the formulations we derived.
\subsection{Biaxial Tension Test}
A long bar loaded equally in two directions with fixed tractions is considered. The section is a $1$ $m^2$ square. Define the stretch $\lambda$ as the ratio between the deformed length to the original length, then for isotropic incompressible material, the stress is easy to find.
\begin{equation}
S_{11} = S_{22} = 2(1 - {\lambda}^{-6})(\frac{\partial\Psi_{iso}}{\partial\bar{I}_1} + {\lambda}^2\frac{\partial\Psi_{iso}}{\partial\bar{I}_2}), \quad S_{33} = 0
\end{equation}
where $S_{11}$, $S_{22}$, $S_{33}$ are the diagonal components of the PK2 stress, all the off-diagonal components are $0$.

Denote the nonzero component of PK2 stress as $S$. For Mooney-Rivlin model, substitute Equation \ref{iso} and \ref{isopart} into the last equation, we have
\begin{equation}
S = (1 - {\lambda}^{-6})(\mu_1 + \mu_2{\lambda}^2)
\end{equation}
and therefore the nominal stress $P$ is
\begin{equation}
P = \lambda S =  (\lambda - {\lambda}^{-5})(\mu_1 + \mu_2{\lambda}^2)
\end{equation}
Neglecting the $\mu_2$ terms we have the nominal stresse for Neo-Hookean model
\begin{equation}
P = \lambda S =  \mu_1(\lambda - {\lambda}^{-5})
\end{equation}
Similarly, for Yeoh model
\begin{equation}
P = \lambda S = 2(\lambda - {\lambda}^{-5})[c_1 + 2c_2(2{\lambda}^2 + {\lambda}^{-4} - 3) + 3c_3(2{\lambda}^2 + {\lambda}^{-4} - 3)^2]
\end{equation}

The material constants used in these models are curve-fitted from the standard ASTM412 tensile test results of the same material. See Shashikant. For Neo-Hookean model, $\mu_1 = 0.6548$ MPa; for Mooney-Rivlin model, $\mu_1 = 0.595522$ MPa, $\mu_2 = -0.0508009$ MPa; for Yeoh model, $c_1 = 0.358756$ MPa, $c_2 = - 0.0508009$ MPa, $c_3 = 0.0142132$ MPa. The Poisson's ratios for all the models are over $\nu = 0.49999$, closed enough to incompressibility.

Figure \ref{fig:biaxial1} shows the results of the theoretical calculation and numerical computation. Perfect agreement is achieved. As expected, in the small strain range, all these models have similar linear performance. Also, it can be seen that $\mu_2$ in Mooney-Rivlin model acts as a modification to $\mu_1$ in Neo-Hookean model which increases the stiffness. In large strain range, the stiffness of Yeoh model grows quickly, becoming significantly larger than the others.
\begin{figure}[h!]
\centering
\includegraphics[width=.6\textwidth]{./figures/biaxial1.png}
\caption{Biaxial Tension Test}
\label{fig:biaxial1}
\end{figure}


\subsection{Cylindrical Pressure Vessel}