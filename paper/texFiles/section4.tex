\section{Numerical Experiments}
In this section, we present two sets of test cases to validate the results we obtained with the formulations we derived. The first case is the biaxial tension test which is often used to calibrate material constants. For isotropic materials, the analytical solution is easy to find if we assume the material is incompressible. The second case is the expansion of a vessel, which is a very representative application in the field of biomechanics.  

\subsection{Biaxial Tension Test}
\label{biaxial_tension_test}
In this test, we examine $3$ isotropic models and $1$ anisotropic model. Consider a long bar loaded equally in two directions with fixed tractions. The section of the bar is a $1$ $m^2$-square. As shown in Figure \ref{fig:biaxial_schematic}, the bottom surface is fixed in the $x$-direction; the left surface is fixed in the $z$-direction and the back surface is fixed in the $y$-direction. External tractions are applied to the front and top surfaces. The stretch $\lambda$ is defined as the ratio between the deformed length to the original length, for isotropic incompressible material, the stress is expressed as:
\begin{equation}
S_{11} = S_{22} = 2(1 - {\lambda}^{-6})(\frac{\partial\Psi_{iso}}{\partial\bar{I}_1} + {\lambda}^2\frac{\partial\Psi_{iso}}{\partial\bar{I}_2}), \quad S_{33} = 0
\end{equation}
where $S_{11}$, $S_{22}$, $S_{33}$ are the diagonal components of the PK2 stress, all the off-diagonal components are $0$.

Since the nonzero components of PK2 stress are the same, we denote them as $S$. For Mooney-Rivlin model, plug in Equation \ref{iso1}, we have

\begin{equation}
S = (1 - {\lambda}^{-6})(\mu_1 + \mu_2{\lambda}^2)
\end{equation}
and therefore the nominal stress $P$ is
\begin{equation}
P = \lambda S =  (\lambda - {\lambda}^{-5})(\mu_1 + \mu_2{\lambda}^2)
\end{equation}
Neglecting the $\mu_2$ terms we have the nominal stresse for Neo-Hookean model
\begin{equation}
P = \lambda S =  \mu_1(\lambda - {\lambda}^{-5})
\end{equation}
Similarly, for Yeoh model
\begin{equation}
P = \lambda S = 2(\lambda - {\lambda}^{-5})[c_1 + 2c_2(2{\lambda}^2 + {\lambda}^{-4} - 3) + 3c_3(2{\lambda}^2 + {\lambda}^{-4} - 3)^2]
\end{equation}

\begin{figure}[h!]
\centering
\includegraphics[width=.8\textwidth]{./figures/biaxial_schematic.png}
\caption{Schematic Diagram of Biaxial Tension Test}
\label{fig:biaxial_schematic}
\end{figure}

Table \ref{parameters} shows the material constants used in these models, which are curve-fitted from the standard ASTM412 tensile test results of the rubber used in the transmission mounts \cite{Sharma}. The Poisson's ratios for all the models are over $\nu = 0.49999$, close enough to incompressibility.

\begin{table}
\centering
\caption{Material Parameters}
\begin{tabular} { l  l  l }
	\hline
	Neo-Hookean & Mooney-Rivlin & Yeoh \\
	\hline
	$\mu_1 = 0.595522$ MPa & $\mu_1 = 0.595522$ MPa & $c_1 = 0.358756$ MPa \\
	& $\mu_2 = 0.050381$ MPa & $c_2 = - 0.0508009$ MPa \\
	& & $c_3 = 0.0142132$ MPa \\
	\hline
	For all models, $\kappa = 1 \times 10^5 $ MPa \\
	\hline
\end{tabular}
\label{parameters}
\end{table}

Figure \ref{fig:biaxial1} shows the results of the theoretical calculations and numerical computations for all $3$ models for $\lambda$ in the range of $1$ to $1.8$. Perfect agreement is achieved. As expected, in the small strain range, all these models have similar linear performance. Also, it can be seen that $\mu_2$ in Mooney-Rivlin model acts as a modification to $\mu_1$ in Neo-Hookean model which increases the stiffness. In large strain range, the stiffness of Yeoh model grows quickly, becoming significantly larger than the others due to the inclusion of the higher order terms.


\begin{figure}[t!p]
	\centering
	\begin{subfigure}[b]{0.45\textwidth}
		\centering
		\includegraphics[width=\textwidth]{./figures/biaxial1.png}
		\caption{Biaxial Tension Test for Isotropic Models}
		\label{fig:biaxial1}
	\end{subfigure}
	\begin{subfigure}[b]{0.45\textwidth}
		\centering
		\includegraphics[width=\textwidth]{./figures/biaxial2.png}
		\caption{Biaxial Tension Test for HGO Model}
		\label{fig:biaxial2}
	\end{subfigure}
\end{figure}


To demonstrate the anisotropy of HGO model, we compare the stretches in two directions in the biaxial tension test. We use the material parameters calibrated from the media layer of the artery in \cite{Holzapfel2} but only one direction is strengthened. That is, both two families of fibers are arranged in the same direction while the tractions are still applied in two directions. The material parameters are $\mu_1 = 3$ KPa, $k_1 = 2.3632$ KPa, $k_2 = 0.8393$ and $\kappa = 1 \times 10^5$ KPa. We vary the direction of the fibers measured by the angle from the $x$ direction denoted as $\alpha$, from $0^\circ$ to $90^\circ$. 

The stretch ratio $\lambda$ is defined in Equation \ref{stretchratio}, and Figure \ref{fig:biaxial2} shows the nominal stress as a function of stretch ratio. Just as what we expect, the stretch ratio at $\alpha = 0^\circ$ is the same as $\alpha = 90^\circ$, and same for $\alpha = 30^\circ$ and $\alpha = 60^\circ$. 
$\lambda_y$ is greater than $\lambda_x$ when $0^\circ < \alpha < 45^\circ$ since the $x$ direction is strengthened more. As the angle gets larger, $\lambda_x$ becomes greater too. When $\alpha = 45^\circ$ two directions become symmetric again.

\begin{equation}
\label{stretchratio}
\lambda = 
\begin{cases}
	\lambda_y/\lambda_x, & \text{$\alpha = 0^\circ$, $30^\circ$ or $45^\circ$} \\
	\lambda_x/\lambda_y, & \text{$\alpha = 60^\circ$ or $90^\circ$}
\end{cases}
\end{equation}
 

\subsection{Cylindrical Pressure Vessel}
\label{pressure_vessel}
In the second test, we consider a vessel under lumen pressure, which is a typical application in biomechanics. The vessel has an internal radius of $R_i = 7$ m and an external radius of $R_o = 18.625$ m. Only a quarter of vessel is considered with plain strain assumption as shown in Figure \ref{fig:vessel_schematic}. $21$ nodes are uniformly distributed in the radial direction. The material parameters we used is the same as in Section \ref{biaxial_tension_test} except that in the HGO model, two families of fibers are arranged in the $\theta-z$ plane with constant angles to the circumferential direction of $|\beta| = 50^\circ$, as shown in Figure \ref{fig:vessel_schematic2}.

\begin{figure}[t!p]
	\centering
	\begin{subfigure}[b]{0.4\textwidth}
		\centering
		\includegraphics[width=\textwidth]{./figures/vessel_schematic.jpg}
		\caption{Pressure Vessel}
		\label{fig:vessel_schematic}
	\end{subfigure}
	\begin{subfigure}[b]{0.4\textwidth}
		\centering
		\includegraphics[width=\textwidth]{./figures/vessel_schematic2.png}
		\caption{Fiber Alignments}
		\label{fig:vessel_schematic2}
	\end{subfigure}
\end{figure}

As a benchmark, we model the vessel with Mooney-Rivlin model and compare the radial displacement, hoop stress, radial stress, and axial stress with the analytical solutions. The analytical solutions for incompressible material are found in \cite{Green} as follows: 
\begin{subequations}
\begin{align}
p_i &= 2(\mu_1 + \mu_2)(log\frac{{R_i}^2 + b}{{R_o}^2 + b} - 2log\frac{R_i}{R_o} +
b\frac{{R_o}^2 - {R_i}^2}{({R_o}^2+b)({R_i}^2+b)}) \\
u_r &= -R + \sqrt{R^2 + b} \\
p &= - p_i - \mu_2 - (\mu_1 + \mu_2)(log\frac{r}{R_i} + \frac{b}{2}(r^2 - {R_i}^2) - log\frac{R}{R_i} + {(\frac{R_i}{r})}^2) ) \\
\sigma_{\theta\theta} &= p + \mu_2 + (\mu_1 + \mu_2)(\frac{r}{R})^2 + C \\
\sigma_{rr} &= p + \mu_2 + (\mu_1 + \mu_2)(\frac{R}{r})^2 + C \\
\sigma_{zz} &= p +  \mu_1 + \mu_2[(\frac{R}{r})^2 + (\frac{r}{R})^2] + C
\end{align}
\end{subequations}
where $R$ and $r$ are the radial coordinate before and after deformation, $p$ is the hydrostatic pressure, $u_r$ is the radial displacement, and $\sigma_{\theta\theta}$, $\sigma_{rr}$ and $\sigma_{zz}$ are the hoop, radial and axial stresses respectively. $b$ is a constant determined by the internal pressure $p_i$, and $C$ is an arbitrary constant since the material is incompressible. 

Fig \ref{fig:mooney-rivlin1} shows the stress components and the radial displacement along radius direction under the pressure of $p_i = 200$ KPa. Good agreement is achieved on all the components. Furthermore, we pick the node at the midpoint in the radial direction ($R = 12.8125$ m) and study its deformation under different internal pressures. As shown in Fig \ref{fig:mooney-rivlin2}, the computational result and analytical solution agree perfectly with each other. 

\begin{figure}[t!p]
	\begin{subfigure}[b]{0.5\textwidth}
		\centering
		\includegraphics[width=\textwidth]{./figures/ur_200.png}
		\caption{Radial Displacement}
		\label{ur_200}
	\end{subfigure}
	\begin{subfigure}[b]{0.5\textwidth}
		\centering
		\includegraphics[width=\textwidth]{./figures/hoop_stress_200.png}
		\caption{Hoop Stress}
		\label{hoop_200}
	\end{subfigure}
	
	\begin{subfigure}[b]{0.5\textwidth}
		\centering
		\includegraphics[width=\textwidth]{./figures/radial_stress_200.png}
		\caption{Radial Stress}
		\label{radial_200}
	\end{subfigure}
	\begin{subfigure}[b]{0.5\textwidth}
		\centering
		\includegraphics[width=\textwidth]{./figures/axial_stress_200.png}
		\caption{Axial Stress}
		\label{axial_200}
	\end{subfigure}
	\caption{Vessel Expansion Under $p_i = 200$ KPa with Mooney-Rivlin Model}
	\label{fig:mooney-rivlin1}
\end{figure}

\begin{figure}[t!p]
	\begin{subfigure}[b]{0.5\textwidth}
		\centering
		\includegraphics[width=\textwidth]{./figures/ur.png}
		\caption{Radial Displacement}
		\label{ur}
	\end{subfigure}
	\begin{subfigure}[b]{0.5\textwidth}
		\centering
		\includegraphics[width=\textwidth]{./figures/hoop.png}
		\caption{Hoop Stress}
		\label{hoop}
	\end{subfigure}
	
	\begin{subfigure}[b]{0.5\textwidth}
		\centering
		\includegraphics[width=\textwidth]{./figures/radial.png}
		\caption{Radial Stress}
		\label{radial}
	\end{subfigure}
	\begin{subfigure}[b]{0.5\textwidth}
		\centering
		\includegraphics[width=\textwidth]{./figures/axial.png}
		\caption{Axial Stress}
		\label{axial}
	\end{subfigure}
	\caption{Vessel Expansion Under Different Pressure with Mooney-Rivlin Model}
	\label{fig:mooney-rivlin2}
\end{figure}

Now that the validity of our computation is confirmed. We then compare the results of Neo-Hookean, Mooney-Rivlin, Yeoh and HGO model. Figure \ref{fig:models} shows the results of these models under an inner pressure of $p_i = 350$ KPa. As expected, the isotropic models have similar trends, and the differences decrease from the inner surface to the outer surface. It means that under a moderate pressure, different isotropic models do not make too much difference. However, the two families of fibers in HGO model causes significant differences even though the ground material is modeled with Neo-Hookean with identical parameters. Due to the strengthening of the fibers, the radial displacement drops significantly. The trend of the axial stress is completely opposite to the isotropic models, i.e., the maximum axial stress appears at the inner radius. 

\begin{figure}[t!p]
	\begin{subfigure}[b]{0.5\textwidth}
		\centering
		\includegraphics[width=\textwidth]{./figures/ur_models.png}
		\caption{Radial Displacement}
		\label{ur_models}
	\end{subfigure}
	\begin{subfigure}[b]{0.5\textwidth}
		\centering
		\includegraphics[width=\textwidth]{./figures/hoop_models.png}
		\caption{Hoop Stress}
		\label{hoop_models}
	\end{subfigure}
	
	\begin{subfigure}[b]{0.5\textwidth}
		\centering
		\includegraphics[width=\textwidth]{./figures/radial_models.png}
		\caption{Radial Stress}
		\label{radial_models}
	\end{subfigure}
	\begin{subfigure}[b]{0.5\textwidth}
		\centering
		\includegraphics[width=\textwidth]{./figures/axial_models.png}
		\caption{Axial Stress}
		\label{axial_models}
	\end{subfigure}
	\caption{Vessel Expansion with Different Models under $350$ KPa}
	\label{fig:models}
\end{figure}

As the last case, we use HGO model to simulate the expansion of the carotid artery of a healthy rabbit. The real artery tissues consist three layers: intima, media, and adventitia. In this case we only consider media and adventitia because the intima is made up of one layer of endothelial cells and is not of our mechanical interest. Therefore the artery is modeled as a cylinder with $2$ layers. The geometry of the artery is shown in Figure \ref{fig:vessel_schematic3}. The fibers are still arranged in the $\theta-z$ plane, being symmetric about the $\theta$-direction. The angles between the fiber directions and the circumferential direction are denoted as $\pm\beta$. The values of the material parameters differ in the two layers, determined by Holzapfel \cite{Holzapfel2} , as shown in Table \ref{table:artery}. 

\begin{figure}[h!]
\centering
\includegraphics[width=.3\textwidth]{./figures/vessel_schematic3.png}
\caption{Schematic Diagram for the Artery Model}
\label{fig:vessel_schematic3}
\end{figure}

\begin{table}
\centering
\caption{The Configuration of the Artery Model}
\label{table:artery}
\begin{tabular}{ l l l l l l l}
\hline
& $\mu_1$/KPa & $k_1$/KPa & $k_2$/- & $\kappa$/KPa & $r$/m & $\beta/^\circ$ \\
 \hline
 Media &   $3.0$ & $2.3632$ & $0.8393$ & $10^5$ & $[0.71, 0.97)$ & $29.0$\\
 Adventitia & $0.3$ & $0.5629$ & $0.7112$ & $10^5$ & [0.97, 1.43] & $62.0$\\
 \hline
\end{tabular}
\end{table}

Figure \ref{fig:artery} shows the computational results. Because the media is much stiffer than the adventitia, the stress gradient is significant in the media while negligible in the adventitia. 

\begin{figure}[t!p]
	\begin{subfigure}[b]{0.5\textwidth}
		\centering
		\includegraphics[width=\textwidth]{./figures/artery_ur.png}
		\caption{Radial Displacement}
		\label{ur_artery}
	\end{subfigure}
	\begin{subfigure}[b]{0.5\textwidth}
		\centering
		\includegraphics[width=\textwidth]{./figures/artery_hoop.png}
		\caption{Hoop Stress}
		\label{hoop_artery}
	\end{subfigure}
	
	\begin{subfigure}[b]{0.5\textwidth}
		\centering
		\includegraphics[width=\textwidth]{./figures/artery_radial.png}
		\caption{Radial Stress}
		\label{radial_artery}
	\end{subfigure}
	\begin{subfigure}[b]{0.5\textwidth}
		\centering
		\includegraphics[width=\textwidth]{./figures/artery_axial.png}
		\caption{Axial Stress}
		\label{axial_artery}
	\end{subfigure}
	\caption{Rabbit Carotid Artery under $10$ KPa}
	\label{fig:artery}
\end{figure}

%As the last test case we compare HGO model to Neo-Hookean model. The parameters used in the HGO model is still the same as in Section \ref{biaxial_tension_test}, the geometry and mesh are the same as in previous cases. The pressure is $1.5$ KPa. To demonstrate the effect of the anisotropic part of the HGO model, we use the same $\mu_1$ and $\kappa$ in the Neo-Hookean model. This time, the two families of fibers are arranged in the $\theta-z$ plane with constant angles to the circumferential direction of $\pm 50^\circ$. From Figure \ref{fig:nh_hgo} we can see that overall the vessel is significantly strengthened as the radial displacement is much smaller. While the radial stress does not change much (similar to Yeoh model), the hoop stress is smaller but has a similar trend. However the axial stress changes dramatically. In Neo-Hookean model the inner surface has the maximum axial stress while in HGO model the outer surface has the maximum axial stress. But overall both hoop stress and axial stress are much smaller in the HGO model.

















