\section{Tests}
In this section, we present two sets of test cases to validate the results we obtained with the formulations we derived. The first case is the biaxial tension test which is used to calibrate material constants. For isotropic materials, the analytical solution is easy to find. The second case is the expansion of a vessel, which is a very representative application in the field of biomechanics.  

\subsection{Biaxial Tension Test}
A long bar loaded equally in two directions with fixed tractions is considered. The section is a $1$ $m^2$ square. Define the stretch $\lambda$ as the ratio between the deformed length to the original length, then for isotropic incompressible material, the stress is easy to find.
\begin{equation}
S_{11} = S_{22} = 2(1 - {\lambda}^{-6})(\frac{\partial\Psi_{iso}}{\partial\bar{I}_1} + {\lambda}^2\frac{\partial\Psi_{iso}}{\partial\bar{I}_2}), \quad S_{33} = 0
\end{equation}
where $S_{11}$, $S_{22}$, $S_{33}$ are the diagonal components of the PK2 stress, all the off-diagonal components are $0$.

Denote the nonzero component of PK2 stress as $S$. For Mooney-Rivlin model, plug in Equation \ref{iso1}, we have

\begin{equation}
S = (1 - {\lambda}^{-6})(\mu_1 + \mu_2{\lambda}^2)
\end{equation}
and therefore the nominal stress $P$ is
\begin{equation}
P = \lambda S =  (\lambda - {\lambda}^{-5})(\mu_1 + \mu_2{\lambda}^2)
\end{equation}
Neglecting the $\mu_2$ terms we have the nominal stresse for Neo-Hookean model
\begin{equation}
P = \lambda S =  \mu_1(\lambda - {\lambda}^{-5})
\end{equation}
Similarly, for Yeoh model
\begin{equation}
P = \lambda S = 2(\lambda - {\lambda}^{-5})[c_1 + 2c_2(2{\lambda}^2 + {\lambda}^{-4} - 3) + 3c_3(2{\lambda}^2 + {\lambda}^{-4} - 3)^2]
\end{equation}

The material constants used in these models are curve-fitted from the standard ASTM412 tensile test results of the same material. See Shashikant. In Neo-Hookean model, $\mu_1 = 0.6548$ MPa; in Mooney-Rivlin model, $\mu_1 = 0.595522$ MPa, $\mu_2 = -0.0508009$ MPa; in Yeoh model, $c_1 = 0.358756$ MPa, $c_2 = - 0.0508009$ MPa, $c_3 = 0.0142132$ MPa. The Poisson's ratios for all the models are over $\nu = 0.49999$, closed enough to incompressibility.

Figure \ref{fig:biaxial1} shows the results of the theoretical calculation and numerical computation. Perfect agreement is achieved. As expected, in the small strain range, all these models have similar linear performance. Also, it can be seen that $\mu_2$ in Mooney-Rivlin model acts as a modification to $\mu_1$ in Neo-Hookean model which increases the stiffness. In large strain range, the stiffness of Yeoh model grows quickly, becoming significantly larger than the others.

\begin{figure}[h!]
\centering
\includegraphics[width=.6\textwidth]{./figures/biaxial1.png}
\caption{Biaxial Tension Test for Isotropic Models}
\label{fig:biaxial1}
\end{figure}

To demonstrate the anisotropy of HGO model, we compare the stretches in two directions in the biaxial tension test. We use the material parameters calibrated from the media layer of artery in [ ] but only one direction is strengthened. That is, both two families of fibers are arranged in the same direction. The material parameters are $\mu_0 = 3$ KPa, $k_1 = 2.3632$ KPa and $k_2 = 0.8393$. We change the direction of the fibers which measured by the angle from the $x$ direction denoted as $\alpha$. 

We define the stretch ratio $\lambda$ in Equation \ref{stretchratio}, and plot the stretch ratio against the nominal stress in Figure \ref{fig:biaxial2}. Just as what we expect, the stretch ratio at $\alpha = 0^\circ$ is the same as $\alpha = 90^\circ$, and similarly for $\alpha = 30^\circ$ and $\alpha = 60^\circ$. 
$\lambda_y$ is greater than $\lambda_x$ when $0^\circ < \alpha < 45^\circ$ since the $x$ direction is strengthened more. As the angle gets larger, $\lambda_x$ becomes greater too. While $\alpha = 45^\circ$ two directions become symmetric again.

\begin{equation}
\label{stretchratio}
\lambda = 
\begin{cases}
	\lambda_y/\lambda_x, & \text{$\alpha = 0^\circ$, $30^\circ$ or $45^\circ$} \\
	\lambda_x/\lambda_y, & \text{$\alpha = 60^\circ$ or $90^\circ$}
\end{cases}
\end{equation}
 
\begin{figure}[h!]
\centering
\includegraphics[width=.6\textwidth]{./figures/biaxial2.png}
\caption{Biaxial Tension Test for HGO Model}
\label{fig:biaxial2}
\end{figure}

\subsection{Cylindrical Pressure Vessel}
As a typical application, we consider an artery under lumen pressure. The vessel has an internal radius of $R_i = 0.6$ mm and an external radius of $R_o = 0.9$ mm. Only a quarter of vessel is considered with plain strain assumption. First we model the vessel in Mooney-Rivlin model. The material parameters used are $\mu_1 = 160$ MPa and $\mu_2 = 40$ MPa. $\kappa$ is set to $1 \times 10^5$ MPa to make the material nearly incompressible. We compare the radial displacement, hoop stress, radial stress, and axial stress with the analytical solutions under the pressure of $p_i = 50$ MPa. The analytical solutions are found in [] as follows 














