\begin{appendices}
\section{Updated Lagrangian Formulation}
In the updated Lagrangian Formulation, the Kirchhoff stress $\boldsymbol{\tau}$ or the Cauchy stress $\boldsymbol{\sigma}$ is used instead of the PK2 stress $\bold{S}$, meanwhile the principle of virtual work or principle of stationary potential energy and their linearization are written in the current configuration as well. In this appendix we will show the corresponding formulations in the current configuration.

%
The stress tensor we work with in the updated Lagrangian formulation is the Kirchhoff stress $\boldsymbol{\tau}$, which is obtained by a push-forward operation of the PK2 stress $\bold{S}$:

\begin{equation}
\boldsymbol{\tau} = \bold{F}\bold{S}{\bold{F}}^T \label{current_stress}
\end{equation}
Recall the definition of the elasticity tensor $\mathbb{C}$ in Equation \ref{def_elasticity_tensor}, the spatial tensor of elasticity $\mathbb{c}$ is defined as:

\begin{equation}
\mathbb{c} = \frac{\partial{ \pounds_{\bold{v}}({\boldsymbol{\tau}}^{\sharp}) } }{\partial{ \bold{d} }}
\end{equation} 
where $\pounds_{\bold{v}}({\boldsymbol{\tau}}^{\sharp})$ is the objective Oldroyd stress rate defined as: $\pounds_{\bold{v}}({\boldsymbol{\tau}}^{\sharp}) = \dot{\boldsymbol{\tau}} - \bold{l}\boldsymbol{\tau} - \boldsymbol{\tau}\bold{l}^T$, $\bold{l} = \nabla_{\bold{x}}\bold{v}$ and $\bold{d} = (\bold{l} + {\bold{l}}^T)/2$. 
The spatial tensor of elasticity $\mathbb{c}$ can be transformed from the elasticity tensor $\mathbb{C}$ through a push-forward operation:
\begin{equation}
c_{ijkl} = F_{iI}F_{jJ}F_{kK}F_{lL}C_{IJKL} \label{current_elasticity}
\end{equation}

Next we will derive the Kirchhoff stress $\boldsymbol{\tau}$ and the spatial tensor of elasticity $\mathbb{c}$ for Mooney-Rivlin model, Yeoh model and HGO model respectively. All these models have the same volumetric part, which can be obtained by substituting Equation \ref{Svol1} into Equation \ref{current_stress}:

\begin{subequations}
\label{Tauvol1}
\begin{align}
\boldsymbol{\tau}_\mathrm{vol} &= \bold{F}\bold{S}_\mathrm{vol}{\bold{F}}^T \\
				    &= Jp\bold{I} \label{Tauvol11}\\
		      		    &= J(J-1)\kappa \bold{I} \label{Tauvol12}
\end{align}
\end{subequations}
Substituting Equation \ref{Cvol1} into Equation \ref{current_elasticity} gives the volumetric spatial tensor of elasticity. For convenience, here we use the index notation:

\begin{subequations} \label{spatial_cvol1}
\begin{align}
(c_\mathrm{vol})_{ijkl} &= F_{iI}F_{jJ}F_{kK}F_{lL}(C_\mathrm{vol})_{IJKL} \\
&= J\tilde{p}\delta_{ij}\delta_{kl} - Jp(\delta_{ik}\delta_{jl} + \delta_{il}\delta_{jk}) \label{spatial_cvol11} \\
&= \kappa[J(2J-1)\delta_{ij}\delta_{kl} - J(J-1)(\delta_{ik}\delta_{jl} + \delta_{il}\delta_{jk})] \label{spatial_cvol12}
\end{align}
\end{subequations} 
Equations \ref{Tauvol1} and \ref{spatial_cvol1} complete the volumetric part of the Kirchhoff stress and spatial tensor of elasticity for all these models. Next we will derive the isochoric parts individually.

The isochoric Kirchhoff stress $\boldsymbol{\tau}_\mathrm{iso}$ for Mooney-Rivlin model is transformed from Equation \ref{Siso1}:
\begin{equation} \label{Tauiso1}
\begin{split}
\boldsymbol{\tau}_\mathrm{iso} &= \bold{F}\bold{S}_\mathrm{iso}{\bold{F}}^T \\
	   			    &= \bold{F}  J^{-2/3} \left[    - \frac{1}{3}(\mu_1\bar{I_1} + 2\mu_2\bar{I_2}) \overline{\bold{C}}^{-1}  + (\mu_1 + \mu_2\bar{I_1})\bold{I} - \mu_2\overline{\bold{C}} \right]   \bold{F}^T 
\\
				    &= - \frac{1}{3}(\mu_1\bar{I_1} + 2\mu_2\bar{I_2})\bold{I} - \mu_2{\overline{\bold{B}}}^2 + (\mu_1 + \mu_2\bar{I_1}){\overline{\bold{B}}}
\end{split}
\end{equation}
where $\overline{\bold{B}}$ is the modified right Cauchy-Green tensor defined as $\overline{\bold{B}} = \overline{\bold{F}}\overline{\bold{F}}^T = \overline{\bold{C}}^T$, and the isochoric spatial tensor of elasticity $\mathbb{c}_\mathrm{iso}$ is obtained from Equation \ref{Ciso1}:

\begin{equation} \label{spatial_ciso1}
\begin{split}
(c_\mathrm{iso})_{ijkl} = {}&  F_{iI}F_{jJ}F_{kK}F_{lL}(C_\mathrm{iso})_{IJKL} \\
= {} &
2\mu_2 \left[{\overline{\bold{B}}}_{ij} {\overline{\bold{B}}}_{kl} - \frac{1}{2}({\overline{\bold{B}}}_{ik}{\overline{\bold{B}}}_{jl} + {\overline{\bold{B}}}_{il}{\overline{\bold{B}}}_{jk}) \right] 
- \frac{2}{3}(\mu_1 + 2\mu_2\bar{I_1})({\overline{\bold{B}}}_{ij}\delta_{kl} + {\overline{\bold{B}}}_{kl}\delta_{ij}) 
+ \frac{4}{3}\mu_2({\overline{\bold{B}}}^2_{ij}\delta_{kl} + {\overline{\bold{B}}}^2_{kl}\delta_{ij}) \\
&+
\frac{2}{9}(\mu_1\bar{I_1} + 4\mu_2\bar{I_2}) \delta_{ij}\delta_{kl}
+ \frac{1}{3}(\mu_1\bar{I_1} + 2\mu_2\bar{I_2})(\delta_{ik}\delta_{jl} + \delta_{il}\delta_{jk})
\end{split}
\end{equation} 

Similarly, the isochoric stress $\boldsymbol{\tau}_{iso}$ and the isochoric spatial tensor of elasticity $\mathbb{c}_{iso}$ for Yeoh model are transformed from Equations \ref{Siso2} and \ref{Ciso2}:

\begin{equation} \label{Tauiso2}
\begin{split}
\boldsymbol{\tau}_\mathrm{iso} 
&= \bold{F}\bold{S}_\mathrm{iso}{\bold{F}}^T \\
&= \bold{F}  J^{-2/3}[2c_1 + 4c_2(\bar{I_1} - 3) + 6c_3(\bar{I_1} - 3)^2] \left(\bold{I} - \frac{1}{3}\bar{I_1}{\overline{\bold{C}}}^{-1} \right)  \bold{F}^T \\
&= [2c_1 + 4c_2(\bar{I_1} - 3) + 6c_3(\bar{I_1} - 3)^2] \left(\overline{\bold{B}} - \frac{1}{3}\bar{I}_1\bold{I} \right)
\end{split}
\end{equation}

\begin{equation} \label{spatial_ciso2}
\begin{split}
(c_\mathrm{iso})_{ijkl} = {} &  F_{iI}F_{jJ}F_{kK}F_{lL}(C_\mathrm{iso})_{IJKL} \\
= {}& 
[8c_2+24c_3({\bar{I}}_1 - 3)] {\overline{\bold{B}}}_{ij}{\overline{\bold{B}}}_{kl} 
- \left[\frac{4}{3}c_1 + c_2\left(\frac{16}{3}{\bar{I}}_1 - 8\right) + 12c_3({\bar{I}}_1 - 1)({\bar{I}}_1 - 3)\right]({\overline{\bold{B}}}_{ij}\delta_{kl} + {\overline{\bold{B}}}_{kl}\delta_{ij}) \\
&+ \left[\frac{4}{9}c_1{\bar{I}}_1 + c_2\left(\frac{16}{9}{{\bar{I}}_1}^2 - \frac{8}{3}{\bar{I}}_1\right) + 4c_3{\bar{I}}_1({\bar{I}}_1 - 1)({\bar{I}}_1 - 3)\right] \delta_{ij}\delta_{kl} \\
&+ \left[\frac{2}{3}c_1{\bar{I}}_1 + \frac{4}{3}c_2({\bar{I}}_1 - 3){\bar{I}}_1 + 2c_3({\bar{I}}_1 - 3)^2{\bar{I}}_1\right](\delta_{ik}\delta_{jl} + \delta_{il}\delta_{jk})
\end{split}
\end{equation} 

For the HGO model, we will first derive the isotropic parts of the isochoric Kirchhoff stress and the isochoric spatial tensor of elasticity by setting $\mu_2$ to $0$ in Equations \ref{Tauiso1} and \ref{spatial_ciso1}, respectively:

\begin{equation} \label{Tauisotropic}
\boldsymbol{\tau}_\mathrm{isotropic} = - \frac{1}{3}\mu_1\bar{I_1}\bold{I} + \mu_1{\overline{\bold{B}}}
\end{equation}

\begin{equation} \label{spatial_cisotropic}
(c_\mathrm{isotropic})_{ijkl} 
= - \frac{2}{3}\mu_1({\overline{\bold{B}}}_{ij}\delta_{kl} + {\overline{\bold{B}}}_{kl}\delta_{ij}) 
+ \frac{2}{9}\mu_1\bar{I_1}  \delta_{ij}\delta_{kl}
+ \frac{1}{3}\mu_1\bar{I_1} (\delta_{ik}\delta_{jl} + \delta_{il}\delta_{jk})
\end{equation} 
Next we derive the anisotropic part of the isochoric Kirchhoff stress by substituting Equation \ref{Sanisotropic} into Equation \ref{current_stress}:

\begin{equation} \label{Tauiso3}
\begin{split}
\boldsymbol{\tau}_\mathrm{aniso} 
&= \bold{F}\bold{S}_\mathrm{aniso}{\bold{F}}^T \\
&= \bold{F} 2J^{-2/3} \left[\sum_{i = 4, 6}\frac{\partial{\Psi_\mathrm{aniso}}}{\partial{\bar{I}_i}}  \left(\bold{A}_{0i} - \frac{1}{3}\bar{I}_i\overline{\bold{C}}^{-1} \right)\right] {\bold{F}}^T \\
&= 2\sum_{i = 4, 6}\left[\frac{\partial{\Psi_\mathrm{aniso}}}{\partial{\bar{I}_i}} \left({\overline{\bold{F}}}\bold{A}_{0i}{\overline{\bold{F}}}^T - \frac{1}{3}\bar{I}_i{\overline{\bold{F}}}\overline{\bold{C}}^{-1}{\overline{\bold{F}}}^T \right)\right] \\
&= 2\sum_{i = 4, 6}\left[\frac{\partial{\Psi_\mathrm{aniso}}}{\partial{\bar{I}_i}}\bar{I}_i \left(\bold{A}_{i} - \frac{1}{3}\bold{I} \right)\right]
\end{split}
\end{equation}
where $\bold{A}_i$ is dyadic product of the deformed fiber vectors defined as $\bold{A}_i = \bold{a}_i \otimes \bold{a}_i$. 
The anisotropic part of the isochoric spatial tensor of elasticity is obtained by substituting Equation \ref{Ciso31} into Equation \ref{current_elasticity}:

\begin{equation} \label{spatial_ciso3}
\begin{split}
(c_\mathrm{aniso})_{ijkl} &=  F_{iI}F_{jJ}F_{kK}F_{lL}(C_\mathrm{aniso})_{IJKL} \\
&= 
 \sum_{i = 4, 6} \biggl[4{\bar{I}_i}^2\frac{\partial^2\Psi_\mathrm{aniso}}{\partial{\bar{I}_i}^2} \bold{A}_{i} \otimes \bold{A}_{i} - \frac{4}{3}\bar{I}_i\left(\bar{I}_i\frac{\partial^2\Psi_\mathrm{aniso}}{\partial{\bar{I}_i}^2} + \frac{\partial\Psi_\mathrm{aniso}}{\partial{\bar{I}_i}}\right)
 (\bold{A}_{i} \otimes \bold{I} + \bold{I} \otimes \bold{A}_i) \\
&+ \frac{4}{9}\left({\bar{I}_i}^2\frac{\partial^2\Psi_\mathrm{aniso}}{\partial{\bar{I}_i}^2} + \bar{I}_i\frac{\partial\Psi_\mathrm{aniso}}{\partial{\bar{I}_i}}\right)\bold{I} \otimes \bold{I} 
+ \frac{4}{3}\bar{I}_i \frac{\partial\Psi_\mathrm{aniso}}{\partial{\bar{I}_i}} \mathbb{I} \biggr]
\end{split}
\end{equation}
Therefore the isochoric Kirchhoff stress is obtained by adding up Equations \ref{Tauisotropic} and \ref{Tauiso3}, and the isochoric spatial tensor of elasticity is obtained by adding up Equations \ref{spatial_cisotropic} and \ref{spatial_ciso3}, respectively:

\begin{equation} \label{HGO_tauiso}
\boldsymbol{\tau}_\mathrm{iso} =  - \frac{1}{3}\mu_1\bar{I_1}\bold{I} + \mu_1{\overline{\bold{B}}}
+ 2\sum_{i = 4, 6}\left[\frac{\partial{\Psi_\mathrm{aniso}}}{\partial{\bar{I}_i}}\bar{I}_i (\bold{A}_{i} - \frac{1}{3}\bold{I})\right]
\end{equation}
 
\begin{equation} \label{HGO_spatial_ciso}
\begin{split}
(c_\mathrm{iso})_{ijkl} = {} &  - \frac{2}{3}\mu_1({\overline{\bold{B}}}_{ij}\delta_{kl} + {\overline{\bold{B}}}_{kl}\delta_{ij}) 
+ \frac{2}{9}\mu_1\bar{I_1}  \delta_{ij}\delta_{kl}
+ \frac{1}{3}\mu_1\bar{I_1} (\delta_{ik}\delta_{jl} + \delta_{il}\delta_{jk}) \\
&+ 
\sum_{i = 4, 6} \biggl[4{\bar{I}_i}^2\frac{\partial^2\Psi_\mathrm{aniso}}{\partial{\bar{I}_i}^2} \bold{A}_{i} \otimes \bold{A}_{i} - \frac{4}{3}\bar{I}_i\left(\bar{I}_i\frac{\partial^2\Psi_\mathrm{aniso}}{\partial{\bar{I}_i}^2} + \frac{\partial\Psi_\mathrm{aniso}}{\partial{\bar{I}_i}}\right)
 (\bold{A}_{i} \otimes \bold{I} + \bold{I} \otimes \bold{A}_i) \\
&+ 
\frac{4}{9}\left({\bar{I}_i}^2\frac{\partial^2\Psi_\mathrm{aniso}}{\partial{\bar{I}_i}^2} + \bar{I}_i\frac{\partial\Psi_\mathrm{aniso}}{\partial{\bar{I}_i}}\right)\bold{I} \otimes \bold{I} 
+ \frac{4}{3}\bar{I}_i \frac{\partial\Psi_\mathrm{aniso}}{\partial{\bar{I}_i}} \mathbb{I} \biggr]
\end{split}
\end{equation}

Using the total Lagrangian formulation introduced in Section \ref{formulation}, the linearization of the internal work is then transforming Equation \ref{Kint} into the following equation written in the current configuration:

\begin{equation}
D^2_{\delta\bold{u}, \Delta\bold{u}}\Pi_\mathrm{int}(\bold{u}) = D_{\Delta{\bold{u}}}\delta{W_\mathrm{int}}(\bold{u}, \delta{\bold{u}}) = \int_{\Omega_0}(\nabla_{\bold{x}}\delta\bold{u} : \nabla_{\bold{x}}\Delta\bold{u}\boldsymbol{\tau} + \nabla_{\bold{x}}\delta{\bold{u}} : \mathbb{c} :  \nabla_{\bold{x}}\Delta\bold{u})dV
\end{equation}


%
\section{Summary of Formulas}
Table \ref{summary1} lists the stress and elasticity tensors for Mooney-Rivlin, Yeoh and HGO models in the reference configuration. Accordingly Table \ref{summary2} lists the stress and elasticity tensors for those models in the current configuration.

\begin{table}[H]
\centering
\caption{Formulas in the Reference Configuration}
\label{summary1}
\begin{tabular}{l c c c c} 
\toprule
\multirow{2}{*}{Model}  &  \multicolumn{2}{c}{Stress tensor} &  \multicolumn{2}{c}{Elasticity tensor} \\
\cmidrule(l){2-5}
& Isochoric  & Volumetric &  Isochoric  & Volumetric  \\
\midrule
Mooney-Rivlin  & \ref{Siso1} & \multirow{3}{*}{\ref{Svol11} or \ref{Svol12}} & \ref{Ciso1} & \multirow{3}{*}{\ref{Cvol11} or \ref{Cvol12}} \\
Yeoh & \ref{Siso2} & & \ref{Ciso2} &\\
HGO & \ref{HGOSiso} & & \ref{Ciso3} &\\ 
\bottomrule
\end{tabular}
\end{table}

\begin{table}[H]
\centering
\caption{Formulas in the Current Configuration}
\label{summary2}
\begin{tabular}{l c c c c} 
\toprule
\multirow{2}{*}{Model}  &  \multicolumn{2}{c}{Stress tensor} &  \multicolumn{2}{c}{Elasticity tensor} \\
\cmidrule(l){2-5}
& Isochoric  & Volumetric &  Isochoric  & Volumetric  \\
\midrule
Mooney-Rivlin  & \ref{Tauiso1} & \multirow{3}{*}{\ref{Tauvol11} or \ref{Tauvol12}} & \ref{spatial_ciso1} & \multirow{3}{*}{\ref{spatial_cvol11} or \ref{spatial_cvol12}} \\
Yeoh & \ref{Tauiso2} & & \ref{spatial_ciso2} &\\
HGO & \ref{HGO_tauiso} & & \ref{HGO_spatial_ciso} &\\ 
\bottomrule
\end{tabular}
\end{table}

\end{appendices}
